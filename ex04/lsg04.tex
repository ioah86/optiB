\documentclass[a4paper]{article}
\usepackage[utf8]{inputenc}
\usepackage{fancyhdr}
\usepackage{amsmath}
\usepackage[ngerman]{babel}
\usepackage{amsthm}
\usepackage{tikz}
\usepackage{listings}
\usepackage{fullpage}

\lstset{numbers=left, basicstyle=\ttfamily, numberstyle=\tiny, mathescape=true} %listing style für code

\usetikzlibrary{positioning}
\pagestyle{fancy}

\usetikzlibrary{automata, positioning, arrows, calc}

\setlength{\parindent}{0pt}
\setlength{\parskip}{1ex}

% Kopfzeile
\lhead{Optimierung B}
\chead{Übung 4}
\rhead{Niklas Fischer 298418 \\ Gereon Kremer 288911}

% Fußzeile
\lfoot{}
\cfoot{Seite \thepage{}}
\rfoot{}

\renewcommand{\thesection}{}
\renewcommand{\thesubsection}{(\alph{subsection})}

\begin{document}

\section{Aufgabe 1}
\subsection{}


\subsection{}
Wir wissen, dass ein $k$-regulärer bipartiter Graph mit $k \geq 1$ ein perfektes
Matching besitzt. Zeige die Aussage durch Induktion.

Für $k = 1$ existiert ein perfektes Matching. da jeder Knoten nur genau eine
inzidente Kante hat ($k$-regulär), kann kein anderes, kantendisjunktes
Matching existieren. Die Aussage gilt also für $k = 1$.

Die Aussage gelte nun für $n$.
Entfernt man aus einem $n+1$-regulären bipartiten Graphen die Kanten eines perfekten
Matchings, so wird der Grad jedes Knotens um $1$ verringert. Es ergibt sich
also ein $n$-regulärer bipartiter Graph, der nach Vorraussetzung $n$ kantendisjunkte
Matchings besitzt. Der $n+1$-reguläre bipartite Graph besitzt ein weiteres Matching,
das soeben entfernt wurde um den $n$-regulären bipartiten Graph zu erhalten.

Die Aussage gilt somit für alle $k$ durch vollständige Induktion.

\subsection{}
Für gerade $k$:

Sei $G$ die $k+1$-Clique. Jeder Knoten in $G$ hat dann $k$ Kanten, da er mit
jedem anderen Knoten verbunden ist, der Graph also $k$-regulär. Die 
Gesamtanzahl der Knoten ist allerdings ungerade, was ein perfektes Matching
bereits unmöglich macht.

Für ungerade $k$:

Sei $G_i$ mit $i = 1, ..., k$ eine Familie von Graphen.
Jedes $G_i$ ist $k$-regulär bis auf einen ausgezeichneten Knoten $v_{i,
in}$, der nur $k-1$ inzidente Kanten besitzt. Die Existenz solcher Graphen
wird im folgenden gezeigt.

Aus diesen $G_i$ lässt sich nun $G$ erzeugen, indem ein weiterer Knoten
$v_c$ eingefügt wird, der mit einer Kante mit allen $v_{i, in}$ verbunden
wird. Alle $v_{i, in}$ haben dann, ebenso wie $v_c$, $k$ viele inzidente
Kanten und der Graph $G$ ist $k$-regulär.

Beispiel für $k = 5$.

\begin{tikzpicture}[node distance=2cm]
	\node (vc) {$v_c$};
	\node (g1) [above left of=vc] {$G_1$};
	\node (g2) [left of=vc] {$G_2$};
	\node (g3) [below of=vc] {$G_3$};
	\node (g4) [right of=vc] {$G_4$};
	\node (g5) [above right of=vc] {$G_5$};
	\path [draw]
		(vc) -- (g1) (vc) -- (g2) (vc) -- (g3) (vc) -- (g4) (vc) -- (g5)
	;
\end{tikzpicture}

Die Graphen $G_i$ können wie folgt konstruiert werden:
$G_i = (V_i, E_i)$ mit
\[
	V_i = \{ v_{i,in}, v_{i, 1}, ..., v_{i, k-1}, v_{i, t_1}, v_{i, t_2} \}
\]
und
\begin{align*}
E_i &= \{ \{ v_{i, in}, v_{i, j} \} \mid j \in \{ 1, ..., k-1 \} \} \\
	&\cup \{ \{ v_{i, t_1}, v_{i, j} \}, \{ v_{i, t_2}, v_{i, j} \} \mid j \in \{ i, ..., k-1 \} \} \\
	&\cup E_{k-3\text{-reg}}
\end{align*}
$E_{k-3\text{-reg}}$ ist dabei eine Menge von Kanten, so dass der Graph
\[
	G_{k-3\text{-reg}} = (\{v_{i, 1}, ..., v_{i, k-1}\}, E_{k-3\text{-reg}})
\]
$k-3$-regulär ist.

Beispiel eines $G_i$ für $k = 7$. Der graue Bereich soll den $k-3$-regulären Teilgraphen andeuten, 
in dem weitere Kanten existieren.

\begin{tikzpicture}[node distance=1cm]
	\node (vin) {$v_{i,in}$};
	\node (v3) [below left=2cm of vc] {$v_{i,3}$};
	\node (v2) [left of=v3] {$v_{i,2}$};
	\node (v1) [left of=v2] {$v_{i,1}$};
	\node (v4) [below right=2cm of vc] {$v_{i,4}$};
	\node (v5) [right of=v4] {$v_{i,5}$};
	\node (v6) [right of=v5] {$v_{i,6}$};
	\node (vend1) [below=2cm of v3] {$v_{i, t_1}$};
	\node (vend2) [below=2cm of v4] {$v_{i, t_2}$};
	\path [draw]
		(vin) -- (v1) (vin) -- (v2) (vin) -- (v3) (vin) -- (v4) (vin) -- (v5) (vin) -- (v6)
		(vend1) -- (v1) (vend1) -- (v2) (vend1) -- (v3) (vend1) -- (v4) (vend1) -- (v5) (vend1) -- (v6)
		(vend2) -- (v1) (vend2) -- (v2) (vend2) -- (v3) (vend2) -- (v4) (vend2) -- (v5) (vend2) -- (v6)
	;
	\draw [fill=black, opacity=0.2] ($0.5*(v3)+0.5*(v4)$) ellipse [x radius=5cm, y radius=1cm];
\end{tikzpicture}

Für diesen $k-3$-regulären Teilgraphen kann beispielsweise eine $k-1$-Clique
ohne die Kanten
\[
	E_- = \{\{ v_j, v_{k-j} \} \mid j \in \{1, ..., \frac{k-1}{2}\}\}
\]
sein, also 
\[
	G_{k-3\text{-reg}} = (\{v_{i, 1}, ..., v_{i, k-1}\}, \{ \{ v_j, v_l\} \mid j \neq l \wedge j+l \neq k\})
\]
Der so entstandene Graph $G_i = (V_i, E_i)$ erfüllt die an $G_i$ gestellten Bedingungen.

Beispiel für $k=7$.

\begin{tikzpicture}[node distance=2cm]
	\node (v1) {$v_{i,1}$};
	\node (v2) [right of=v1] {$v_{i,2}$};
	\node (v3) [below right of=v2] {$v_{i,3}$};
	\node (v4) [below left of=v3] {$v_{i,4}$};
	\node (v6) [below left of=v1] {$v_{i,6}$};
	\node (v5) [below right of=v6] {$v_{i,5}$};
	\path [draw]
		(v1) -- (v2) (v1) -- (v3) (v1) -- (v4) (v1) -- (v5)
		(v2) -- (v3) (v2) -- (v4) (v2) -- (v6)
		(v3) -- (v2) (v3) -- (v5) (v3) -- (v6)
		(v4) -- (v5) (v4) -- (v6)
		(v5) -- (v6)
	;
\end{tikzpicture}

Jeder Graph $G_i$ hat genau $k+2$, also ungerade viele Knoten. Bei einem
Matching innerhalb von $G_i$ bleibt also stets ein Knoten übrig.
Die einzige Möglichkeit ein perfektes Matching zu erreichen besteht darin,
dass dieser übrige Knoten $v_{i, in}$ ist. So bleiben jedoch alle $v_{i,
in}$ sowie $v_c$ übrig. Da in dem durch diese Knoten induzierten Teilgraphen
alle $v_{i_in}$ jedoch nur genau mit $v_c$ verbunden sind, kann nur eine
weitere Matchingkante zum Matching hinzugefügt werden. Die anderen $v_{i,
in}$ bleiben als isolierte Knoten zurück. 
Daher kann kein perfektes Matching gefunden werden.

\section{Aufgabe 2}

Wähle $u_i$ mit kleinstem $i$.

Wähle Pfad anschließend so, dass das $w_k$ mit kleinstem $k$ herauskommt.

\begin{tikzpicture}[node distance=2cm, line width=1pt]
\node (u1) {$u_1$};
\node (u2) [right of=u1] {$u_2$};
\node (u3) [right of=u2] {$u_3$};
\node (u4) [right of=u3] {$u_4$};
\node (u5) [right of=u4] {$u_5$};
\node (u6) [right of=u5] {$u_6$};
\node (u7) [right of=u6] {$u_7$};
\node (u8) [right of=u7] {$u_8$};

\node (w1) [below of=u1] {$w_1$};
\node (w2) [right of=w1] {$w_2$};
\node (w3) [right of=w2] {$w_3$};
\node (w4) [right of=w3] {$w_4$};
\node (w5) [right of=w4] {$w_5$};
\node (w6) [right of=w5] {$w_6$};
\node (w7) [right of=w6] {$w_7$};
\node (w8) [right of=w7] {$w_8$};
\node (w9) [right of=w8] {$w_9$};

\path
(u1) edge (w1) (u1) edge (w3)
(u2) edge (w1) (u2) edge (w5) (u2) edge (w6) (u2) edge (w7)
(u3) edge (w1) (u3) edge (w3)
(u4) edge (w1) (u4) edge (w2)
(u5) edge (w8) (u5) edge (w9) (u5) edge [red] (w5)
(u6) edge (w2)
(u7) edge (w8) (u7) edge (w9)
(u8) edge (w9) (u8) edge (w8) (u8) edge (w7) (u8) edge (w6) (u8) edge (w5) (u8) edge (w4)
;
\end{tikzpicture}

$u_1 w_1$

\begin{tikzpicture}[node distance=2cm, line width=1pt]
\node (u1) {$u_1$};
\node (u2) [right of=u1] {$u_2$};
\node (u3) [right of=u2] {$u_3$};
\node (u4) [right of=u3] {$u_4$};
\node (u5) [right of=u4] {$u_5$};
\node (u6) [right of=u5] {$u_6$};
\node (u7) [right of=u6] {$u_7$};
\node (u8) [right of=u7] {$u_8$};

\node (w1) [below of=u1] {$w_1$};
\node (w2) [right of=w1] {$w_2$};
\node (w3) [right of=w2] {$w_3$};
\node (w4) [right of=w3] {$w_4$};
\node (w5) [right of=w4] {$w_5$};
\node (w6) [right of=w5] {$w_6$};
\node (w7) [right of=w6] {$w_7$};
\node (w8) [right of=w7] {$w_8$};
\node (w9) [right of=w8] {$w_9$};

\path
(u1) edge [red] (w1) (u1) edge (w3)
(u2) edge (w1) (u2) edge (w5) (u2) edge (w6) (u2) edge (w7)
(u3) edge (w1) (u3) edge (w3)
(u4) edge (w1) (u4) edge (w2)
(u5) edge (w8) (u5) edge (w9) (u5) edge [red] (w5)
(u6) edge (w2)
(u7) edge (w8) (u7) edge (w9)
(u8) edge (w9) (u8) edge (w8) (u8) edge (w7) (u8) edge (w6) (u8) edge (w5) (u8) edge (w4)
;
\end{tikzpicture}

$u_2 w_1 u_1 w_3$

\begin{tikzpicture}[node distance=2cm, line width=1pt]
\node (u1) {$u_1$};
\node (u2) [right of=u1] {$u_2$};
\node (u3) [right of=u2] {$u_3$};
\node (u4) [right of=u3] {$u_4$};
\node (u5) [right of=u4] {$u_5$};
\node (u6) [right of=u5] {$u_6$};
\node (u7) [right of=u6] {$u_7$};
\node (u8) [right of=u7] {$u_8$};

\node (w1) [below of=u1] {$w_1$};
\node (w2) [right of=w1] {$w_2$};
\node (w3) [right of=w2] {$w_3$};
\node (w4) [right of=w3] {$w_4$};
\node (w5) [right of=w4] {$w_5$};
\node (w6) [right of=w5] {$w_6$};
\node (w7) [right of=w6] {$w_7$};
\node (w8) [right of=w7] {$w_8$};
\node (w9) [right of=w8] {$w_9$};

\path
(u1) edge (w1) (u1) edge [red] (w3)
(u2) edge [red] (w1) (u2) edge (w5) (u2) edge (w6) (u2) edge (w7)
(u3) edge (w1) (u3) edge (w3)
(u4) edge (w1) (u4) edge (w2)
(u5) edge (w8) (u5) edge (w9) (u5) edge [red] (w5)
(u6) edge (w2)
(u7) edge (w8) (u7) edge (w9)
(u8) edge (w9) (u8) edge (w8) (u8) edge (w7) (u8) edge (w6) (u8) edge (w5) (u8) edge (w4)
;
\end{tikzpicture}

$u_3 w_1 u_2 w_6$

\begin{tikzpicture}[node distance=2cm, line width=1pt]
\node (u1) {$u_1$};
\node (u2) [right of=u1] {$u_2$};
\node (u3) [right of=u2] {$u_3$};
\node (u4) [right of=u3] {$u_4$};
\node (u5) [right of=u4] {$u_5$};
\node (u6) [right of=u5] {$u_6$};
\node (u7) [right of=u6] {$u_7$};
\node (u8) [right of=u7] {$u_8$};

\node (w1) [below of=u1] {$w_1$};
\node (w2) [right of=w1] {$w_2$};
\node (w3) [right of=w2] {$w_3$};
\node (w4) [right of=w3] {$w_4$};
\node (w5) [right of=w4] {$w_5$};
\node (w6) [right of=w5] {$w_6$};
\node (w7) [right of=w6] {$w_7$};
\node (w8) [right of=w7] {$w_8$};
\node (w9) [right of=w8] {$w_9$};

\path
(u1) edge (w1) (u1) edge [red] (w3)
(u2) edge (w1) (u2) edge (w5) (u2) edge [red] (w6) (u2) edge (w7)
(u3) edge [red] (w1) (u3) edge (w3)
(u4) edge (w1) (u4) edge (w2)
(u5) edge (w8) (u5) edge (w9) (u5) edge [red] (w5)
(u6) edge (w2)
(u7) edge (w8) (u7) edge (w9)
(u8) edge (w9) (u8) edge (w8) (u8) edge (w7) (u8) edge (w6) (u8) edge (w5) (u8) edge (w4)
;
\end{tikzpicture}

$u_4 w_2$

\begin{tikzpicture}[node distance=2cm, line width=1pt]
\node (u1) {$u_1$};
\node (u2) [right of=u1] {$u_2$};
\node (u3) [right of=u2] {$u_3$};
\node (u4) [right of=u3] {$u_4$};
\node (u5) [right of=u4] {$u_5$};
\node (u6) [right of=u5] {$u_6$};
\node (u7) [right of=u6] {$u_7$};
\node (u8) [right of=u7] {$u_8$};

\node (w1) [below of=u1] {$w_1$};
\node (w2) [right of=w1] {$w_2$};
\node (w3) [right of=w2] {$w_3$};
\node (w4) [right of=w3] {$w_4$};
\node (w5) [right of=w4] {$w_5$};
\node (w6) [right of=w5] {$w_6$};
\node (w7) [right of=w6] {$w_7$};
\node (w8) [right of=w7] {$w_8$};
\node (w9) [right of=w8] {$w_9$};

\path
(u1) edge (w1) (u1) edge [red] (w3)
(u2) edge (w1) (u2) edge (w5) (u2) edge [red] (w6) (u2) edge (w7)
(u3) edge [red] (w1) (u3) edge (w3)
(u4) edge (w1) (u4) edge [red] (w2)
(u5) edge (w8) (u5) edge (w9) (u5) edge [red] (w5)
(u6) edge (w2)
(u7) edge (w8) (u7) edge (w9)
(u8) edge (w9) (u8) edge (w8) (u8) edge (w7) (u8) edge (w6) (u8) edge (w5) (u8) edge (w4)
;
\end{tikzpicture}

$u_7 w_8$

\begin{tikzpicture}[node distance=2cm, line width=1pt]
\node (u1) {$u_1$};
\node (u2) [right of=u1] {$u_2$};
\node (u3) [right of=u2] {$u_3$};
\node (u4) [right of=u3] {$u_4$};
\node (u5) [right of=u4] {$u_5$};
\node (u6) [right of=u5] {$u_6$};
\node (u7) [right of=u6] {$u_7$};
\node (u8) [right of=u7] {$u_8$};

\node (w1) [below of=u1] {$w_1$};
\node (w2) [right of=w1] {$w_2$};
\node (w3) [right of=w2] {$w_3$};
\node (w4) [right of=w3] {$w_4$};
\node (w5) [right of=w4] {$w_5$};
\node (w6) [right of=w5] {$w_6$};
\node (w7) [right of=w6] {$w_7$};
\node (w8) [right of=w7] {$w_8$};
\node (w9) [right of=w8] {$w_9$};

\path
(u1) edge (w1) (u1) edge [red] (w3)
(u2) edge (w1) (u2) edge (w5) (u2) edge [red] (w6) (u2) edge (w7)
(u3) edge [red] (w1) (u3) edge (w3)
(u4) edge (w1) (u4) edge [red] (w2)
(u5) edge (w8) (u5) edge (w9) (u5) edge [red] (w5)
(u6) edge (w2)
(u7) edge [red] (w8) (u7) edge (w9)
(u8) edge (w9) (u8) edge (w8) (u8) edge (w7) (u8) edge (w6) (u8) edge (w5) (u8) edge (w4)
;
\end{tikzpicture}

$u_8 w_4$

\begin{tikzpicture}[node distance=2cm, line width=1pt]
\node (u1) {$u_1$};
\node (u2) [right of=u1] {$u_2$};
\node (u3) [right of=u2] {$u_3$};
\node (u4) [right of=u3] {$u_4$};
\node (u5) [right of=u4] {$u_5$};
\node (u6) [right of=u5] {$u_6$};
\node (u7) [right of=u6] {$u_7$};
\node (u8) [right of=u7] {$u_8$};

\node (w1) [below of=u1] {$w_1$};
\node (w2) [right of=w1] {$w_2$};
\node (w3) [right of=w2] {$w_3$};
\node (w4) [right of=w3] {$w_4$};
\node (w5) [right of=w4] {$w_5$};
\node (w6) [right of=w5] {$w_6$};
\node (w7) [right of=w6] {$w_7$};
\node (w8) [right of=w7] {$w_8$};
\node (w9) [right of=w8] {$w_9$};

\path
(u1) edge (w1) (u1) edge [red] (w3)
(u2) edge (w1) (u2) edge (w5) (u2) edge [red] (w6) (u2) edge (w7)
(u3) edge [red] (w1) (u3) edge (w3)
(u4) edge (w1) (u4) edge [red] (w2)
(u5) edge (w8) (u5) edge (w9) (u5) edge [red] (w5)
(u6) edge (w2)
(u7) edge [red] (w8) (u7) edge (w9)
(u8) edge (w9) (u8) edge (w8) (u8) edge (w7) (u8) edge (w6) (u8) edge (w5) (u8) edge [red] (w4)
;
\end{tikzpicture}

Keine weiteren M-erweiternden Pfade vorhanden.

\[
\Rightarrow M = \{ (1, 3), (2, 6), (3, 1), (4, 2), (5, 5), (7, 8), (8, 4) \}
\]

\begin{tikzpicture}[node distance=2cm, line width=1pt, auto]
\node (u1) {$u_1$};
\node (u2) [circle, draw, right of=u1] {$u_2$};
\node (u3) [right of=u2] {$u_3$};
\node (u4) [right of=u3] {$u_4$};
\node (u5) [circle, draw, right of=u4] {$u_5$};
\node (u6) [right of=u5] {$u_6$};
\node (u7) [circle, draw, right of=u6] {$u_7$};
\node (u8) [circle, draw, right of=u7] {$u_8$};

\node (w1) [circle, draw, below of=u1] {$w_1$};
\node (w2) [circle, draw, right of=w1] {$w_2$};
\node (w3) [circle, draw, right of=w2] {$w_3$};
\node (w4) [right of=w3] {$w_4$};
\node (w5) [right of=w4] {$w_5$};
\node (w6) [right of=w5] {$w_6$};
\node (w7) [right of=w6] {$w_7$};
\node (w8) [right of=w7] {$w_8$};
\node (w9) [right of=w8] {$w_9$};



\path
(u1) edge (w1) (u1) edge [red] (w3)
(u2) edge (w1) (u2) edge (w5) (u2) edge [red] (w6) (u2) edge (w7)
(u3) edge [red] (w1) (u3) edge (w3)
(u4) edge (w1) (u4) edge [red] (w2)
(u5) edge (w8) (u5) edge (w9) (u5) edge [red] (w5)
(u6) edge (w2)
(u7) edge [red] (w8) (u7) edge (w9)
(u8) edge (w9) (u8) edge (w8) (u8) edge (w7) (u8) edge (w6) (u8) edge (w5) (u8) edge [red] (w4)
;
\end{tikzpicture}

\begin{lstlisting}[caption=Algorithmus zum Berechnen des Vertex Covers]
$T' \leftarrow U \setminus \bigcup_{e \in M} e$
Wiederhole:
    $T \leftarrow T'$
    $T' \leftarrow T \cup (\{w \in W \mid \exists e \in E \setminus M : e = \{w, v\}, v \in T \})$
    $T' \leftarrow T' \cup (\{u \in U \mid \exists e \in M : e = \{u, v\}, v \in T \})$
Solange $T' \neq T$
Vertex Cover $C = (U \setminus T) \cup (W \cap T)$
\end{lstlisting}

Ausführen des Algorithmus:
\begin{align*}
T' &= \{ u_6 \} \\
T &= T' \\
T' &= \{ u_6, w_2 \} \\
T' &= \{ u_6, w_2, u_4 \} \\
T &= T' \\
T' &= \{ u_6, w_2, u_4, w_1 \} \\
T' &= \{ u_6, w_2, u_4, w_1, u_3 \} \\
T &= T' \\
T' &= \{ u_6, w_2, u_4, w_1, u_3, w_3 \} \\
T' &= \{ u_6, w_2, u_4, w_1, u_3, w_3, u_1 \} \\
T &= T' \\
(U \setminus T) \cup (W \cap T) &= \{ u_2, u_5, u_7, u_8, w_1, w_2, w_3 \} 
\end{align*}

\section{Aufgabe 3}

\section{Aufgabe 4}

Zum Beweis betrachten wir ein Verfahren, ein perfektes Matching auf einem
beliebigen Baum zu konstruieren, falls eines existiert. Das Verfahren wird
eindeutig sein. Das erzeugte Matching ist somit auch eindeutig (oder 
existiert nicht) und jeder Baum hat somit maximal ein perfektes Matching.

Betrachte folgenden Algorithmus auf dem Baum $B = (V, E)$:
\begin{enumerate}
\item Falls $|V| \mod 2 \neq 0$: Gebe Fehler aus
\item OBdA: wähle einen beliebigen Knoten des Baumes als Wurzelknoten. So
ergibt sich ein natürlicher Begriff von Blattknoten und dem Wurzelknoten.
\item Solange Knoten in $V$ existieren:
	\begin{enumerate}
	\item Wähle den Blattknoten $v$ mit maximaler Distanz zum Wurzelknoten
	\item Füge die von $v$ ausgehende Kante $e = \{v, w\}$ in das Matching ein
	\item Falls $w$ weitere Kindknoten hat: Breche mit Fehler ab
	\item Entferne $v$ und $w$ aus $V$ sowie alle zu $v$ und $w$ inzidenten
	Kanten aus $E$
	\end{enumerate}
\end{enumerate}

Behauptung: Der Baum $B$ hat ein perfektes Matching $\Leftrightarrow$ der
Algorithmus findet ein perfektes Matching auf $B$

Beweis:

$\Rightarrow$: Es existiert ein perfektes Matching auf $B$, für jeden Knoten
ist also eine inzidente Kante in $M$ enthalten. Insbesondere ist also für
den von der Wurzel am weitesten entfernten Blattknoten $v$ eine inzidente Kante
in $M$ enthalten. Ein Blattknoten hat nur eine inzidente Kante $e = \{v, w\}$, also muss
genau diese in $M$ enthalten sein.

Mit $e$ ist nun auch bereits eine zu $w$ inzidente Kante in $M$ enthalten,
es darf also keine weitere zu $w$ inzidente Kante in $M$ enthalten sein.
Sie können daher im folgenden völlig ignoriert werden, bzw aus dem Baum
gelöscht werden.

Hierbei besteht prinzipiell die Möglichkeit, dass $w$ weitere Kindknoten
besitzt und der Baum somit in Teilbäume aufgeteilt wird. Da $v$ jedoch der
Blattknoten mit der größten Distanz zur Wurzel ist, kann $w$ nur weitere
einzelne Kindknoten, nicht jedoch Kindknoten mit weiteren Kindern haben.
Ist dies der Fall, so liegt ein Fehler vor, da auch dieser andere Kindknoten
lediglich eine Kante mit $w$ gemeinsam hat, $w$ jedoch keine zwei inzidenten
Kanten in $M$ haben darf.

Der Baum wird so iterativ verkleinert und immer weitere Kanten in $M$
eingefügt. Der Restbaum besteht zu jedem Zeitpunkt aus den Knoten und
Kanten, die noch für weitere Matchingkanten zur Verfügung stehen.
Sind keine Knoten mehr vorhanden, wurden alle Knoten durch Matchingkanten
abgedeckt und ein perfektes Matching gefunden.
Dieses Verfahren entspricht genau dem Algorithmus, der daher ein perfektes
Matching findet, falls eines existiert.

$\Leftarrow$: Der Algorithmus findet ein Matching auf $B$. Da der
Algorithmus erst terminiert, wenn alle Knoten durch das Matching abgedeckt
sind, muss das Matching perfekt sein. Die Matchingeigenschaften werden durch
das Löschen der bereits abgedeckten Knoten sichergestellt.

\section{Aufgabe 5}

\end{document}
