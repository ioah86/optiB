\documentclass[a4paper]{article}
\usepackage[utf8]{inputenc}
\usepackage{fancyhdr}
\usepackage{amsmath}
\usepackage[ngerman]{babel}
\usepackage{amsthm}
\usepackage{amssymb}
\usepackage{tikz}
\usepackage{listings}
%\usepackage{fullpage}

\lstset{numbers=left, basicstyle=\ttfamily, numberstyle=\tiny, mathescape=true} %listing style für code

%\usetikzlibrary{positioning, trees, snakes}
\usetikzlibrary{automata, positioning, arrows, calc, topaths}

\setlength{\parindent}{0pt}
\setlength{\parskip}{1ex}
%\setlength{\headheight}{30pt}
\addtolength{\textwidth}{2in}
\addtolength{\textheight}{1.5in}
\addtolength{\hoffset}{-1in}
\addtolength{\voffset}{-0.75in}
\pagestyle{fancy}

\newcommand{\tvs}{\textvisiblespace}

% Kopfzeile
\lhead{Optimierung B}
\chead{Übung 10}
\rhead{Niklas Fischer 298418 \\ Gereon Kremer 288911}

% Fußzeile
\lfoot{}
\cfoot{Seite \thepage{}}
\rfoot{}

\renewcommand{\thesection}{}
\renewcommand{\thesubsection}{(\alph{subsection})}

\begin{document}

\section{Aufgabe 1}
\subsection{}
$\max 2x+y$ s.t. $x+y \leq 4$, $x \leq 3$, $x,y \geq 0$.

Duales LP: \\
$\min 4x+3y$ s.t. $x+y \geq 2$, $x \geq 1$, $x,y \geq 0$.

\subsection{}
$\max 2x+y-2z$ s.t. $x+y-z \leq 4$, $x-z \leq 3$, $x,y,z \geq 0$.

Duales LP: \\
$\min 4x+3y$ s.t. $-x-y \geq 2$, $x \geq 1$, $x+y \geq -1$, $x,y \geq 0$.

\subsection{}
$\max 2x+y$ s.t. $x+y \leq 4$, $-x-y \leq 4$, $x \leq 3$, $x,y \geq 0$.

Duales LP: \\
$\min 4x+4y+3z$, s.t. $x-y+z \geq 2$, $x-y \geq 1$, $x,y,z \geq 0$.

\section{Aufgabe 2}

\section{Aufgabe 3}
\subsection{}
$(s < 0 \Rightarrow r < 0) \wedge (t > 0 \Rightarrow r > 0)$
\subsection{}
$(s < 0 \Rightarrow r < 0) \wedge (t < 0 \wedge r \leq 0)$
\subsection{}
$(s > 0 \wedge r \geq 0) \wedge (t > 0 \Rightarrow r > 0)$
\subsection{}
$(s > 0 \wedge r \geq 0) \wedge (t < 0 \wedge r \leq 0)$

\section{Aufgabe 4}
$A$ ist nicht vollständig unimodular, denn 
$\det \begin{pmatrix} 1&1\\-1&1 \end{pmatrix} = 2 \not\in \{ -1,0,1 \}$.

Es gilt $x = (b_3, b_2 + b_3, b_1 - b_2 - 2 b_3)$, somit ist $x \in
\mathbb{Z}^3$
falls $b \in \mathbb{Z}^3$.

\section{Aufgabe 5}
\subsection{}
\subsection{}
Sei folgende balancierte $\{0,1\}$-Matrix gegeben.

$\begin{pmatrix}
1 & 0 & 1 & 0 \\
1 & 1 & 0 & 0 \\
0 & 0 & 1 & 1 \\
1 & 0 & 0 & 1
\end{pmatrix}$.

Diese Matrix ist vollständig unimodular, denn ihre Determinante ist $2$.

\end{document}
