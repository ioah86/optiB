\documentclass[a4paper]{article}
\usepackage[utf8]{inputenc}
\usepackage{fancyhdr}
\usepackage{amsmath}
\usepackage[ngerman]{babel}
\usepackage{amsthm}
\usepackage{tikz}
\pagestyle{fancy}

\usetikzlibrary{automata, positioning, arrows}

\setlength{\parindent}{0pt}
\setlength{\parskip}{1ex}

% Kopfzeile
\lhead{Optimierung B}
\chead{Übung 2}
\rhead{Niklas Fischer 298418 \\ Gereon Kremer 288911}

% Fußzeile
\lfoot{}
\cfoot{Seite \thepage{}}
\rfoot{}

\renewcommand{\thesection}{}
\renewcommand{\thesubsection}{(\alph{subsection})}

\begin{document}

\section{Aufgabe 1}

\subsection{}
Darstellung der Abhängigkeiten als Graph:

\begin{tikzpicture}[auto, node distance=2cm]
	\node (n1) {$1$};
	\node (n4) [above of=n1] {$4$};
	\node (n3) [right of=n4] {$3$};
	\node (n6) [right of=n3] {$6$};
	\node (n8) [right of=n1] {$8$};
	\node (n7) [right of=n6] {$7$};
	\node (n9) [right of=n8] {$9$};
	\node (n10) [below of=n9] {$10$};
	\node (n11) [right of=n7] {$11$};
	\node (n12) [right of=n9] {$12$};
	\node (n5) [right of=n12] {$5$};
	\node (n2) [right of=n11] {$2$};
	\path [->]
		(n4) edge (n3)
		(n1) edge (n3)
		(n1) edge [dashed, line width=1pt] (n8)
		(n3) edge (n6)
		(n8) edge (n6)
		(n8) edge [dashed, line width=1pt] (n9)
		(n8) edge (n10)
		(n6) edge (n7)
		(n9) edge [dashed, line width=1pt] (n12)
		(n10) edge (n12)
		(n7) edge (n5)
		(n7) edge (n11)
		(n12) edge [dashed, line width=1pt] (n5)
		(n11) edge (n2)
	;
\end{tikzpicture}

Der kritische Pfad ist gestrichelt markiert.
Die kritischen Aufgaben sind also $1$, $8$, $9$, $12$ und $5$.

Die minimale Zeit $t(a)$ zur Erledigung aller Aufgaben $a_i$ beträgt somit
$t(a_1) + t(a_8) + t(a_9) + t(a_12) + t(a_5) = 44 + 9 + 36 + 36 + 45 = 170$

Die folgende Tabelle zeigt für alle Aufgaben die frühesten und spätesten
Startzeitpunkte, die eine optimale Gesamtdauer ermöglichen. Die kritischen Aufgaben sind unterstrichen.

\begin{tabular}{lrrrrrrrrrrrr}
\hline
& \underline{$1$} & $2$ & $3$ & $4$ & \underline{$5$} & $6$ & $7$ & \underline{$8$} & \underline{$9$} & $10$ & $11$ & \underline{$12$} \\ 
\hline
früheste Zeit & $0$ & $137$ & $47$ & $0$ & $125$ & $70$ & $92$ & $44$ & $53$ & $53$ & $108$ & $89$ \\
späteste Zeit & $0$ & $154$ & $64$ & $17$ & $125$ & $87$ & $109$ & $44$ & $53$ & $55$ & $125$ & $89$ \\
\hline
\end{tabular}

\subsection{}
Die folgende Strategie soll iterativ zu beschleunigende Knoten nach einem
Greedy-Verfahren auswählen.

Eine Beschleunigung bringt nur Vorteile, falls dadurch tatsächlich die
Gesamtdauer verkürzt wird. Die Beschleunigung muss also auf eine Aufgabe im
kritischen Pfad angewendet werden. Alle anderen Aufgaben können also
ignoriert werden.

Viele Nachfolger
Reduktion / Kosten
Tatsächliche Veringerung

Zwei kritische Pfade? -> Tatsächliche Veringerung >= 0....

\section{Aufgabe 2}
\subsection{}
\begin{tabular}{l|rrrr}
$B$ & $f_1(b)$ & $f_2(b)$ & $f_3(b)$ \\
\hline
$1$ & $1$ & $1$ & $1$  \\
$2$ & $3$ & $2$ & $2$  \\
$3$ & $4$ & $4$ & $3$  \\
$4$ & $5$ & $5$ & $4$  \\
\end{tabular}

$\Rightarrow$ Lösung: $5$

\subsection{}

\begin{tabular}{l|rrrr}
$B$ & $f_1(b)$ & $f_2(b)$ & $f_3(b)$ \\
\hline
$1$ & $0.5$ & $0.5$ & $0.5$  \\
$2$ & $3$ & $1$ & $1$  \\
$3$ & $4$ & $4$ & $1.5$  \\
$4$ & $6$ & $4.5$ & $2$  \\
$5$ & $7$ & $5$ & $2.5$  \\
\end{tabular}

$\Rightarrow$ Lösung: $7$

\section{Aufgabe 3}

\section{Aufgabe 4}
Ein Graph ist bipartit, falls $V = V_1 \cup V_2$ mit $V_1 \cap V_2 = \emptyset$
und alle Kanten $e \in E$ die Form $e = \{v_1, v_2\}, v_1 \in V_1, v_2 \in
V_2$ haben. Dies entspricht der 2-Färbbarkeit, wobei alle Knoten aus $V_1$
die eine und alle Knoten aus $V_2$ die andere Farbe erhalten.

Behauptung: Für einen Graph, dessen Kreise alle gerade Länge haben, kann mit einem simplen Greddy-Verfahren
eine 2-Färbung bestimmt werden. Der Graph ist somit 2-Färbbar, also
bipartit.

Beweis:
Beginne mit der Menge $T_0 = \{v_i\}$, $v_i$ einem beliebigen Knoten.
Weise allen Knoten der Menge $T_0$ die Farbe $9$ zu.
Bestimme nun die Mengen $T_r = \{ v_j \mid \exists \{ v_j, v_k \} \in E, v_k
\in T_{r-1}, v_j \not\in T_{r mod 2} \cup T_{r mod 2 + 2} \cup ... \cup
T_{r-2} \}$ und weise den Knoten aus $T_r$ die Farbe $(r mod 2) + 1$ zu.
Terminiere, sobald $T_r = \emptyset$.

Falls sich so eine widerspruchsfreie Färbung ergibt -- also keine Färbung
eines Knotens durch eine andere Farbe überschrieben wird -- ist der Graph
bipartit.

Die Terminierung ergibt sich aus der Widerspruchsfreiheit: Da sich ein
Widerspruch in dem Überschreiben einer Färbung durch eine andere Färbung
ergibt, entsteht er genau dann, wenn ein Knoten zunächst in einem $T_r$ mit
geradem $r$ und anschließend in einem ungeraden $r$ (oder umgekehrt)
enthalten ist. Ist dies nie der Fall, so wird kein gefärbter Knoten nochmals
in einem späteren $T_r$ enthalten sein und der Algorithmus terminiert.

Zu zeigen bleibt, dass auf einem Graphen mit Kreisen gerader Länge keine
Widersprüche entstehen. Ein Widerspruch entsteht genau dann, falls ein
Knoten gefärbt wird, und nach einer ungeraden Anzahl von Schritten erneut
gefärbt wird. Dazu muss jedoch ein Pfad ungerader Länge existieren, der in
diesem Knoten beginnt und dort wieder endet -- also ein Kreis ungerader
Länge. Dieser existiert nach Vorraussetzung nicht.

Somit ist gezeigt, dass kein Widerspruch entstehen kann und ein solcher Kreis
stets gefärbt werden kann, der Graph also bipartit ist.

\end{document}
