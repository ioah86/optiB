\documentclass[a4paper]{article}
\usepackage[utf8]{inputenc}
\usepackage{fancyhdr}
\usepackage{amsmath}
\usepackage[ngerman]{babel}
\usepackage{amsthm}
\usepackage{tikz}
\pagestyle{fancy}

\usetikzlibrary{automata, positioning, arrows}

\setlength{\parindent}{0pt}
\setlength{\parskip}{1ex}

% Kopfzeile
\lhead{Optimierung B}
\chead{Übung 2}
\rhead{Niklas Fischer 298418 \\ Gereon Kremer 288911}

% Fußzeile
\lfoot{}
\cfoot{Seite \thepage{}}
\rfoot{}

\renewcommand{\thesection}{}
\renewcommand{\thesubsection}{(\alph{subsection})}

\begin{document}

\section{Aufgabe 1}

\subsection{}
Darstellung der Abhängigkeiten als Graph:

\begin{tikzpicture}[auto, node distance=2cm]
	\node (n1) {$1$};
	\node (n4) [above of=n1] {$4$};
	\node (n3) [right of=n4] {$3$};
	\node (n6) [right of=n3] {$6$};
	\node (n8) [right of=n1] {$8$};
	\node (n7) [right of=n6] {$7$};
	\node (n9) [right of=n8] {$9$};
	\node (n10) [below of=n9] {$10$};
	\node (n11) [right of=n7] {$11$};
	\node (n12) [right of=n9] {$12$};
	\node (n5) [right of=n12] {$5$};
	\node (n2) [right of=n11] {$2$};
	\path [->]
		(n4) edge (n3)
		(n1) edge (n3)
		(n1) edge [dashed, line width=1pt] (n8)
		(n3) edge (n6)
		(n8) edge (n6)
		(n8) edge [dashed, line width=1pt] (n9)
		(n8) edge (n10)
		(n6) edge (n7)
		(n9) edge [dashed, line width=1pt] (n12)
		(n10) edge (n12)
		(n7) edge (n5)
		(n7) edge (n11)
		(n12) edge [dashed, line width=1pt] (n5)
		(n11) edge (n2)
	;
\end{tikzpicture}

Der kritische Pfad ist gestrichelt markiert.
Die kritischen Aufgaben sind also $1$, $8$, $9$, $12$ und $5$.

Die minimale Zeit $t(a)$ zur Erledigung aller Aufgaben $a_i$ beträgt somit
$t(a_1) + t(a_8) + t(a_9) + t(a_12) + t(a_5) = 44 + 9 + 36 + 36 + 45 = 170$

Die folgende Tabelle zeigt für alle Aufgaben die frühesten und spätesten
Startzeitpunkte, die eine optimale Gesamtdauer ermöglichen. Die kritischen Aufgaben sind unterstrichen.

\begin{tabular}{lrrrrrrrrrrrr}
\hline
& \underline{$1$} & $2$ & $3$ & $4$ & \underline{$5$} & $6$ & $7$ & \underline{$8$} & \underline{$9$} & $10$ & $11$ & \underline{$12$} \\ 
\hline
früheste Zeit & $0$ & $137$ & $47$ & $0$ & $125$ & $70$ & $92$ & $44$ & $53$ & $53$ & $108$ & $89$ \\
späteste Zeit & $0$ & $154$ & $64$ & $17$ & $125$ & $87$ & $109$ & $44$ & $53$ & $55$ & $125$ & $89$ \\
\hline
\end{tabular}

\subsection{}
Die folgende Strategie soll iterativ zu beschleunigende Knoten nach einem
Greedy-Verfahren auswählen.

Eine Beschleunigung bringt nur Vorteile, falls dadurch tatsächlich die
Gesamtdauer verkürzt wird. Die Beschleunigung muss also auf eine Aufgabe im
kritischen Pfad angewendet werden. Alle anderen Aufgaben können also
ignoriert werden.

Viele Nachfolger
Reduktion / Kosten
Tatsächliche Veringerung

Zwei kritische Pfade? -> Tatsächliche Veringerung >= 0....

\end{document}
