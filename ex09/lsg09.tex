\documentclass[a4paper]{article}
\usepackage[utf8]{inputenc}
\usepackage{fancyhdr}
\usepackage{amsmath}
\usepackage[ngerman]{babel}
\usepackage{amsthm}
\usepackage{amssymb}
\usepackage{tikz}
\usepackage{listings}
%\usepackage{fullpage}

\lstset{numbers=left, basicstyle=\ttfamily, numberstyle=\tiny, mathescape=true} %listing style für code

%\usetikzlibrary{positioning, trees, snakes}
%\usetikzlibrary{automata, positioning, arrows, calc}

\setlength{\parindent}{0pt}
\setlength{\parskip}{1ex}
%\setlength{\headheight}{30pt}
\addtolength{\textwidth}{2in}
\addtolength{\textheight}{1.5in}
\addtolength{\hoffset}{-1in}
\addtolength{\voffset}{-0.75in}
\pagestyle{fancy}

\newcommand{\tvs}{\textvisiblespace}

% Kopfzeile
\lhead{Optimierung B}
\chead{Übung 9}
\rhead{Niklas Fischer 298418 \\ Gereon Kremer 288911}

% Fußzeile
\lfoot{}
\cfoot{Seite \thepage{}}
\rfoot{}

\renewcommand{\thesection}{}
\renewcommand{\thesubsection}{(\alph{subsection})}

\begin{document}

\section{Aufgabe 1}

\section{Aufgabe 2}

\section{Aufgabe 3}

Betrachte den Lauf für eine beliebige Eingabe der Länge $n$.
Sei $\alpha_i$ mit $i \in \mathbb{Z}$ das Zeichen an der $i$ten Bandposition.
Sei $\delta_k$ die $k$te Transitionsregel.
Der Unterstrich markiert die aktuelle Bandposition.


Die TM startet in $q_0$ auf einer beliebigen Eingabe
\[ \texttt{$q_0$\tvs\$INPUT\$\tvs} \]
Solange das aktuelle Zeichen kein Blank-Symbol ist,
geht die TM nach rechts, ohne etwas zu verändern ($\delta_1$, $\delta_2$).

Hat die TM das Ende des Wortes erreicht, so erscheint das erste Blanksymbol:
\[ \texttt{\tvs\$INPUT\$$q_0$\tvs} \]
Die TM wechselt nun nach $q_1$, schreibt ein \# und geht nach rechts:
\[ \texttt{\tvs\$INPUT\$\#$q_1$\tvs} \]
Ein Blank: Sie wechselt nach $q_2$, schreibt eine 1 und geht nach rechts:
\[ \texttt{\tvs\$INPUT\$\#1$q_2$\tvs} \]
Ein Blank: Sie bleibt in $q_2$ und schreibt eine $0$:
\[ \texttt{\tvs\$INPUT\$\#1$q_2$0} \]
Eine $0$: Sie wechselt nach $q_3$ und geht nach rechts:
\[ \texttt{\tvs\$INPUT\$\#10$q_3$\tvs} \]
Ein Blank: Sie schreibt eine $1$, wechselt nach $q_4$ und geht nach rechts:
\[ \texttt{\tvs\$INPUT\$\#101$q_4$\tvs} \]
Ein Blank: Sie wechselt nach $q_1$, schreibt eine $0$ und geht nach links:
\[ \texttt{\tvs\$INPUT\$\#10$q_1$10\tvs} \]
Eine $1$: Sie geht nach rechts:
\[ \texttt{\tvs\$INPUT\$\#101$q_1$0\tvs} \]
Eine $0$: Sie geht nach rechts und wechselt nach $q_5$:
\[ \texttt{\tvs\$INPUT\$\#1010$q_5$\tvs} \]
Ein Blank: Sie schreibt eine $1$:
\[ \texttt{\tvs\$INPUT\$\#1010$q_5$1} \]
Eine $1$: Sie wechselt nach $q_6$ und geht nach rechts:
\[ \texttt{\tvs\$INPUT\$\#10101$q_6$}\tvs \]
Ein Blank: Sie wechselt nach $q_3$ und schreibt eine $0$:
\[ \texttt{\tvs\$INPUT\$\#10101$q_3$}0 \]
Eine $0$: Sie wechselt nach $q_F$ und geht nach rechts:
\[ \texttt{\tvs\$INPUT\$\#101010$q_F$}\tvs \]

Die Endkonfiguration der TM ist also für jedes Eingabewort die folgende:
Sie endet im Zustand $q_F$, an die Eingabe wurden die Zeichen
\texttt{\#101010} angehängt und der Lesekopf befindet sich beim ersten Blank
hinter diesen Zeichen.

\section{Aufgabe 4}

\end{document}
