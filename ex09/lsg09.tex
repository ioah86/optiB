\documentclass[a4paper]{article}
\usepackage[utf8]{inputenc}
\usepackage{fancyhdr}
\usepackage{amsmath}
\usepackage[ngerman]{babel}
\usepackage{amsthm}
\usepackage{amssymb}
\usepackage{tikz}
\usepackage{listings}
%\usepackage{fullpage}

\lstset{numbers=left, basicstyle=\ttfamily, numberstyle=\tiny, mathescape=true} %listing style für code

%\usetikzlibrary{positioning, trees, snakes}
\usetikzlibrary{automata, positioning, arrows, calc, topaths}

\setlength{\parindent}{0pt}
\setlength{\parskip}{1ex}
%\setlength{\headheight}{30pt}
\addtolength{\textwidth}{2in}
\addtolength{\textheight}{1.5in}
\addtolength{\hoffset}{-1in}
\addtolength{\voffset}{-0.75in}
\pagestyle{fancy}

\newcommand{\tvs}{\textvisiblespace}

% Kopfzeile
\lhead{Optimierung B}
\chead{Übung 9}
\rhead{Niklas Fischer 298418 \\ Gereon Kremer 288911}

% Fußzeile
\lfoot{}
\cfoot{Seite \thepage{}}
\rfoot{}

\renewcommand{\thesection}{}
\renewcommand{\thesubsection}{(\alph{subsection})}

\begin{document}

\section{Aufgabe 1}
\subsection{}
Wir abstrahieren zunächst ein wenig und modellieren einen Hypergraphen, in
dem jede Kante zu genau drei Knoten adjazent ist. Anschließend werden wir
die Knoten so auf drei Teilmengen verteilen, dass sich Mengen wie für ein
3D-Matching gefordert ergeben und jede Kante einen Knoten in der Teilmenge
hat.

Sei eine 3-SAT-Formel $\varphi$ mit $n$ Variablen $x_i$ und $k$ Klauseln
gegeben. $N_+(x_i)$ bezeichne die Anzahl der Vorkommen von $x_i$ in
$\varphi$, $N_-(x_i)$ entsprechend die Anzahl der Vorkommen von $\neg x_i$.
Zudem sei $N(x_i) = \max\{ N_+(x_i), N_-(x_i) \}$.

Der gesamte Graph hat zunächst die folgenden Knoten:
\[
	V = \{ t_{i1},t_{i2}, f_{i1},f_{i2}, i_{i1},\dots, i_{i4} \mid i = 1,\dots, n\}
	\cup \{ c_i, c_i' \mid i = 1,\dots, k \}
	\cup \{ d_{i1},d_{i1}',d_{i2},d_{i2}' \mid i = 1,\dots, n \}
\]

Wir modellieren nun einen Teilgraphen, der für eine Variable $x_i$ aus
der 3-SAT-Formel erzwingt, dass $x_i$ entweder wahr oder falsch ist.
Hier ist der Graph für $N(x_i) = 2$ gegeben.

\begin{tikzpicture}[auto, node distance=1.5cm]

\node (ti1) [state] {$t_{i1}$};
\node (ii1) [state, below of=ti1] {$i_{i1}$};
\node (ii2) [state, right of=ii1] {$i_{i2}$};
\node (ii3) [state, below of=ii1] {$i_{i3}$};
\node (ii4) [state, right of=ii3] {$i_{i4}$};
\node (ti2) [state, below of=ii4] {$t_{i2}$};
\node (fi1) [state, left of=ii3] {$f_{i1}$};
\node (fi2) [state, right of=ii2] {$f_{i2}$};

\begin{scope}[fill opacity=0.15]
\filldraw[fill=black] ($(ti1)+(0,0.75)$)
	to[out=180,in=180] ($(ii1)+(0,-0.75)$)
	to[out=0,in=270] ($(ii2)+(0.75,0)$)
	to[out=90,in=0] ($(ti1)+(0,0.75)$);
	
\filldraw[fill=black] ($(ti2)+(0,-0.75)$)
	to[out=180,in=270] ($(ii3)+(-0.75,0)$)
	to[out=90,in=180] ($(ii4)+(0,0.75)$)
	to[out=0,in=0] ($(ti2)+(0,-0.75)$);
	
\filldraw[fill=black] ($(fi1)+(-0.75,0)$)
	to[out=90,in=180] ($(ii1)+(0,0.75)$)
	to[out=0,in=90] ($(ii3)+(0.75,0)$)
	to[out=270,in=270] ($(fi1)+(-0.75,0)$);

\filldraw[fill=black] ($(fi2)+(0.75,0)$)
	to[out=90,in=90] ($(ii2)+(-0.75,0)$)
	to[out=270,in=180] ($(ii4)+(0,-0.75)$)
	to[out=0,in=270] ($(fi2)+(0.75,0)$);
\end{scope}
\end{tikzpicture}

Für $N(x_i) = j$ werden allgemein $2 \cdot j$ solcher Dreiecke in einem Ring
verbunden, wobei jedes zweite einen $t$-Knoten (true), die anderen
entsprechend $f$-Knoten (false) erhält.

In einem Maxmimum Matching können nun entweder alle Kanten mit $t$-Knoten
oder alle Kanten mit $f$-Kanten gewählt werden.
Im ersten Fall ist der Wert der Variablen $i$ falsch im zweiten wahr.

Nun betrachten wir die Klauseln.
Sei $c_i = \{x_a, \neg x_b, x_c\}$ eine Klausel ($x_a \vee \neg x_b \vee
x_c$).
Wir modellieren einen Teilgraphen, der im Falle eines Maximum Matchings
erzwingt, dass eine Klausel erfüllt ist.

\begin{tikzpicture}[auto, node distance=1.5cm]

\node (ta1) [state] {$t_{a1}$};
\node (ci) [state, right of=ta1] {$c_i$};
\node (ci2) [state, right of=ci] {$c_i'$};
\node (fb1) [state, below of=ci] {$f_{b1}$};
\node (tc1) [state, right of=ci2] {$t_{c1}$};

\begin{scope}[fill opacity=0.15]
\filldraw[fill=black] ($(ta1)+(-0.75,0)$)
	to[out=90,in=180] ($(ci)+(0,0.75)$)
	to[out=0,in=90] ($(ci2)+(0.75,0)$)
	to[out=270,in=0] ($(ci)+(0,-0.75)$)
	to[out=180,in=270] ($(ta1)+(-0.75,0)$);

\filldraw[fill=black] ($(ci)+(-0.75,0)$)
	to[out=90,in=180] ($(ci2)+(0,0.75)$)
	to[out=0,in=90] ($(tc1)+(0.75,0)$)
	to[out=270,in=0] ($(ci2)+(0,-0.75)$)
	to[out=180,in=270] ($(ci)+(-0.75,0)$);
	
\filldraw[fill=black] ($(ci)+(-0.75,0)$)
	to[out=90,in=90] ($(ci2)+(0.75,0)$)
	to[out=270,in=0] ($(fb1)+(0,-0.75)$)
	to[out=180,in=270] ($(ci)+(-0.75,0)$);

\end{scope}
\end{tikzpicture}

Die beiden Klauselknoten $c_i$ und $c_i'$ können nur dann abgedeckt werden,
wenn eine der drei Kanten gewählt wird, gleichzeitig kann nur eine der drei
gewählt werden. Wird eine der drei gewählt, so ist damit gleichzeitig einer
der äußeren Knoten des Teilgraphen für die Variable abgedeckt, so dass dort
nur die Kanten mit $f$-Knoten (für $a$ und $c$) oder $t$-Knoten (für $b$)
gewählt werden können.

Es bleibt noch zu zeigen, dass die Knoten auf drei disjunkte Teilmengen
aufgeteilt werden können, so dass alle Kanten Knoten aus jeder Teilmenge
enthalten.

Sei die folgende Aufteilung gegeben:
\begin{align*}
X &= \{ c_i \mid i = 1, \dots, k \} \cup \{ i_{ij} \mid i=1,\dots,n, j=1,3,\dots,2 \cdot N(x_i) - 1 \} \\
Y &= \{ c_i' \mid i = 1, \dots, k \} \cup \{ i_{ij} \mid i=1,\dots,n, j=2,4,\dots,2 \cdot N(x_i) \} \\
Z &= \{ t_{ij}, f_{ij} \mid i = 1,\dots,n, j = 1,\dots,N(x_i) \} \\
\end{align*}
$X,Y,Z$ bilden eine Partition aller Knoten und jede der oben angegeben
Kanten hat je genau einen Knoten in jeder Teilmenge.

Durch diese Konstruktion können nun genau $k + \sum_{i=1}^n N(x_i)$ Kanten
für das Matching gewählt werden.

Ein Maximum Matching auf dem Graphen definiert somit, welche Variable
welchen Wahrheitswert hat und definiert für jede Klausel genau ein Literal,
das wahr ist.

Anhand einer Lösung für 3-SAT wiederum können sofort die entsprechenden
Kanten gewählt werden: Ist eine Variable wahr, werden die $f$-Knoten gewählt
und andersrum, für jede Klausel wird eine der Literale mit dem Wahrheitswert
true ausgewählt.

\subsection{}

\section{Aufgabe 2}
\subsection{}
Sei $\varphi$ eine Formel in 3-KNF mit Literalen $L$ und $k$ Klauseln und habe die Form
\[ \varphi = (x_{11} \vee x_{12} \vee x_{13}) \wedge (x_{21} \vee x_{22} \vee x_{23}) ... \]
Dabei sind $x_{ij} \in \{ l, \neg l \mid l \in L \}$.

Sei nun $G(\varphi) = (V, E, K = 2 k + |L|)$ ein Graph mit 
\[ V = \{ l, \neg l \mid L \in L \} \cup \{ x_{ij} \} \]
-- wobei die Knoten aus $L$ \emph{Literalknoten} und die Knoten $x_{ij}$ \emph{Klauselknoten} heißen --
und
\[ E = \{ (x_{i1},x_{i2}), (x_{i1},x_{i3}), (x_{i2},x_{i3}) \mid i = 1 ... k \} \cup \{ (L(x_{ij}),x_{ij}) \} \]
wobei $L(x_{ij})$ das Literal (oder die Negation des Literals) von $x_{ij}$ darstellt.
Die $i$ Dreiecke aus Klauselknoten nennen wir \emph{Cluster}.

Auf $G(\varphi)$ muss ein VertexCover der Größe $K$ gefunden werden.

Für die Formel $\varphi = (\neg b \vee a \vee \neg c) \wedge (\neg d \vee c \vee d)$ ergibt sich also der Graph

\begin{tikzpicture}[auto, node distance=1.5cm]
\node (at) [state] {$a$};
\node (af) [state, right of=at] {$\neg a$};
\node (bt) [state, below of=at] {$b$};
\node (bf) [state, right of=bt] {$\neg b$};
\node (ct) [state, below of=bt] {$c$};
\node (cf) [state, right of=ct] {$\neg c$};
\node (dt) [state, below of=ct] {$d$};
\node (df) [state, right of=dt] {$\neg d$};

\node (x11) [state, right of=af] {$x_{11}$};
\node (x12) [state, above right of=x11] {$x_{12}$};
\node (x13) [state, below right of=x11] {$x_{13}$};
\node (x21) [state, right of=df] {$x_{21}$};
\node (x22) [state, above right of=x21] {$x_{22}$};
\node (x23) [state, below right of=x21] {$x_{23}$};

\path [-]
	(at) edge (af) (bt) edge (bf) (ct) edge (cf) (dt) edge (df)
	(x11) edge (x12) (x12) edge (x13) (x13) edge (x11)
	(x21) edge (x22) (x22) edge (x23) (x23) edge (x21)
	(x11) edge (bf) (x12) edge [bend right] (at) (x13) edge (cf)
	(x21) edge (df) (x22) edge [bend left] (ct) (x23) edge [bend left] (dt)
;
\end{tikzpicture}

Behauptung: \texttt{VertexCover} hat auf $G(\varphi)$ genau dann eine Lösung, wenn $\varphi$ erfüllbar ist (\texttt{3-SAT} hat eine Lösung).

Beweis: \\
$\Leftarrow$:

Wir konstruieren ein Vertex Cover $W \subseteq V$ der Größe $K$.
Seien $L_1$ die Literale, denen in der Lösung von 3-Sat der Wahrheitswert wahr zugewiesen wird.
Dann sind alle $l \in L_1$ auch in $W$.
Wähle nun aus jeder 3-Clique, die eine Klausel beschreibt, zwei beliebige Knoten, so dass der verbleibende mit einem bereits markierten Literal-Knoten verbunden ist.
So ein Knoten muss existieren, da mindestens ein Literal (oder eine Negation eines Literals) in einer Klausel wahr sein muss, ein Knoten also mit einem bereits markierten Knoten verbunden sein muss.

$\Rightarrow$:

Beobachtung: Jedes Vertex Cover in $G(\varphi)$ muss mindestens einen Literalknoten jedes Literals sowie mindestens zwei Klauselknoten enthalten.
Ein Vertex Cover der Größe $K = 2k + |L|$ enthält somit genau einen Literalknoten jedes Literals sowie genau zwei Klauselknoten jedes Dreiecks.
Seien nun $L_1$ die Literalknoten des Vertex Covers und wie in $\Leftarrow$ die Menge der Literale mit dem Wahrheitswert wahr.
Die entsprechende Formel $\varphi$ ist dann erfüllbar, da jedem Literal ein Wahrheitswert zugewiesen ist und jede Klausel erfüllbar ist:
Jede Klausel enthält genau einen Knoten, der nicht im Vertex Cover enthalten ist, dessen Kante mit einem Literalknoten aber abgedeckt ist.
Der inzidente Literalknoten muss daher im Vertex Cover enthalten sein, das zum Klauselknoten gehörende Literal also wahr sein.
\hfill{} \qed{}

\subsection{}
Behauptung: $C \subseteq V$ ist genau dann ein \texttt{VertexCover}, wenn $I = V \setminus C$ ein \texttt{IndependentSet} ist.

Aus der Behauptung folgt sofort die Äquivalenz der Problem hinsichtlich NP-Vollständigkeit.

Beweis: \\
$\Rightarrow$:

Ist $C \subseteq V$ ein Vertex Cover, so sind alle Kanten des Graphen zu mindestens einem Knoten in $C$ adjazent.
Somit gibt es keine Kante, die zu zwei Knoten aus $I = V \setminus C$ adjazent ist.
Also gibt es keine zwei inzidenten Knoten in $I$.

$\Leftarrow$:

Ist $I \subseteq V$ ein Independent Set, so existiert keine Kante im Graphen zwischen zwei Knoten in $I$.
Jede Kante ist somit zu mindestens einem Knoten in $C = V \setminus I$ adjazent.
Also deckt $C$ alle Kanten ab.
\hfill{}\qed{}

\section{Aufgabe 3}

Betrachte den Lauf für eine beliebige Eingabe der Länge $n$.
Sei $\alpha_i$ mit $i \in \mathbb{Z}$ das Zeichen an der $i$ten Bandposition.
Sei $\delta_k$ die $k$te Transitionsregel.
Der Unterstrich markiert die aktuelle Bandposition.


Die TM startet in $q_0$ auf einer beliebigen Eingabe
\[ \texttt{$q_0$\tvs\$INPUT\$\tvs} \]
Solange das aktuelle Zeichen kein Blank-Symbol ist,
geht die TM nach rechts, ohne etwas zu verändern ($\delta_1$, $\delta_2$).

Hat die TM das Ende des Wortes erreicht, so erscheint das erste Blanksymbol:
\[ \texttt{\tvs\$INPUT\$$q_0$\tvs} \]
Die TM wechselt nun nach $q_1$, schreibt ein \# und geht nach rechts:
\[ \texttt{\tvs\$INPUT\$\#$q_1$\tvs} \]
Ein Blank: Sie wechselt nach $q_2$, schreibt eine 1 und geht nach rechts:
\[ \texttt{\tvs\$INPUT\$\#1$q_2$\tvs} \]
Ein Blank: Sie bleibt in $q_2$ und schreibt eine $0$:
\[ \texttt{\tvs\$INPUT\$\#1$q_2$0} \]
Eine $0$: Sie wechselt nach $q_3$ und geht nach rechts:
\[ \texttt{\tvs\$INPUT\$\#10$q_3$\tvs} \]
Ein Blank: Sie schreibt eine $1$, wechselt nach $q_4$ und geht nach rechts:
\[ \texttt{\tvs\$INPUT\$\#101$q_4$\tvs} \]
Ein Blank: Sie wechselt nach $q_1$, schreibt eine $0$ und geht nach links:
\[ \texttt{\tvs\$INPUT\$\#10$q_1$10\tvs} \]
Eine $1$: Sie geht nach rechts:
\[ \texttt{\tvs\$INPUT\$\#101$q_1$0\tvs} \]
Eine $0$: Sie geht nach rechts und wechselt nach $q_5$:
\[ \texttt{\tvs\$INPUT\$\#1010$q_5$\tvs} \]
Ein Blank: Sie schreibt eine $1$:
\[ \texttt{\tvs\$INPUT\$\#1010$q_5$1} \]
Eine $1$: Sie wechselt nach $q_6$ und geht nach rechts:
\[ \texttt{\tvs\$INPUT\$\#10101$q_6$\tvs} \]
Ein Blank: Sie wechselt nach $q_3$ und schreibt eine $0$:
\[ \texttt{\tvs\$INPUT\$\#10101$q_3$}0 \]
Eine $0$: Sie wechselt nach $q_F$ und geht nach rechts:
\[ \texttt{\tvs\$INPUT\$\#101010$q_F$\tvs} \]

Die Endkonfiguration der TM ist also für jedes Eingabewort die folgende:
Sie endet im Zustand $q_F$, an die Eingabe wurden die Zeichen
\texttt{\#101010} angehängt und der Lesekopf befindet sich beim ersten Blank
hinter diesen Zeichen.

\section{Aufgabe 4}

Sei $G = (V, E)$ der Graph mit $V = \{ v_i \mid i = 1 ... n \}$ und 
$W = \sum_{i = 1}^n w_i$.

Beobachtungen: \\
Ist $G$ ein Pfad mit $n$ Knoten, so gibt es nur $n-2$
(sinnvolle) Knoten, die in $V'$ enthalten sein können: Alle bis auf die
Endknoten. \\
Das Partitionieren entspricht dem Aufteilen des Pfades in ($k+1$) Abschnitte. \\
Weiterhin kann ein zusätzlicher Knoten in $V'$ nie von Nachteil
sein, o.B.d.A. wird $V'$ also stets $k$ Knoten enthalten.

Algorithmus:
Seien $x_j \in \{ 1, ..., n \}$ die $k$ Positionen der Knoten in $V'$ und
bezeichne $sum(i)$ die Summe der Kantengewichte aller Kanten im $i$ten
Abschnitt. Ein Abschnitt (oder seine Summe) heißt \emph{maximal}, falls kein
Abschnitt Kanten mit größerem Gesamtgewicht hat.

\begin{lstlisting}[language=Python]
$x_i$ = i # fuer alle i
while $sum(1)$ nicht maximal:
	j = maximaler Abschnitt
	if $x_j$ - $x_{j-1}$ == 1:
		return $x_i$
	$x_{j-1}$ = $x_{j-1}$ + 1
	compare($x_i$)
\end{lstlisting}

Bei der ersten Partitionierung sind die ersten $k$ Abschnitt minimal (je nur
eine Kante), der Rest ist im letzten Abschnitt enthalten.

Nun wird der maximale Abschnitt verkleinert, indem die erste Kante dieses
Abschnitts in den vorherigen Abschnitt verschoben wird.
Hat der maximale Abschnitt nur eine Kante, so ist die aktuelle
Partitionierung optimal: Kleiner als die größte Kante kann es nicht werden.

Sobald der erste Abschnitt maximal ist, kann die Partitionierung nicht mehr
besser werden. Die beste Partitionierung wurde somit bereits betrachtet und
ist unter den mittels \texttt{compare} miteinander verglichenen.

Da jeder der $k < n$ Schnitte maximal $n-k$ mal verschoben werden kann,
werden maximal $k \cdot (n-k) < n^2$ Schritte durchgeführt.

Die optimale Partitionierung muss betrachtet worden sein, da in jedem
Schritt versucht wird, die Zielfunktion -- den Wert des maximalen Abschnitts
-- zu verkleinern und am Ende eine besserer Zielwert nicht mehr erreicht
werden kann.

\end{document}
