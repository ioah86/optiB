\documentclass[a4paper]{article}
\usepackage[utf8]{inputenc}
\usepackage{fancyhdr}
\usepackage{amsmath}
\usepackage[ngerman]{babel}
\usepackage{amsthm}
\usepackage{tikz}
\usepackage{listings}
%\usepackage{fullpage}

\lstset{numbers=left, basicstyle=\ttfamily, numberstyle=\tiny, mathescape=true} %listing style für code

\usetikzlibrary{positioning, trees, snakes}
\usetikzlibrary{automata, positioning, arrows, calc}

\setlength{\parindent}{0pt}
\setlength{\parskip}{1ex}
%\setlength{\headheight}{30pt}
\addtolength{\textwidth}{2in}
\addtolength{\textheight}{1.5in}
\addtolength{\hoffset}{-1in}
\addtolength{\voffset}{-0.75in}
\pagestyle{fancy}

% Kopfzeile
\lhead{Optimierung B}
\chead{Übung 6}
\rhead{Niklas Fischer 298418 \\ Gereon Kremer 288911}

% Fußzeile
\lfoot{}
\cfoot{Seite \thepage{}}
\rfoot{}

\renewcommand{\thesection}{}
\renewcommand{\thesubsection}{(\alph{subsection})}

\begin{document}

\section{Aufgabe 1}

\section{Aufgabe 2}
Behauptung: Es existiert ein Algorithmus, der in einem vollständigen
bipartiten Graphen mit Präferenzlisten ein stabiles Matching findet.
Ist dieser Algorithmus korrekt, so existiert in jedem Graphen ein solches
Matching.

Der Algorithmus funktioniert wie folgt:

\begin{lstlisting}
Eingabe: $G = (H \cup D, H \times D), M = \emptyset \subseteq H \times D$
Solange $\exists h \in H: \not \exists d: (h, d) \in M$
	Sei $h$ dieses $h \in H$
	Sei $d \in D$ die fuer $h$ beste Partnerin, die $h$ nicht bereits gefragt hat
	Falls $\not \exists h' \in H: (h', d) \in M$:
		$M = M \cup \{(h, d)\}$
	Sonst Falls $d$ mag $h$ mehr als aktuellen Partner $h'$:
		$M = (M \setminus \{(h', d)\}) \cup \{(h, d)\}$
Ausgabe: $M$
\end{lstlisting}

Zu zeigen ist, dass dieser Algorithmus immer terminiert und ein stabiles Matching liefert.

Angenommen, das Matching $M$ sei instabil. Dann existieren $(h, d), (h', d') \in M$, 
so dass $h$ und $d'$ sich jeweils mehr mögen als den jeweiligen Partner.

Da $h$ seine Prioritätsliste von oben abarbeitet und $d$ auf dieser unter
$d'$ steht, muss nach dem Algorithmus $h$ zu irgendeinem Zeitpunkt $d'$ um 
eine Partnerschaft gebeten haben.

$d'$ muss diese abgelehnt haben oder später zugunsten eines für sie besseren
Partners beendet haben. Da $d'$ aber einem für sie schlechteren Partner
zugeordnet ist, hätte $d'$ dieses Angebot jedoch weder abgelehnt noch die
Partnerschaft zugunsten des schlechteren Kandidaten $h'$ aufgegeben.

Dies ist ein Widerspruch, das Matching kann also nicht instabil sein. \qed


\section{Aufgabe 4}
	Sei $G=(V,E)$ ein euklidischer Graph.
	Sei $C \subseteq E$ eine Lösung des TSPs.
	Angenommen es existieren zwei sich überschneidende Kanten $ab \in C$ (von $a \in V$ nach $b \in V$), und $cd \in C$ (von $c \in V$ nach $d \in V$). 
	
	Falls $a$, $b$, $c$, $d$ colinear sind, gilt die Aussage nicht:
	Sei $G$ ein volltändiger Graph mit $V = \{a,b,c,d\}$, und $a$, $b$, $c$, $d$ seien colinear. Dann ist der Zykel $C_1 = (ab,bc,cd,da)$ eine Lösung des TSP. Dies ist ein Widerspruch zu Behauptung.
	
	Falls $a$, $b$, $c$, $d$ collinear sind, colinear sind existiert $S$, der Schnittpunkt der beiden Kanten $ab$ und $cd$. Dann 	gilt:
	\begin{align*}
		|ab|+|cd| = (|aS| + |Sb|) + (|cS| + |Sd|) = (|aS| + |Sd|) + (|bS| + |Sc|) \stackrel*< |ad| + |bc|
	\end{align*}
	zu *: die Gleichheit würde genau dann gelten, wenn $a$, $S$ und $d$ sowie $b$, $S$ und $c$ colinear sind. Aufgrund der Konstruktion von $S$ würde das jedoch bedeuten, dass $a$, $b$, $c$, und $d$ kolinear sind. 
	
	Das Ersetzen von $ab$, und $cd$ im Zykel $C$ durch $ad$ und $cd$ und Umnummerieren von $C$ führt also zu einem kürzeren Zykel $C'$ der immer noch alle Knoten besucht. Dies ist ein Widerspruch dazu, dass $C$ Lösung des TSP ist. Also gilt die Behauptung.
\end{document}
