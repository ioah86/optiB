\documentclass[a4paper]{article}
\usepackage[utf8]{inputenc}
\usepackage{fancyhdr}
\usepackage{amsmath}
\usepackage[ngerman]{babel}
\usepackage{amsthm}
\usepackage{tikz}
\usepackage{listings}
%\usepackage{fullpage}

\lstset{numbers=left, basicstyle=\ttfamily, numberstyle=\tiny, mathescape=true} %listing style für code

\usetikzlibrary{positioning, trees, snakes}
\usetikzlibrary{automata, positioning, arrows, calc}

\setlength{\parindent}{0pt}
\setlength{\parskip}{1ex}
%\setlength{\headheight}{30pt}
\addtolength{\textwidth}{2in}
\addtolength{\textheight}{1.5in}
\addtolength{\hoffset}{-1in}
\addtolength{\voffset}{-0.75in}
\pagestyle{fancy}

% Kopfzeile
\lhead{Optimierung B}
\chead{Übung 7}
\rhead{Niklas Fischer 298418 \\ Gereon Kremer 288911}

% Fußzeile
\lfoot{}
\cfoot{Seite \thepage{}}
\rfoot{}

\renewcommand{\thesection}{}
\renewcommand{\thesubsection}{(\alph{subsection})}

\begin{document}

\section{Aufgabe 1}

Der Ford-Fulkerson-Algorithmus wählt einen flussvergrößernden Pfad per
Tiefensuche. Dabei wird also ein beliebiger, flussvergrößernder Pfad
gewählt. Betrachte nun den folgenden Ablauf von Ford-Fulkerson, wobei
jeweils die (relevanten) Restkapazitäten angegeben sind.

\begin{tikzpicture}[node distance=1.5cm, auto, baseline=-0.65ex]
	\node (s) {$s$};
	\node (a) [below right of=s] {$a$};
	\node (b) [above right of=s] {$b$};
	\node (t) [below right of=b] {$t$};
	\path [->]
		(s) edge [dashed] node [anchor=base, below left] {$k$} (a)
		(s) edge node {$k$} (b)
		(a) edge node [anchor=base, below right] {$k$} (t)
		(b) edge [dashed] node {$k$} (t)
		(a) edge [dashed] node {$1$} (b)
	;
\end{tikzpicture}
$ \rightarrow $
\begin{tikzpicture}[node distance=1.5cm, auto, baseline=-0.65ex]
	\node (s) {$s$};
	\node (a) [below right of=s] {$a$};
	\node (b) [above right of=s] {$b$};
	\node (t) [below right of=b] {$t$};
	\path [->]
		(s) edge node [anchor=base, below left] {$k-1$} (a)
		(s) edge [dashed] node {$k$} (b)
		(a) edge [dashed] node [anchor=base, below right] {$k$} (t)
		(b) edge node {$k-1$} (t)
		(b) edge [dashed] node {$1$} (a)
	;
\end{tikzpicture}
$ \rightarrow $
\begin{tikzpicture}[node distance=1.5cm, auto, baseline=-0.65ex]
	\node (s) {$s$};
	\node (a) [below right of=s] {$a$};
	\node (b) [above right of=s] {$b$};
	\node (t) [below right of=b] {$t$};
	\path [->]
		(s) edge [dashed] node [anchor=base, below left] {$k-1$} (a)
		(s) edge node {$k-1$} (b)
		(a) edge node [anchor=base, below right] {$k-1$} (t)
		(b) edge [dashed] node {$k-1$} (t)
		(a) edge [dashed] node {$1$} (b)
	;
\end{tikzpicture}
$ \rightarrow ... \rightarrow $
\begin{tikzpicture}[node distance=1.5cm, auto, baseline=-0.65ex]
	\node (s) {$s$};
	\node (a) [below right of=s] {$a$};
	\node (b) [above right of=s] {$b$};
	\node (t) [below right of=b] {$t$};
	\path [->]
		(s) edge [dashed] node [anchor=base, below left] {$1$} (a)
		(s) edge node {$0$} (b)
		(a) edge node [anchor=base, below right] {$0$} (t)
		(b) edge [dashed] node {$1$} (t)
		(b) edge [dashed] node {$1$} (a)
	;
\end{tikzpicture}
$ \rightarrow $
\begin{tikzpicture}[node distance=1.5cm, auto, baseline=-0.65ex]
	\node (s) {$s$};
	\node (a) [below right of=s] {$a$};
	\node (b) [above right of=s] {$b$};
	\node (t) [below right of=b] {$t$};
	\path [->]
		(s) edge node [anchor=base, below left] {$0$} (a)
		(s) edge node {$0$} (b)
		(a) edge node [anchor=base, below right] {$0$} (t)
		(b) edge node {$0$} (t)
		(a) edge node {$1$} (b)
	;
\end{tikzpicture}

Die Kapazitäten der äußeren Kanten sind also nach zwei Schritten von $k$
auf $k-1$ gesunken. Nach $2 \cdot k$ Schritten sind die Kapazitäten dieser
Kanten von $k$ auf $0$ gesunken. Erst in diesem letzten Schritt existiert
kein flussvergrößernder Pfad mehr. In jedem Schritt wird der Fluss um $1$
erhöht und -- da der Fluss jedes flussvergrößernden Pfads im Beispiel durch 
die mittlere Kante beschränkt ist -- daher möglicherweise $2 \cdot k$ Schritte
durchgeführt.

\section{Aufgabe 2}

\section{Aufgabe 3}

\section{Aufgabe 4}

\subsection{maximaler Fluss}
Sei $f$ ein Fluss auf dem Netzwerk $(D, s, t)$ mit
\begin{align*}
f(s, v_1) &= 7 		& f(s, v_3) &= 3 	& f(s, v_5) &= 8 	& f(s, v_6) &= 2 \\
f(v_1, v_2) &= 3 	& f(v_1, v_3) &= 4	& f(v_2, v_3) &= 1	& f(v_2, v_7) &= 0 \\
f(v_2, t) &= 5		& f(v_3, v_4) &= 8	& f(v_3, v_7) &= 0	& f(v_4, t) &= 13 \\
f(v_5, v_2) &= 3	& f(v_5, v_4) &= 5	& f(v_6, v_5) &= 0	& f(v_6, v_7) &= 2 \\
f(v_7, t) &= 2
\end{align*}
und dem Betrag $20$. Der Fluss ist maximal, da alle eingehenden Kanten von $t$
vollständig genutzt werden. Daraus ergibt sich direkt ein
\subsection{minimaler Schnitt}
Da, wie bereits gezeigt, die eingehenden Kanten von $t$ von $f$ vollständig
genutzt werden, ist $\{s, v_1, ..., v_7\}, \{t\}$ ein minimaler Schnitt.

\section{Aufgabe 5}

\end{document}
