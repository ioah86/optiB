\documentclass[a4paper]{article}
\usepackage[utf8]{inputenc}
\usepackage{fancyhdr}
\usepackage{amsmath}
\usepackage[ngerman]{babel}
\usepackage{amsthm}
\usepackage{tikz}
\usepackage{listings}
%\usepackage{fullpage}

\lstset{numbers=left, basicstyle=\ttfamily, numberstyle=\tiny, mathescape=true} %listing style für code

%\usetikzlibrary{positioning, trees, snakes}
%\usetikzlibrary{automata, positioning, arrows, calc}

\setlength{\parindent}{0pt}
\setlength{\parskip}{1ex}
%\setlength{\headheight}{30pt}
\addtolength{\textwidth}{2in}
\addtolength{\textheight}{1.5in}
\addtolength{\hoffset}{-1in}
\addtolength{\voffset}{-0.75in}
\pagestyle{fancy}

% Kopfzeile
\lhead{Optimierung B}
\chead{Übung 7}
\rhead{Niklas Fischer 298418 \\ Gereon Kremer 288911}

% Fußzeile
\lfoot{}
\cfoot{Seite \thepage{}}
\rfoot{}

\renewcommand{\thesection}{}
\renewcommand{\thesubsection}{(\alph{subsection})}

\begin{document}

\section{Aufgabe 1}

Der Ford-Fulkerson-Algorithmus wählt einen flussvergrößernden Pfad per
Tiefensuche. Dabei wird also ein beliebiger, flussvergrößernder Pfad
gewählt. Betrachte nun den folgenden Ablauf von Ford-Fulkerson, wobei
jeweils die (relevanten) Restkapazitäten angegeben sind.

\begin{tikzpicture}[node distance=1.5cm, auto, baseline=-0.65ex]
	\node (s) {$s$};
	\node (a) [below right of=s] {$a$};
	\node (b) [above right of=s] {$b$};
	\node (t) [below right of=b] {$t$};
	\path [->]
		(s) edge [dashed] node [anchor=base, below left] {$k$} (a)
		(s) edge node {$k$} (b)
		(a) edge node [anchor=base, below right] {$k$} (t)
		(b) edge [dashed] node {$k$} (t)
		(a) edge [dashed] node {$1$} (b)
	;
\end{tikzpicture}
$ \rightarrow $
\begin{tikzpicture}[node distance=1.5cm, auto, baseline=-0.65ex]
	\node (s) {$s$};
	\node (a) [below right of=s] {$a$};
	\node (b) [above right of=s] {$b$};
	\node (t) [below right of=b] {$t$};
	\path [->]
		(s) edge node [anchor=base, below left] {$k-1$} (a)
		(s) edge [dashed] node {$k$} (b)
		(a) edge [dashed] node [anchor=base, below right] {$k$} (t)
		(b) edge node {$k-1$} (t)
		(b) edge [dashed] node {$1$} (a)
	;
\end{tikzpicture}
$ \rightarrow $
\begin{tikzpicture}[node distance=1.5cm, auto, baseline=-0.65ex]
	\node (s) {$s$};
	\node (a) [below right of=s] {$a$};
	\node (b) [above right of=s] {$b$};
	\node (t) [below right of=b] {$t$};
	\path [->]
		(s) edge [dashed] node [anchor=base, below left] {$k-1$} (a)
		(s) edge node {$k-1$} (b)
		(a) edge node [anchor=base, below right] {$k-1$} (t)
		(b) edge [dashed] node {$k-1$} (t)
		(a) edge [dashed] node {$1$} (b)
	;
\end{tikzpicture}
$ \rightarrow ... \rightarrow $
\begin{tikzpicture}[node distance=1.5cm, auto, baseline=-0.65ex]
	\node (s) {$s$};
	\node (a) [below right of=s] {$a$};
	\node (b) [above right of=s] {$b$};
	\node (t) [below right of=b] {$t$};
	\path [->]
		(s) edge [dashed] node [anchor=base, below left] {$1$} (a)
		(s) edge node {$0$} (b)
		(a) edge node [anchor=base, below right] {$0$} (t)
		(b) edge [dashed] node {$1$} (t)
		(b) edge [dashed] node {$1$} (a)
	;
\end{tikzpicture}
$ \rightarrow $
\begin{tikzpicture}[node distance=1.5cm, auto, baseline=-0.65ex]
	\node (s) {$s$};
	\node (a) [below right of=s] {$a$};
	\node (b) [above right of=s] {$b$};
	\node (t) [below right of=b] {$t$};
	\path [->]
		(s) edge node [anchor=base, below left] {$0$} (a)
		(s) edge node {$0$} (b)
		(a) edge node [anchor=base, below right] {$0$} (t)
		(b) edge node {$0$} (t)
		(a) edge node {$1$} (b)
	;
\end{tikzpicture}

Die Kapazitäten der äußeren Kanten sind also nach zwei Schritten von $k$
auf $k-1$ gesunken. Nach $2 \cdot k$ Schritten sind die Kapazitäten dieser
Kanten von $k$ auf $0$ gesunken. Erst in diesem letzten Schritt existiert
kein flussvergrößernder Pfad mehr. In jedem Schritt wird der Fluss um $1$
erhöht und -- da der Fluss jedes flussvergrößernden Pfads im Beispiel durch 
die mittlere Kante beschränkt ist -- daher möglicherweise $2 \cdot k$ Schritte
durchgeführt.

\section{Aufgabe 2}

\section{Aufgabe 3}

Siehe Corollary 14.6 aus der Vorlesung.
Der Beweis lautet:

Sei $D' = D \setminus \{s, t\}$, $S = N^+_D(s)$, $T = N^-_D(t)$.
Wende Theorem 14.4 (Mengers Theorem) an.

$\qed$

\section{Aufgabe 4}

\subsection{maximaler Fluss}
Sei $f$ ein Fluss auf dem Netzwerk $(D, s, t)$ mit
\begin{align*}
f(s, v_1) &= 7 		& f(s, v_3) &= 3 	& f(s, v_5) &= 8 	& f(s, v_6) &= 2 \\
f(v_1, v_2) &= 3 	& f(v_1, v_3) &= 4	& f(v_2, v_3) &= 1	& f(v_2, v_7) &= 0 \\
f(v_2, t) &= 5		& f(v_3, v_4) &= 8	& f(v_3, v_7) &= 0	& f(v_4, t) &= 13 \\
f(v_5, v_2) &= 3	& f(v_5, v_4) &= 5	& f(v_6, v_5) &= 0	& f(v_6, v_7) &= 2 \\
f(v_7, t) &= 2
\end{align*}
und dem Betrag $20$. Der Fluss ist maximal, da alle eingehenden Kanten von $t$
vollständig genutzt werden. Daraus ergibt sich direkt ein
\subsection{minimaler Schnitt}
Da, wie bereits gezeigt, die eingehenden Kanten von $t$ von $f$ vollständig
genutzt werden, ist $\{s, v_1, ..., v_7\}, \{t\}$ ein minimaler Schnitt.

\section{Aufgabe 5}

Wir gehen wie folgt vor:

\begin{tikzpicture}[node distance=2cm, auto]
	\node (nu) {$\nu(G)$};
	\node (tau) [right of=nu] {$\tau(G)$};
	\node (flow) [below of=nu] {max flow};
	\node (cut) [below of=tau] {min cut};
	\path --
		(nu) edge node {?} (tau)
		(nu) edge node {$1$} (flow)
		(flow) edge node {$\star$} (cut)
		(cut) edge node {$2$} (tau)		
	;
\end{tikzpicture}

Aussage $\star$ folgt aus dem Max-Flow-Min-Cut-Theorem.

Zeige Aussage $1$:

Sei $G = (A, V = V_1 \cup V_2)$ ein bipartiter Graph.
Seien die Kanten in $A$ oBdA gerichtet, also $A \subseteq V_1 \times V_2$.
Sei $G' = (A', V')$ mit $V' = V \cup
\{s, t\}$ und $A' = A \cup (\{s\} \times V_1) \cup (V_2 \times \{t\})$.
Sei die Kapazität $c(v) = 1$ für alle $v \in V'$.

Analog zum Beweis des Max-Flow-Min-Cut-Theorems (folgend aus Mengers
Theorem) gibt es einen Maximalen Fluss, der aus kantendisjunkten Pfaden von
$s$ nach $t$ besteht.

Behauptung: Dieser maximale Fluss ist, eingeschränkt auf den Graphen $G$, ein maximales Matching.

Da die Pfade kantendisjunkt sind und jeder Knoten aus $V$ entweder nur eine
eingehende Kante (von $s$) oder nur eine ausgehende Kante (von $t$). Somit
kann jeder Knoten auch nur maximal eine im Fluss verwendete Kante in $A$
haben. Also ist der auf $G$ eingschränkte Fluss ein Matching.

Wäre der eingeschränkte Fluss kein maximales Matching, so gäbe es eine Kante
$(u, v) \in A$, die man dem Matching hinzufügen könnte. Dann gäbe es auch
noch keinen Pfad durch $u$ oder $v$, und der Pfad $(s,u), (u,v), (v, t)$
könnte zu den kantendisjunkten Pfaden hinzugefügt werden. Dann wäre der
Fluss jedoch nicht maximal, was einen Widerspruch zur Annahme darstellt.

Somit ist gezeigt, dass aus dem maximalen Fluss auf $G'$ ein maximales
Matching auf $G$ gleicher Kardinalität berechnet werden kann.

Zeige Aussage $2$:

Betrachte den Graphen $G' = (E, V = V_1 \cup V_2 \cup \{s, t\})$, der mit
der für Aussage $1$ beschriebenen Transformation aus $G$ hervorgeht.
Zeige, dass die Größe eines minimalen Vertex Covers gleich der Größe eines
minimalen Schnittes ist.

"`$\geq$"':

Sei $C = S_1 \cup S_2$ mit $S_1 \subseteq V_1$ und $S_2 \subseteq V_2$ ein Vertex Cover.
Sei nun $( S, V \setminus S)$ mit $S = (V_1 \setminus S_1) \cup S_2 \cup
\{s\}$ ein Schnitt. Betrachte alle Kanten in diesem Schnitt:

Alle Kanten, die von $s$ in einen Knoten in $S_1$ gehen, sind im Schnitt -- $|S_1|$ viele.
\\
Alle Kanten, die von einem Knoten in $S_2$ nach $t$ gehen, sind im Schnitt
-- $|S_2|$ viele. \\
Es bleiben Schnittkanten zwischen $V_1$ und $V_2$. Kanten von $S_1$ nach
$S_2$ sind "`Rückwärtskanten"' und bleiben daher unberücksichtigt. Kanten
von $V_1 \setminus S_1$ nach $V_2 \setminus S_2$ können nicht existieren, da
sie durch das Vertex Cover nicht abgedeckt wären.

Somit ist die Größe des Schnitts $|S_1| + |S_2| = |C|$, also ide sie Größe
eines minimalen Vertex Covers mindestens so groß wie die eines minimalen
Schnitts.

"`$\leq$"':

TODO

\end{document}
