\begin{lec}[2011-10-31]\end{lec}

\stepcounter{section}
\tocsection{Matchings, Stable Sets, Vertex Covers and Edge Covers}

\begin{defn}
Let $G = (V, E)$ be an undirected graph. A \new{matching}{Paarung} is a
subset $M \subset E$ such that $e \cap e' = \emptyset \forall e, e' \in M$
with $e \neq e'$.

A matching $M$ is called \emph{perfect} if all vertices are incident to
some edge in $M$.
\end{defn}

\begin{xmp+}
An airline has to allocate two pilots to each of the (round)trips on a
single day. Only certain pairs of pilots can work together, due to
experience, qualification, location, etc. By defining a graph with vertex
set the pilots and edges if two pilots can work together, a daily
corresponds to a matching since a pilot can work at two trips at the same
time.
\end{xmp+}

\begin{defn}
A matching $M \subseteq E$ is called \emph{maximal} if no further edges can
be added.

A \emph{maximum} matching is a matching $M$ with maximal cardinality, i.e.
no other matching $M'$ exists with $|M'| > |M|$.

$\nu(G) =$ \new{matching number}{Paarungszahl}, size of a maximum matching.
\end{defn}

\tocsubsection{Further related graph parameters}
$\alpha(G) = $ \new{Stable set number / independent set number}{Stabile-Mange-Zahl}, size of a maximum stable set in
$G$: a subset $S \subseteq V$ such that $\forall_{\{v, w\} \in E} \left|
\{v, w\} \cap S \right| \leq 1$

$\rho(G) = $ \new{Edge cover number}{Kantenüberdeckungszahl}, minimum size of an
edge cover of $G$: a subset $F \subseteq E$ such that $\forall v \in V:
\exists e \in F: e = \{ v, w \}$.

$\tau(G) = $ Vertex \new{cover number}{Knotenüberdeckungszahl}, minium size of a
vertex cover of $G$: a subset of $W \subseteq V$ such that $\forall e \in E:
e \cap W \neq \emptyset$.

\begin{lem}
$\rho(G) = \infty$ if and only if $G$ contains isolated vertices.
\end{lem}
\begin{proof}
If $G$ contains isolated vertices, such vertices cannot be covered by any
edge, hence $\rho(G) = \infty$. If $G$ does not contain isolated vertices,
then $E$ is an edge cover itself, thus $\rho(G) \leq |E|$.
\end{proof}

\begin{lem} % 7.4
$S \subseteq V$ is a stable set if and only if $V \setminus S$ is a vertex
cover.
\end{lem}
\begin{proof}
Exercise sheet.
\end{proof}

\begin{lem} % 7.5
$\alpha(G) \leq \rho(G)$.
\end{lem}
\begin{proof}
If $\rho(G) = \infty$, then $\alpha(G) \leq \rho(G)$ follows automatically.
If $\rho(G) < \infty$, then every vertex has degree at least one. Let $F$ be
an edge cover in $G$. Since the vertices of a stable set are not adjacent,
there must exist an edge $e \in F$ for all $v \in S$ such that $v \in e_v$
and $e_v \neq e_w$ for all $v, w \in S, v \neq w$. Hence, $|F| \geq |S|$ and
it follows $\alpha(G) \leq \rho(G)$.
\end{proof}

\begin{lem} % 7.6
$\nu(G) \leq \tau(G)$
\end{lem}
\begin{proof}
To cover all edges of a matching $M$ by vertices, we need at last $|M|$
vertices. Hence, $\nu(G) \leq \tau(G)$.
\end{proof}
% 7.7
\begin{thm}[Gallai's Theorem]

For every graph $G = (V, E)$ without isolated vertices, it holds that
\[
\alpha(G) + \tau(G) = |V| = \nu(G) + \rho(G)
\]
\end{thm}
\begin{proof}
$\alpha(G) + \tau(G) = |V|$ follows directly from Lemma 7.4.

To show $\nu(G) + \rho(G) = |V|$, consider a maximum matching $M$ ($|M| =
\nu(G)$). $M$ covers all $2 |M|$ vertices in $M$. For every vertex not
covered by $M$, add an incident edge to $M$. The resulting edge cover has
\[
|M| + (|V| - 2 |M|) = |V| - |M| = |V| - \nu(G)
\] edges. Hence, $\rho(G) \leq
|V| - \nu(G)$.

Now, let $F$ be an edge cover with $|F| = \rho(G)$. Remove for every vertex
$v \in V$ $deg_F(v) - 1$ edges incident to $v$. The resulting edge set is a
matching with at least
\[
\nu(G) \geq |F| - \sum\limits_{v \in V} (deg_F(v) - 1) = |F| - 2 |F| + |V| = |V| - |F|
= |V| - \rho(G)
\]
\end{proof}

For all graph parameters, there exists a weighted version:
\[
\nu(G, w), \tau(G, w), \alpha(G, w), \rho(G, w)
\]

\begin{qstn}
How do we find a maximum (weighted) matching?
\end{qstn}

\begin{defn} % 7.8
Let $M$ be a matching in $G$. A path $P = (v_0 e_1 v_1 ... e_r v_r)$ in $G$
is called \emph{$M$-alternating ($M$-alternierend)}, if $M$ contains either
all edges $e_i$ with $i$ even or all edges $e_i$ with $i$ odd.

A $M$-alternating path $P$ is called \emph{$M$-augmenting ($M$-augmentierender)} path if $v_0$ and
$v_r$ are not matched in $M$, i.e. $v_0, v_r \not\in \bigcup_{e \in M} e$.
\end{defn}

\begin{lem} % 7.9
If $P$ is $M$-augmenting, then $r$ odd and
\[
	M' = M \setminus \{ e_2, e_4, ..., e_{r-1} \} \cup \{ e_1, e_3, ..., e_r \}
\]
is a matching with $|M'| = |M| + 1$.
\end{lem}
\begin{proof}
Trivial.
\end{proof}

\begin{lem} % 7.10
(Berge)

Let $G = (V, E)$ be a graph and $M$ a matching in $G$.
Then either $M$ is a maximum matching or there exists a $M$-augmenting path.
\end{lem}
