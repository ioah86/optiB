\documentclass[a4paper,twoside]{article}

\usepackage[utf8]{inputenc}
\usepackage[T1]{fontenc}
\usepackage{cmbright}
\usepackage{tikz}
\usepackage{amsmath}
\usepackage{enumerate}
\usepackage{amssymb}
\usepackage{amsthm}
\usepackage[normalem]{ulem}
\usepackage{listings} %für code
\usepackage[pdfborder={1 1 0}]{hyperref}
\usepackage[acronym]{glossaries}
\usepackage{fancyhdr}

\makeglossaries

% \renewcommand{\chaptermark}[1]{\markboth{#1}{}}
% \renewcommand{\sectionmark}[1]{\markright{#1}{}}
  
\fancypagestyle{mystyle}{ %
	\fancyhf{} % remove everything
	\renewcommand{\headrulewidth}{0.25pt} % remove lines as well
	\renewcommand{\footrulewidth}{0.25pt}
	\fancyhead[LE,RO]{\thepage}
	%\fancyhead[CE,CO]{\sectionname}
	\fancyhead[RE,LO]{Lecture \thelec}
}

\pagestyle{mystyle}

\definecolor{highlight}{cmyk}{0.8,0,0.38,0.34}
\newcommand{\new}[2]{\newacronym{#1}{#2}{#1}\emph{\textcolor{highlight}{\gls{#1}}}}

\newtheoremstyle{numbered}
   {}
   {}
   {}
   {}
   {\normalfont\bfseries}
   {:}
   {\newline} 
   {}

\theoremstyle{numbered}

\newtheorem{thm}[subsection]{Theorem}
\newtheorem{sat}[subsection]{Satz}
\newtheorem{defn}[subsection]{Definition}
\newtheorem{defn+}[subsubsection]{Definition}
\newtheorem{lem}[subsection]{Lemma}
\newtheorem{cor}[subsection]{Corollary}
\newtheorem{bsp}[subsection]{Beispiel}
\newtheorem{xmp}[subsection]{Example}
\newtheorem{xmp+}[subsubsection]{Example}
\newtheorem{bem}[subsection]{Bemerkung}
\newtheorem{bem+}[subsubsection]{Bemerkung} %für nicht im krieg skript vorhandene Bemerkungen
\newcounter{question}
\newtheorem{qstn}{Question}[subsection]
\newcounter{remark}
\newtheorem{rem}[remark]{Remark}
\newcounter{approach}
\newtheorem{appr}[approach]{Approach}

\newcounter{lecture}
\newtheorem{lec}[lecture]{Lecture}
\newtheorem{vl}[lecture]{Vorlesung}

\renewcommand{\labelenumi}{(\alph{enumi})} %erste Ebene (a)
\newcommand{\N}[0]{\mathbb{N}}
\newcommand{\Z}[0]{\mathbb{Z}}
\newcommand{\Q}[0]{\mathbb{Q}}
\newcommand{\R}[0]{\mathbb{R}}
\newcommand{\C}[0]{\mathbb{C}}

\newcommand{\plainset}[1]{\left\{#1\right\}}
\newcommand{\ouptoset}[1]{\plainset{1, …, #1}} %one up to set
\newcommand{\zuptoset}[1]{\plainset{0, …, #1}} %zero up to set
\newcommand{\condset}[2]{\left\{ #1: \; #2\right\}} %condition set

\renewcommand{\Re}[0]{\operatorname{Re}}
\renewcommand{\Im}[0]{\operatorname{Im}}
\newcommand{\sgn}[0]{\operatorname{Sgn}}
\newcommand{\argmin}[0]{\operatorname{Argmin}}
\newcommand{\argmax}[0]{\operatorname{Argmax}}
%\newcommand{\deg}[0]{\operatorname{deg}}
%\newcommand{\det}[0]{\operatorname{det}}

\renewcommand{\thesubsubsection}[0]{\thesubsection(+\arabic{subsubsection})} %Subsections richtig numerieren

\lstset{numbers=left, basicstyle=\ttfamily, numberstyle=\tiny, mathescape=true} %listing style für code

\setlength{\parindent}{0pt}
\setlength{\parskip}{0.25em}

\begin{document}

\setcounter{lecture}{1}
\setcounter{section}{2}
\begin{lec}[2011-10-13]\end{lec}

\tocsection{Graph basics}

\begin{defn}[connected]
A graph is called \new{connected}{zusammenhängend} if there exists a [s,t]-Path between all pairs of vertices $s,t \in V$.
\end{defn}

\begin{defn}[forrest, tree, spanning, forest problem, minimum spanning tree]
A \new{forest}{Wald} is a graph that does not contain a cycle (Kreis). A connected forest is called a
\new{tree}{Baum}. A tree in a graph (as subgraph) is called
\new{spanning}{aufspannend}, if it contains all vertices.

Given a graph $G=(V,E)$ with edge weights $c_e \in \mathbb{R}$ for all $e \in E$, the task to find a forest $W \subset E$ such that \[c(W):=\sum\limits_{e\in W} \] is maximal, is called the
\new{Maximum Forest Problem}{Problem des maximalen Waldes}. 
The task to find a tree $T\subset E$ which spans $G$ and which weight $c(T)$ is minimal, is called the
\new{Minimum Spanning Tree (MST) problem}{minimaler Spannbaum}.
\end{defn}

\begin{lem}
A tree $G=(V,E)$ with at least 2 vertices has at least 2 vertices of degree 1.
\end{lem}
\begin{proof}
Let $v$ be arbitrary. Since $G$ is connected, $\deg(v) \geq 1$. Assume $\deg(v)=1$. So $\delta(v)=\{vw\}$. If $\deg(w)=1$, we found two vertices with degree 1. If $\deg(w)>1$, there exist a neighbour of $w$, different from $v$ which we call $u$. Now, again $u$ has degree $1$ or higher. If we repeat this procedure we either find a vertex of degree 1 or find again \emph{new} vertices. Hence, after at most $n-1$ vertices we end up at a vertex of degree 1. 
Now, if $\deg(v) \geq 2$, we do the same and find a vertex of degree 1, say $w$. Then repeat the above, staring from $w$ to find a second vertex of degree 1.
\end{proof}

\begin{cor}
A tree $G=(V,E)$ with maximum degree $\Delta$ has at least $\Delta$ vertices of degree 1.
\end{cor}

\begin{lem}
	\begin{enumerate}
	\item For every graph $G=(V,E)$ it holds that 2$|E|=\sum\limits_{u \in V} \deg(u)$
	\item for every tree $G=(v,E)$ it holds that $|E|=|V|-1$.
	\end{enumerate}
	\end{lem}

\begin{proof}



	\begin{enumerate}
		\item trivial
		\item Proof by induction. Clearly, if $|V|=1$ or $|V|=2$ it holds. Assumption: true for $n \geq 2.$
		Let $G$ be a tree with $n+1$ vertices. By Lemma 2.3, there exists a vertex $v \in G$ with $\deg(v)=1$. $G-v=G[V\setminus \{v\}]$ is a tree again with $n$ vertices and thus $|E(G-v)|=|V(G-v)|-1$. Since G differs by one vertex and one edge from $G-v$, the claim holds got $G$ as well.
	\end{enumerate}
\end{proof}


\begin{lem}
If $G=(V,E)$ with $|V| \geq 2$ has $|E|< |V|-1, G$ is not connected.
\end{lem}

\tocsection{MST and Shortest Path Algorithms}
$\min_{x \in X} = -\max_{x \in X} -c(x)$ \sout{maximal forest}\\
X spanning trees\\
$\min_{x \in X} + (n-1)D= -\max_{x \in X} -c(x) (n-1)D =\max_{x\in X} \sum \underbrace{D-C_{ij}x_{ij}}_{\geq 0 if D \geq \max_{ij \in E} c_{ij}}$

\begin{thm}
Kruskal's Algorithm returns the optimal solution.
\end{thm}
\begin{proof}
Let $T$ be Kruskal's tree and assume there exists a tree $T'$ with $c(T') < c(T)$. Then there exist an edge $e' \in T'\setminus T$. Then $T \cup \{e'\} $ contains a cycle $\{e_1, e_2, \hdots, e_k, e'\}$. Let $ c_f=max_{i=1, \hdots k}c_{l_i} $. 
At the moment Kruskal chooses edge $f$, edge $e'$ cannot be added yet and therefore $c(e')\geq c(f)$. Now exchange $e'$ by $f$ in $T'$. Hence the number of differences beetween $T'$ and $T$ is reduced by one, $C(T'_{new})\leq c(T') < c(T)$. Repeating the procedure results in $c(T) \leq \hdots < c(T)$, a contradiction.
\end{proof}

\begin{lec}[2011-10-17]\end{lec}

\begin{defn+}
The \new{running time of algorithms}{Laufzeit} of an algorithm is measured by the number of operations needed in worst case of a function of the input size. We use the $O(\cdot)$ notation (Big-O-notation) ot focus on the most important factor of the running time, ignoring constants and smaller factors.
\end{defn+}

\begin{xmp+}
	If the running time is $3n \cdot \log n + 26n$, the algorithm runs in $O(n \cdot \log n)$. If the running time is $3n \cdot \log n + 25 n^2$, the algorirthm runs in $O(n^2)$.
\end{xmp+}

For graph Problems, the running is expressed in the number of vertices $n=|V|$ and the number of edges $m=|E|$. Sometimes $m$ is approximated by $n^2$.

\begin{xmp+}[Kruskal's Algorithm]
	First, the edged are sorted according to nondecreasing weights. This can be done in $O(m\cdot \log m)$. Next, we repeatedly select an edge or reject its selection until $n-1$ edges are selected. Since the last selected edge might be after $m$ steps, this routine is performed at most $O(m)$ times.
	
	Checking whether the end nodes of $\{u,v\}$ are already in the same tree can be done in constant time, if we label the vertices of the trees selected so far: $r(u) = \# trees\: containing\: u$. If $r(u) \neq r(v)$, the trees are connected by $\{u,v\}$ to a new tree. 
	
	Without going into details, the resetting of labels in one of the old trees, can be done $O(\log n)$ on average. Since this update has to be done at most $n-1$ times, it takes $O(n\cdot \log n)$. 
	
	Overall, Kruskal runs in \[ O(n \log m + m + n \cdot \log n) = O(m \cdot \log m) = O(m \cdot \log n^2) = O(m \cdot \log n) \]
\end{xmp+}

\begin{defn+}[Shortest paths in acyclic digraphs]
A directed graph (digraph) $D=(V,A)$  is called \new{acyclic}{azyklisch} if it does not contain any
\new{directed cycles}{}, i.e. a \new{chain}{Kette} $(v_0,a_1,v_1,a_2,v_2, ... a_k,v_k)$, $k \geq 0$, with $a_i(v_{i-1},v_i) \in A$ and $v_k = v_0$. In particular, D does not contain
\new{antiparallel}{entgegengesetzt} arcs: if $(u,v)\in A$, $(v,u)\not \in A$. With $\delta_D^+(v)$ we denote the arcs leaving vertex v: 
\[ \delta_D^+(v) = \{(u,w) \in A: u = v\}\]
similarly:
\[\delta_D^-(v) = \{(u,w) \in A: w = v\}\]
are the arcs entering v.

The \new{outdegree}{Ausgangsgrad} of $v$ is $\deg^+_D(v) = | \delta^+(v) |$ (assuming simple digraph)

The \new{indegree}{Eingangsgrad} of $v$ is $\deg^-_D(v) = | \delta^-(v)|$
\end{defn+}

\stepcounter{section}

\begin{defn}
	The \new{shortest path}{} problem in a acyclic digraph is, given an acyclic digraph $D=(V,A)$, a length function $C: A\rightarrow \R$ and two vertices $s,t\in V$, find a $[s,t]$-path of minimal length.
\end{defn}

\begin{qstn}
Does there exist a $[s,t]$-path at all?
\end{qstn}
\begin{thm}
A digraph $D = (V, A)$ is acyclic, if and only if there exists a permutaion $\sigma : V \rightarrow \{ 1, .., n \}$ of the vertices such that $\deg^-_{D [v_1, ...,  v_n]} (v_i) = 0$ for all $i = 1, ..., n$ with $v_i = \sigma^{-1}(i)$.
\end{thm}
\begin{proof}
By induction:

For digraph with $| V |=1$, the statement is true. Assume the statement is true for all digraphs with $|V| \leq n$ and consider $D = (V, A)$ acyclic with $n+1$ vertices. 
If there does not exist a vertex with $\deg_D^-(v) = 0$, a directed cycle can be detected by following incoming arcs backwards until a vertex is repeated, a contradiction regarding the acyclic property of $D$.

Hence, let $v$ be a vertex with $\deg^-_D(v)=0$. Set $v_1 = v$. The digraph $D - v_1$ has $n$ vertices and is acyclic, and thus has a permutation $(v_2, ..., v_{n+1})$ with
\[ \deg^-_{D[v_i, ..., v_{n+1}]}(v_i) = 0 \qquad \forall i = 2, ..., n+1 \]
Now, $(v_1, ..., v_{n+1})$ is a permutation fulfilling the condition.

In reverse, if there exists a permutation $(v1, ..., v_{n+1})$, $\deg^-_D(v_1) = 0$ and there cannot exist a directed cycle containing $v_1$. By induction, neither cycles containing $v_i, i = 2, ..., n+1$ exist.
\end{proof}

\begin{thm}
A $[s,t]$-path exists in a acyclic Digraph $D = (V, A)$ if and only if in all permutations $\sigma: V \rightarrow \{ 1, ..., n\}$ with $\deg^-_{D[v_i, ..., v_n]}(v_i) = 0$ for all $i = 1, ..., n$, it holds that $\sigma(s) < \sigma(t)$.
\end{thm}
\begin{proof}
Assume there exists a permutation $\sigma$ with $\sigma(s) > \sigma(t)$. Since outgoing arcs only go to higher ordered vertices, there does not exist a path from $s$ to $t$ in $D$.

In reverse, if there does not exist a path from $s$ to $t$, we order all vertices with paths to $t$ first, followed by $t$ and $s$ afterwards. 
\end{proof}

\begin{qstn}
How do we find the shortest $[s, t]$-path if it exists?
\end{qstn}
To simplify notation, let $V = \{1, ..., n\}, s=1, t=n$ and $(i, j) \in A \Rightarrow i < j$.
Let $D(i)$ be the distance from $i$ to $n$ and $NEXT(i)$ be the next vertex on the shortest path from $i$ to $n$.

\tocsubsection{Bellman's Algorithm}
\begin{lstlisting}
$	D(i) = \{ \infty: i<n and NEXT(i) = NIL, 0: i = n \}$
FOR $i = n-1$ DOWNTO $1$ DO
	$D(i) = \min_{j=j+1, ..., n} \{ D(j) + c(i, j) \}$ with $c(i, j) = \infty$ if $(i, j) \not\in A$
	$NEXT(i) = \argmin_{j=i+1, ..., n} \{ D(j) + c(i, j) \}$
\end{lstlisting}

\begin{thm}
Bellman's Algorithm is correct and runs in $O(m + n)$ time.
\end{thm}
\begin{proof}
Every path from $1$ to $n$ passes through vertices of increasing ID. Assume there exists a path $(a_1, ..., a_k)$ with $\sum_{i=1}^k c(a_i) < D(1)$.
Let $a_1 = (1, j_1)$. Since $D(1) \leq c(a_1) + D(j_1)$, it should hold that
\[ \sum\limits_{i=2}^2 c(a_i) < D(j_1) \]
But $D(j_1) \leq c(a_2) + D(j_2)$ with $a_2 = (j_1, j_2)$, etc.

In the end, $c(a_k) < D(j_{k-1})$ but $D(j_{k-1}) \leq c(a_k) + D(n) = c(a_k)$, contradiction.
\end{proof}

\begin{lec}[2011-10-20]\end{lec}

\begin{thm} % 3.4
Bellman's Algorithm is correct and runs in $O(m + n) = O(n)$.
\end{thm}
\begin{proof}
of runtime:

\[ D(i) = \min\limits_{(i,j) \in A} D(j) + D(i, j) \]

$\Rightarrow$ Every arc is considered once, and thus overall $O(m)$
computations are needed. Initialization costs $O(n)$.
\end{proof}

\begin{bem+}
The running time does not contain the time to find the permutation.
\end{bem+}

Observation 1: We not only found the shortest path from $1$ to $n$, but
also from $i$ to $n$, $i = 2, ..., n$.

Observation 2: We can use a similar procedure for the shortest path from
$1$ to $i$, $i = 2, ..., n$. (with $PREV(i)$ for previous instead of
$NEXT(i)$).

\begin{qstn}
Can we find a shortest path from $1$ to $i$ in a digraph that is
not acyclic, i.e. it contains cycles?
\end{qstn}

%\setcounter{section}{4}
%\setcounter{subsection}{0}

\stepcounter{section}

\begin{thm} % 4.1
The Moore-Bellman-Algorithm returns the shortest paths from $1$ to $i = 1,
..., n$ provided $D$ does not contain negative-weighted directed cycles.
\end{thm}
\begin{proof}
We call an arc $(i, j) \in A$ an \emph{upgoing} arc (Aufwärtsbogen) if $i < j$ and a
\emph{downgoing} arc (Abwärtsbogen) if $i > j$.

A shortest path from $1$ to $i$ contains at most $n-1$ arcs. If an upgoing
arc is followed by a downgoing arc (or vice versa), we have a \emph{change
of direction} (Richtungswechsel). With at most $n-1$ arcs, at most $n-2$
changes of direction are possible.

Let $D(i, m)$ be the value of $D(i)$ at the end of the $m$-th iteration.
We will show (and this is enough):
\[ D(i, m) = min \{ c(W): \textrm{$W$ is the directed $[1,i]$-path with at most $m$
changes of directions} \} \]

We prove it by induction on $m$.

\begin{itemize}
\item For $m = 0$, the algorithm is equivalent to Bellman's algorithm for acyclic
grpahs. Thus, $D(i, 0)$ is the length of the shortest path without any
changed of direction.

\item Now, let us assume, that the statement is true for $m \geq 0$ and the
subroutine is executed for the $m+1$-st time. The set of $[1,i]$-paths with
at most $m+1$ changes of direction consists of 
\begin{enumerate}[(a)]
\item\label{item:bellman:a} $[1,i]$-paths with $\leq m$ changes of direction
\item\label{item:bellman:b} $[1,i]$-paths with exactly $m+1$ changes of direction
\end{enumerate}
$\Rightarrow D(i, m)$

\item Since every path starts with an upgoing arc $(1, k)$, the last arc
after $m+1$ changes is either a downgoing arc if $m+1$ is odd or an upgoing
arc if $m+1$ is even. We restrict ourselves to $m+1$ odd ($m+1$ even is
similar).

To compute the minimum length path in (\ref{item:bellman:b}) we use an
additional induction on $i= n, n-1, ..., j+1$. Since every path ending at
$n$ ends with an upgoing arc, there do not exist such $[1,n]$-paths.
Hence, $D(n, m+1) = D(n, m)$.

Now assume that $D(k, m+1)$ is correctly computed for $i \leq k \leq n$. The
shortest path from $1$ to $i-1$ with exactly $m+1$ changes ends with a
downgoing arc $(j, i-1)$, $j > i-1$.

$D(j, m+1)$ is already computed correctly. If $PREV(j) > j$, no change of
direction is required in $j$ and $D(i-1, m+1) = D(j, m+1) + c(j,i-1)$
If $PREV(j) < j$, the last avec of the $[1,j]$-Path is upgoing, and thus $D(i-1,m+1)=D(j,m)+c(j,i)$. The last change of direction at $j$ is thus, in worst case, the $(m+1)$-st change. Hence, $D(i-1,m+1)$ fulfills the statement. 
\end{itemize}
\end{proof}

\begin{rem}
In fact, the algorithm finds the minimum length of a chain (kette) with at most $n-2$ changes of direction. In case of negative weighted cycles these might be in a chain several times. 

In case no negative weighted cycles exist, the min. length chains are indeed paths. Hence, the algoithm only works correctly if \emph{all} cycles are non-negative weighted.
\end{rem}

\begin{rem}
If a further executing of the subroutine $(m=n-1)$ results in at least one change of a value $D(i)$, then the digraph contains negative weighted cycles.
\end{rem}

\begin{rem}
A more efficient implementation is given by E'sopo-Pape-Variant.
\end{rem}

\subsection*{Dijkstra's Algorithm for non-negative weights}

\begin{thm}
Dijkstra returns the shortest paths from $1$ to $i, i= 1 ... n$, provided all weights $\geq 0$.
\end{thm}

\begin{proof}
Each step, one vetex is moved from $T$ to $S$. At the end of a step, $D(j)$ is the shortest path from $1$ to $j$ via vertices in $S$. 

If $S=V (T= \emptyset), D(i)$ is thus the shortiest $[1,i]$-path
\end{proof}

\begin{lec}[2011-10-24]\end{lec}

\subsubsection*{Shortest paths between all pairs of vertices}

Solution 1: Apply Moore-Bellman or Dijkstra to all vertices $i$ as starting vertex

Solution 2: Apply Floyd's Algorithm

Notation: 

$w_{ij}$ is the length of the shortest $[i,j]$-path, $i \neq j$

$w_{ii}$ is the length of the shortest directed cycle containing $i$

$p_{ij}$ is the predecessor of $j$ on the shortest $[i,j]$-path (cycle)

$W =(w_{ij}) $ is the \new{shortest path length matrix}{???}

%\setcounter{section}{5}
%\setcounter{subsection}{0}

\stepcounter{section}

\begin{thm} %5.1
The Floyd Algorithm works correctly if and only if $D = (V,A)$ does not contain any negative weighted cycles.

$D$ contains a negative weighted cycle if and only if one of the diagonal elements $w_{ii}<0$.
\end{thm}

\begin{proof}
\newcommand{\wii}[0]{w_{ii}}
\newcommand{\wij}[0]{w_{ij}}

Let $W^k$ be the matrix $W$ after iteration $k$, with $W^0$ being the initial matrix. By induction on $k = 0, ..., n$ we show that $W^k$ is the matrix of shortest path lengths with vertices $1, ..., k$ as \emph{possible} internal vertices, provided $D$ does not contain a negative cycle on these vertices.

If $D$ has a negative cycle, then $\wii^k < 0$ for an $i \in \{ 1, ..., n\}$

For $k = 0$, the statement clearly true.

Assume, it is correct for $k \geq 0$, and we have executed the $(k+1)$st iteration.

It holds that $\wij^{k+1} = \min\{\wij^k, w_{i,k+1}^k + w_{k+1,j}^k\}$. Note that, provided no negative cycle exists, $w_{i,k+1}^{k+1}$ does not have any vertex $k+1$ as internal vertex, and thus $w_{i,k+1}^{k+1} = w_{i,k+1}^k$ (similarly, $w_{k+1,j}^{k+1} = w_{k+1,j}^k$).

$w_{i,k+1}^k$ is the minimal length of a $[i,k+1]$-path with $\{1, ..., k\}$ as allowed internal vertices. Similarly, $w_{k+1,j}^k$.

Thus, $w_{i,k+1}^k+w_{k+1,j}^k$ is the minimal length of an $[i,j]$-path (not necessarily simple) containing $k+1$ (mandatory) and $\{ 1, ..., k \}$ (voluntary). If the shortest path from $i$ to $j$ using $\{1, ..., k+1\}$ does not contain $k+1$, it only contains $\{1, ... , k\}$ (voluntary) and, hence, $\wij^k$ is the right value.

What remains to show is that the connection of the $[i,k+1]$-path with the $[l+1,j]$-path is indeed a simple path.

Let $K$ be this chain. After removal of cycles, the chain $K$ contains (of course) a simple $[i,j]$-path $\bar K$. Since such cycles may only contain vertices from $\{1, ..., k+1\}$, one cycle must contain $k+1$.
If this cycle is not negatively weighted, then path $\bar K$ is shorter and $\wij< w_{i,k+1}^k + w_{k+1,j}^k$.

If this cycle is negatively weighted, $w_{k+1,k+1}^k<0$ (the cycle only contains internal vertices from $\{1, ..., k\}$) and algorithm would have stopped earlier.
\end{proof}

\subsection*{Min-Max-Theorems for combinatorial Optimization Problems}

From "Optimierung A": Duality of linear programs

\[\max_{\text{s. t.}, Ax \leq b, x \geq 0} c^Tx = \min_{\text{s.t.}, A^T y \geq c, y \geq 0} b^T y\]

For several combinatorial problems $\min\{c(x): x \in X\}$

We can define a second set $Y$ and a function $b(y)$ with $\max\{b(y): y \in Y \} = \min \{ c(x): x \in X\}$ where Y and b(y) have a graph theoretical interpretation. 

Existence of such a "Dual" Problem indicates often that the problem can be solved "efficiently". For the shortest path problem several max-min-theorems exist.

\begin{defn}
An \new{$(s,t)$-cut}{$(s,t)$-Schnitt} in a digraph $D=(V,A)$ with $s,t \in V$ is a subset $B \subset A$ of the arcs with the property that every $(s,t)$-path contains at least one arc of $B$.

Stated  otherwise, for every cut $B$, there exists a vertex set $W \subset V$ such that
\begin{itemize}
	\item $s \in W$, $t \in V \setminus W$
	\item $\delta^+(w) = \{(i,j) \in A: i \in W, j \in V \setminus W\} \subseteq B$
\end{itemize}
\end{defn}

\begin{thm}
	Let $D=(V,A)$ be a digraph, $c(a)=1 \, \forall a \in A, s,t \in V, s \neq t$. Then the minimum length of a $[s,t]$-path equals the maximum number of arc-disjoint $(s,t)$-cuts. 
\end{thm}

\begin{proof}
	Folows from \ref{!!!} %TODO! 5.4
\end{proof}

\begin{thm} %5.4
	Let $D=(V,A)$ be a digraph, $c(a) \in \Z_+ \, \forall a \in A \wedge s,t \in V \wedge s\neq t$. Then the $\min$ length of an $[s,t]$-path equals the maximum number $d$ of (not necessarily different) $(s,t)$-cuts $C_1, ..., C_d$ such that every arc $a \in A$ is contained in at most $c(a)$ cuts.
\end{thm}

\begin{proof}
	We define $(s, t)$-cuts $C_i = \delta^+(v_i)$ with $V_i=\{v \in V: \exists \text{$(s,v)$-path with $c(P)\leq i-1$}\}$
	\begin{align*}
		v_1 & = \{s\} \\
		v_2 & = \{5,3,4\} \\
		v_3 & = \{5,2,3,4\} \\
		v_4 & = v_3 \cup \{ 6 \}
	\end{align*}
	
	(for the example graph on the board)
	
	The shortest $[s,t]$-path $P$ consists of arcs $a_1, ... a_k$ with arc $a_j$ contained in $(s,t)$-cuts $C_i$,  $i \in \{\sum_{l=1}^{j-1} c(a_l) + 1, ..., \sum_{l=1}^j a(a_l)\}$: exactly $c(a)$ cuts.
\end{proof}	

\begin{lec}[2011-10-26]\end{lec}

\subsubsection*{Knapsack problem} 
\stepcounter{section}

\begin{defn} %6.1
	The \new{Knapsack problem}{Knapsack Problem} is defined by a set of items $N = \{ 1, ... , n \}$ weights $a_i \in \N$, value $c_i \in \N$, and a bound $b\in \N$. We search for a subset $S\subset \N$ such that 
	\[
		a(S) = \sum_{i \in S} a_i \leq b \; \text{and} \; c(S) = \sum_{i\in S} c_i \; \text{maximum}
	\]
\end{defn}

\appr{Greedy algorithm}

Idea: Items with small weight but high value are the most atrractive ones.

Procedure:
\begin{lstlisting}
Sort the items such that $\frac{c_1}{a_1} \leq \frac{c_2}{a_2} \leq ... \leq \frac{c_n}{a_n}$.

Set $S = \emptyset$.
For $i = 1$ to $n$ do
	if $( a(s) + a_i \leq b) $ then
		$S = S \cup \{ i \}$
	endif
endfor
return $S$ and $c(S)$
\end{lstlisting}

\begin{thm}
	The greedy algorithm does \emph{ not } guarantee an optimal solution.
\end{thm}

\begin{proof}
	Let $b=10$, $n = 6$
	
	\begin{align*}
		\begin{tabular}{l|ccccc}
			$i$ & 2 & 3 & 4 & 5 & 6 \\ \hline
			$a_i$ & 9 & 2 & 2 & 2 & 2 \\
			$c_i$ &19 & 4 & 4 & 4 & 4
		\end{tabular}
	\end{align*}
	
	Greedy: $S=\{1\}$, $c(s)=20$
	
	Optimal: $S=\{2,3,4,5,6,\}$, $c(S)=20$
\end{proof}

Approach 2: Integer Linear Programming

The set of solutions $X$ of a combinatorial optimization problem can (almost always) be written as the intersection of integer points in $\N_0^n$ and a polyhedron $\{x \in \R^n: Ax\leq b\}$

Let $x \in \{0,1\}^n$ be a vector representing all solutions of the knapsack problem:
\[
	x_i = \begin{cases}
		1 \qquad \text{if} i \in S \\
		0 \qquad \text{otherwise}
	\end{cases}
\]

$X = \{0,1\} \cap \{ x \in \R^n: \sum\limits_{i = 0}^n a_ix_i\leq b\}$

Knapsack: $\max \sum{i = 0}^n c_i x_i$

The \new{linear relaxation}{Lineare Relaxierung} of an ILP is the linear program optained by relaxing the integrality of the variables:

$\max \sum_{i=1}^n c_i x_i$

s. t. $\sum_{i=1}^n a_i x_i \leq b, 0 \leq x_i \leq 1 \qquad \forall i \in \{1, ..., n\}$

\begin{thm}
An optimal solution $\tilde{x}$ of the linear relaxation of the knapsack
problem is:

There exists a $k \in \{1, ..., n\}$ such that 
\[
\tilde{x}_i = \left \{ \begin{array}{ll}
1 & \text{if } i \leq k \\ 
0 & \text{if } i > k+1 \\
(b-\sum^k_{i=1} a_j) / a_{k+1} & \text{if } i=k+1
\end{array} \right.
\]
where $c_1/a_1 \geq c_2 / a_2 \geq ... \geq c_n / a_n$.
\end{thm}

\begin{proof}
Let $x^*$ be an optimal solution with $c^T c^* > c^T \tilde{x}$.
If $x^*_i < 1$ for $i \leq k$, there must exist a $j \geq k+1$ with $x^*_j >
\tilde{x}_j$.

We define $\bar{x}$ with $\varepsilon \leq x^*_j - \tilde{x}_j$ as
\[
\bar{x}_l = \left \{ \begin{array}{ll}
x^*_k & \text{for } l \not\in \{i, j\} \\
x^*_l - \varepsilon & \text{for } l=j \\
x^*_l + \frac{a_j}{a_l} \cdot \varepsilon & \text{for } l=i
\end{array} \right.
\]
Then $\bar{x}$ is feasible and
\[
c^T \bar{x} = \sum_{l=1}^n c_l \bar{x}_l =
\sum_{l=1}^n c_l x^*_l + \underbrace{c_i \cdot \frac{a_j}{a_i} \varepsilon - c_j
\varepsilon}_{\geq 0} \geq c^T x^* > c^T \tilde{x}
\]

Repetition yields $c^T \bar{x} > c^T \bar{x}$, a contradiction.
\end{proof}

Note:

If $\tilde{x}$ is integer valued, then the solution is also optimal for the
knapsack problem. In this case, also the greedy algorithm is optimal.

Approach 3: Dynamic Programming

A dynamic program algorithm to solve a problem first solves similar, but
smaller subproblems in order to use their solution to solve the original
problem.

The problem should conform to the \emph{optimality principle} of Bellman:
Given an optimal solution for the original problem, a partial solution
restricted to a subproblem is also optimal for the subproblem.

Let $f_k(b)$ be the optimal solution value of the knapsack problem with
total weight equal to $b$ and items from $\ouptoset{k}$.

\begin{thm}
$f_{k+1}(b) = max \{ f_k(b), f_k(b-a_{k+1} + c_{k+1} \}$.
\end{thm}
\begin{proof}
An optimal solution of $f_{k+1}(b)$ either contains item $k+1$ or not.
If $k+1$ is not contained, the problem is identical to $f_k(b)$.
If $k+1$ is contained, other items in the solution should have total weight
$b - a_{k+1}$.

Hence, $f_k(b - a_{k+1})$ is an optimal solution for the remaining items $+
c_{k+1}$ for the item $k+1$.
\end{proof}

\begin{cor}
The knapsack problem can be solved in $O(nb)$ with value $max_{d=0, ..., b}
f_n(d)$.
\end{cor}

\begin{lec}[2011-10-31]\end{lec}

\subsection*{Matchings, Stable Sets, Vertex Covers and Edge Covers}

\stepcounter{section}

\begin{defn}
Let $G = (V, E)$ be an undirected graph. A \new{matching}{Paarung} is a
subset $M \subset E$ such that $e \cap e' = \emptyset \forall e, e' \in M$
with $e \neq e'$.

A matching $M$ is called \emph{perfect} if all vertices are incident to
some edge in $M$.
\end{defn}

\begin{xmp+}
An airline has to allocate two pilots to each of the (round)trips on a
single day. Only certain pairs of pilots can work together, due to
experience, qualification, location, etc. By defining a graph with vertex
set the pilots and edges if two pilots can work together, a daily
corresponds to a matching since a pilot can work at two trips at the same
time.
\end{xmp+}

\begin{defn}
A matching $M \subseteq E$ is called \emph{maximal} if no further edges can
be added.

A \emph{maximum} matching is a matching $M$ with maximal cardinality, i.e.
no other matching $M'$ exists with $|M'| > |M|$.

$\nu(G) =$ \new{matching number}{Paarungszahl}, size of a maximum matching.
\end{defn}

\subsection*{Further related graph parameters}
$\alpha(G) = $ Stable set number / independent set number (Stabile-Mange-Zahl), size of a maximum stable set in
$G$: a subset $S \subseteq V$ such that $\forall_{\{v, w\} \in E} \left|
\{v, w\} \cap S \right| \leq 1$

$\rho(G) = $ Edge cover number (Kantenüberdeckungszahl), minimum size of an
edge cover of $G$: a subset $F \subseteq E$ such that $\forall v \in V:
\exists e \in F: e = \{ v, w \}$.

$\tau(G) = $ Vertex cover number (Knotenüberdeckungszahl), minium size of a
vertex cover of $G$: a subset of $W \subseteq V$ such that $\forall e \in E:
e \cap W \neq \emptyset$.

\begin{lem}
$\rho(G) = \infty$ if and only if $G$ contains isolated vertices.
\end{lem}
\begin{proof}
If $G$ contains isolated vertices, such vertices cannot be covered by any
edge, hence $\rho(G) = \infty$. If $G$ does not contain isolated vertices,
then $E$ is an edge cover itself, thus $\rho(G) \leq |E|$.
\end{proof}

\begin{lem} % 7.4
$S \subseteq V$ is a stable set if and only if $V \setminus S$ is a vertex
cover.
\end{lem}
\begin{proof}
Exercise sheet.
\end{proof}

\begin{lem} % 7.5
$\alpha(G) \leq \rho(G)$.
\end{lem}
\begin{proof}
If $\rho(G) = \infty$, then $\alpha(G) \leq \rho(G)$ follows automatically.
If $\rho(G) < \infty$, then every vertex has degree at least one. Let $F$ be
an edge cover in $G$. Since the vertices of a stable set are not adjacent,
there must exist an edge $e \in F$ for all $v \in S$ such that $v \in e_v$
and $e_v \neq e_w$ for all $v, w \in S, v \neq w$. Hence, $|F| \geq |S|$ and
it follows $\alpha(G) \leq \rho(G)$.
\end{proof}

\begin{lem} % 7.6
$\nu(G) \leq \tau(G)$
\end{lem}
\begin{proof}
To cover all edges of a matching $M$ by vertices, we need at last $|M|$
vertices. Hence, $\nu(G) \leq \tau(G)$.
\end{proof}
[B
\begin{thm} % 7.7
( Gallai's Theorem)

For every graph $G = (V, E)$ without isolated vertices, it holds that
\[
\alpha(G) + \tau(G) = |V| = \nu(G) + \rho(G)
\]
\end{thm}
\begin{proof}
$\alpha(G) + \tau(G) = |V|$ follows directly from Lemma 7.4.

To show $\nu(G) + \rho(G) = |V|$, consider a maximum matching $M$ ($|M| =
\nu(G)$). $M$ covers all $2 |M|$ vertices in $M$. For every vertex not
covered by $M$, add an incident edge to $M$. The resulting edge cover has
\[
|M| + (|V| - 2 |M|) = |V| - |M| = |V| - \nu(G)
\] edges. Hence, $\rho(G) \leq
|V| - \nu(G)$.

Now, let $F$ be an edge cover with $|F| = \rho(G)$. Remove for every vertex
$v \in V$ $deg_F(v) - 1$ edges incident to $v$. The resulting edge set is a
matching with at least
\[
\nu(G) \geq |F| - \sum\limits_{v \in V} (deg_F(v) - 1) = |F| - 2 |F| + |V| = |V| - |F|
= |V| - \rho(G)
\]
\end{proof}

For all graph parameters, there exists a weighted version:
$\nu(G, w), \tau(G, w), \alpha(G, w), \rho(G, w)$.

\begin{qstn}
How do we find a maximum (weighted) matching?
\end{qstn}

\begin{defn} % 7.8
Let $M$ be a matching in $G$. A path $P = (v_0 e_1 v_1 ... e_r v_r)$ in $G$
is called \emph{$M$-alternating ($M$-alternierend)}, if $M$ contains either
all edges $e_i$ with $i$ even or all edges $e_i$ with $i$ odd.

A $M$-alternating path $P$ is called \emph{$M$-augmenting ($M$-augmentierender)} path if $v_0$ and
$v_r$ are not matched in $M$, i.e. $v_0, v_r \not\in \bigcup_{e \in M} e$.
\end{defn}

\begin{lem} % 7.9
If $P$ is $M$-augmenting, then $r$ odd and
\[
	M' = M \setminus \{ e_2, e_4, ..., e_{r-1} \} \cup \{ e_1, e_3, ..., e_r \}
\]
is a matching with $|M'| = |M| + 1$.
\end{lem}
\begin{proof}
Trivial.
\end{proof}

\begin{lem} % 7.10
(Berge)

Let $G = (V, E)$ be a graph and $M$ a matching in $G$.
Then either $M$ is a maximum matching or there exists a $M$-augmenting path.
\end{lem}

\begin{lec}[2011-11-04]\end{lec}

\stepcounter{section}
\tocsubsection{Matchings in bipartite graphs}

% theorem 8.1
\begin{thm}(Berge) Let $G=(V,E)$ be a grpah an $M$ a matching in $G$. Then either $M$ is a maximum matching or there exists an $M$-augmentige path.
\end{thm}
\begin{proof}
If $M$ is a maximum matching, no $M$-augmenting path can exist, since $M'$ would be a larger matching. \\
Let $\bar{M}$ a matching with $|\bar M | > | M |$. Consider the components (Komponente) of $G'=(V,M \cup \bar M)$. Since the the degree of vertices in $(V,M)$ and $(V,\bar M)$ is at most one, the degree in $G'$ is at most two. Thus, each component of $G'$ is either a path (possibly of length zero) or a cycle. Since $|\bar M| > |M|$, at least one component of $G'$ has to have more edges from $\bar M$ than from $M$. Such a component cannot be a cycle and thus is a path, better an $M$-augmenting path since the end nodes are not matched in $M$.
\end{proof}

\begin{defn} 
A graph $ G=(V,E) $ is called \new{bipartite}{bipartit} if and only if $V=U \cup W (U\cap W =\emptyset)$ such that $\{v,w\} \in E \Rightarrow v \in U, w \in W$ (or vice versa). The set $U$ and $W$ are called the color classes of $V$.
\end{defn}

\begin{xmp+}
$n$ workers, $m$ Jobs. Not every worker is qualified for every job. How many jobs can be processed simultaneously? $U=$ set of worker. $W=$ set of jobs, $\{u_i,w_j\} \in E \Leftrightarrow$ worker $i$ is qualified for job $j$.
\end{xmp+}

$\nu(G) \leq \tau(G)$ vertex cover 
\begin{thm}(König's Matching Theorem, 1931)
For every bipartite graph $G=(V,E): \nu(G)=\tau(G)$ 
\end{thm}
\begin{proof} $\nu(G) \leq \tau(G)$ holds by Lemma 7.6. %TODO 
We therefore only show $\nu(G) \geq \tau(G)$.
We may assume that $G$ has at least one edge (otherwise $\nu(G)=\tau(G)=0).$ We will show that $G$ has a vertex $v$ that is matched in every maximum matching. Let $\{v,w\}=e$ be an arbitrary edge in $G$ and assume that $M$ and $N$ are two maximal mathcings with $u$ not matched in $M$ and $v$ not matched in $N$. Define $P$ as the component of $(V,M \cup N)$ containing $u$.
Since $u$ is only matched in $N$, $deg_P(u)=1$ and thus $u$ is an end node of the path $P$. Since $M$ is maximum, $P$ is \emph{not} an $M$-augmenting path. Hece, the length of $P$ is even. Consequently, $P$ does not contain $v$ (otherwise $P$ ends at $v$ which contradict the bipartiteness of $G:P\cup \{u,v\}$ would be an odd cycle).
The path $P \cup \{v,v\}$ is thus odd, starts with vertex $v$ not matched in $N$ and ends with another vertex not matched in $N$. Hence $P \cup e$ is $N$-augmenting path; contracdiction since $N$ is a maximum matching.\\
So, either $u$ or $v$ must be contained in all maximum matchings, let's say $v$. Now, consider $G':=G-v$. It holds that $\nu(G)=\nu(G)-1$. By induction on $n=|V|$ we may assume that $G'$ has a vertex cover $W$ with $|W|=\nu(G')$. Then $W \cup \{v\}$ is a vertex cover of $G$ of size $\nu(G')+1=\nu(G)$. It follows $\tau(G)\leq \nu(G)$.
\end{proof}

\begin{cor}(König's Edge Cover Theorem)
Every bipartite graph has $\alpha(G)=\rho(G)$.
\end{cor}
\begin{proof}
Foolows from Thm 7.7(Gallai's Thm) and 8.3 %TODO
\end{proof}

\tocsubsubsection{Matching augmenting algorithm for bipartite graphs}

\emph{Input}:bipartite graph $G=(V,E)$ and a matching $M$

\emph{Output}: matching $M'$ with $|M'| > |M|$ (if it exists)

\emph{Description}: Let $U,W$ be the color classes of $G$. Orientate every edge $e=\{v,w\} (u \in U, w \in W)$ as follows: \\
if $e \in M$, then orientate from $w$ to $u$ \\
if $e \notin M$, then orientate from $u$ to $w$ \\
Let $D$ be the digraph constructed this way. Consider \[ U':=U \setminus \bigcup\limits_{e \in M}e \quad \text{and} \quad W':=W \setminus \bigcup\limits_{e \in M}e \]An $M$-augmenting path exists if and only if a directed path in $D$ exists starting at a vertex $U'$ and ending at a vertex $W'$. Augment $M$ with this path return $M'$.

\begin{thm}
A maximum matching can be found by applying at most $\frac{1}{2} |V| $ times the above algorithm.
\end{thm}
\begin{proof} Thm 8.2 %TODO
says that either a maximum matching or an $M$-augmenting path exists. This path can be found by the digraph as it has to start with an unmatched vertex and alternates between matched and not matched edges. Thes is guaranteed by the direction of the edges in the digrpah.. If we start with $M = \emptyset$, the size increases by one in every iteration. A max matching can at most $\frac{1}{2} |V|$ edges.
\end{proof}


\begin{lec}[2011-11-07]\end{lec}

\stepcounter{section}

\begin{cor}[Frobenius theorem]
	A bipartite graph $G=(V,E)$ has a perfect matching, iff all the vertex covers contain at least $\frac{1}{2} |V|$ nodes.
\end{cor}

\begin{proof}
	Let $\tau(G) \geq \frac{1}{2} |V|$ hold. We derive $\nu(G) \geq \frac{1}{2} |V|$ from König's theorem, but $\nu(G) \leq \frac{1}{2} |V|$ holds as well. If $\tau(G) < \frac{1}{2} |V|$ holds, $\nu(G) < \frac{1}{2} |V|$ holds as well and therefore no perfect matching exists.
\end{proof}

\begin{cor}
	Every regular bipartite graph ($\deg(v) = \deg(w) \; \forall v,w \in B$) (with positive degree) has a perfect matching
\end{cor}

\begin{proof}
	exercise
\end{proof}

\begin{thm}[Hall's theorem (Hochzeitssatz)]
	Let $G =( U \cup W, E)$ be a bipartite graph with classes $U, W$. For each subset $X \subseteq U$, $N(X)$ describes the set of nodes in $W$, which are adjacent to a node in $X$ (neighborhood)
	\[
		\nu(G) = |U| \Leftrightarrow | \nu(X)| \geq |X| \; \forall X \subseteq U
	\]
\end{thm}

\begin{proof}
	necessity. If $|N(X)| < |X|$ holds, then a matching $M$ can contain at most 
	$| N(X)|$ edges with ending nodes in $X$. Therefore noes remain in $X$ which 
	cannot be covered by a matching.
	
	Sufficient: Assume $\nu(G)<|U|$, but $|N(X)| \geq |X| \; \forall X \subseteq U$.
	Then there exists a vertex cover $Z$ for $G$ with $|Z| < |U|$.
	
	Define $X:=U \setminus Z$. Then it follows that $N(X) \subseteq W \cap Z$ and 
	therefore \[
		|X| - |N(X)| \geq \underbrace{|U| - |U \cap Z|}_{=|X|} - \underbrace{| W \cap Z |}_{\geq |N(X)|} = |U| - |Z| > 0
	\]
	produces a contradiction
\end{proof}

\tocsubsubsection{Maximum Weighted Matching on bipartite graphs}

Let $G=(V,E)$ be a graph and let $\omega: E \rightarrow \R$ be a weight function
For any subset $M$ of $E$ define the weight $\omega(M)$ of $M$ by \[
	\omega(M) = \sum\limits_{e \in M}\omega(e)
\]

\begin{defn+}
	We call a matching extreme if it has maximum weight among all matchings of 
	\new{cardinality $|M|$}{Kardnialität $|M|$}
	\[
		l(e) =
		\begin{cases}
			w(e), & \text{if } e\in M \\
			-w(e), & \text{if } e \not \in M
		\end{cases}
	\]	
\end{defn+}

\begin{lem}
	Let $P \plainset{e_1, e_2, ..., e_r}$ be an $M$-augmenting path of minimal length.
	If $M$ is extreme, then $M' = M \setminus \underbrace{\plainset{e_2, ..., e_{n-1}}}_{\in M} \cup \underbrace{\plainset{e_1, e_2, ..., e_r}}_{\not \in M}$ is again extreme
\end{lem}

\begin{proof}
	Let $N$ be any extreme matching of size $|M| + 1$. As $|N| > |M|$, $M \cup N$
	has a component $Q = \plainset{f_1, f_2, ..., f_q}$ that is an $M$-augmenting path.
	As $P$ is a shortest $M$-augmenting path, we know $l(P) \leq l(Q)$. Moreover,
	as $M^*=N\setminus \plainset{f_1, f_3, ..., f_q} \cup \plainset{f_2, ..., f_{q-1}}$ 
	is a matching of size $|M|$ and as $M$ is extreme, we know that $\omega(M^*) \leq \omega(M)$
	Hence $\omega(N) = \omega(M^*) - l(Q) \leq \omega (M) - l(P) = \omega(M')$.
	Hence $M'$ is extreme.
\end{proof}

\tocsubsubsection{Hungarian method for maximum weighted matching}
"find iteratively extreme matchings $M_1, M_2, ..., M_j$ with $|M_K|=K$. Then the
matching among $M_1, ..., M_j$ of maximum weight is a maximum weight matching."

Define  $D$ as in the maximum cardinality matching algorithm.

set \[U'=U\setminus (\cup_{e\in M}e) \qquad W'=W\setminus ( \cup_{e \in M}e) \]

extend $D$ with nodes $s$ and $t$ and arcs $(s,u) \; \forall u \in U'$ with 
length $0$ and $(w,t) \; \forall w \in W'$ with length $0$.

Now we find a shortest path form $s$ to $t$, to get an extreme matching $M'$ 
from extreme matching $M$ $(|M'| > |M|)$.

%9.5
\begin{thm}
	Let $M$ be an extreme matching. Then $D$ does not contain a directed circuit 
	of negative length.
\end{thm}

\begin{proof}
	Suppose $C$ is a directed circuit in $D$ with length $l(C) < 0$. We may assume 
	$C=(u_0, w_1, u_1, ..., w_t, u_t)$ with $u_0=u_t$, $u_1, ..., u_t \in U$ and 
	$w_1, ..., w_t \in W$. Then $w_1u_1, ..., w_tu_t$ belong to $M$ and the edges 
	$u_0w_1, ..., u_{t-1}w_t$ do not belong to $M$.
	Then $M''=M\setminus \plainset{w_1u_1, ..., w_tu_t} \cup \plainset{u_0w_1, ..., u_{t+1}w_t}$
	matching of cardinality $|M''| = |M|$ with weight: $\omega(M'') = \omega(M) - l(C) > \omega(M)$.
	This contradicts to $M$ being extreme.
\end{proof}

\begin{cor}
	The maximum weighted matching in bipartite graphs can be found by using a 
	shortest path algorithm $\frac{1}{2}|V|$-times.
\end{cor}

\begin{lec}[2011-11-10]\end{lec}

\stepcounter{section}

\tocsubsection{Matchings in non-bipartite graphs}

%10.1
\begin{defn}
	A component of a graph is called \new{odd}{ungerade}, if its number 
	of vertices is odd. We define $o(G):=$ number of odd components of $G$. 
	$G-U:=G[V \setminus U]$ denotes the subgraph induced by $V \setminus U$.
\end{defn}

%10.2
\begin{thm}[Tutte-Berge-Formula]
	For a graph $G=(V,E)$ it holds that $D(G)=\min_{U \subset V} \plainset{\frac{1}{2}(|V|+|U|-o(G-U))}$
\end{thm}
	
\begin{proof}
	We first prove $\leq$. For $U \subseteq V$ it holds that 
	\begin{align*}
		\nu(F) &\leq |U| + \nu(G-U) \leq |U| + \frac{1}{2} (| V \setminus U| - o(G-U)) \\
		       &= \frac{1}{2}(|U \setminus V| + 2 |U| - o(G-U)) \\
					 &= \frac{1}{2}(|V| + |U| - o(G-U))
	\end{align*}
	Now, we prove $\geq$. We apply inductions on $|V|$. For $|V| = 0$, the statement is trivial.
	Furterh, we may assume w.l.o.g (o.B.d.A) that $G$ is connected, otherwise the
	result follows by induction on the components of $G$.
	
	Case 1: \\
	There exists a vertex $v$ that is matched in all maximum matches (like in the 
	bipartite case). Thus $\nu (G-V) = \nu (G)-1$ and by induction there exists
	a subset $U'\subseteq V-v$ with $\nu (G-V) = \frac{1}{2}(|V\setminus \plainset{v}|+ |U'|-o(G-v-U'))$
	Set $U:=U' \cup \plainset{V}$. Then \begin{align*}
		\nu(G) &= \nu(G-v)-1=\frac{1}{2}(|V\setminus \plainset{v}|+ |U'| - o(G-U'-v))-1 \\
		       &= \frac12 (|V| - 1 + |U| - 1 - o(G-U) + 2) \\
					 &\leq \min_{T \subseteq V}\frac12(|V|+|T| -O(G-T))
	\end{align*}
	
	Case 2: \\
	There does \emph(not) exist a vertex matched in all maximum matchings. So for 
	all $v \in V$, there exists a maximum matching without $v$. Then, in particular
	$\nu(G)<\frac12|V|$. We will show that there exists in this case a matching 
	of size $\frac12(|V|-1)$. By this, we have proven the theorem (set $U=\emptyset$).

	If there does not exist a matching of size $\frac12(|V|-1)$, then for every
	matching $M$, there exist two vertices $u$ and $v$ such that both are not 
	matched. Among all maximum matchings, select a triple $(M,u,v)$ for which 
	$\dist(u,v)$ is minimum there, $\dist(u,v)$ is the minimum number of edges 
	or a path between $u$ and $v$. That is, for any other maximum matching $N$
	and pair of unmatched vertices $s,t$, $\dist(s,t) \geq \dist(u,v)$.
	
	If $\dist(u,v)=1$, then $u$ and $v$ are adjacent and we can extend $M$ with 
	$\plainset{u,v}$, a contradiction. Thus, $\dist(u,v)\geq 2$ and hence there 
	exists a vertex $t$ on the shortest path from $u$ to $v$. Not that $t$ is 
	matched in $M$, otherwise $\dist(u,t)<\dist(u,v)$; a contradition.	For $t$ 
	there exist other maximum matchings not covering $t$. Choose $N$ such that 
	$|M \cap N|$ is maximal.
	
	Also, $u$ and $v$ are covered by $N$, since otherwise $N$ would have a pair
	$(t,u)$ (or $(t,v)$) of unmatched vertices with $\dist(u,t) < \dist (u,v)$ 
	(or $\dist(t,v) < \dist(u,v)$, respectively).
	
	Since $|M| = |N|$, there must be a second vertex $x\neq t$ which is not
	covered by $N$, but covered by $M$. Let $e=\plainset{x,y} \in M$. Then $y$ is
	also covered by $N$, otherwise $N$ could be extended by $\plainset{x,y}$, 
	contradiction to maximality of $|N|$. Let $f=\plainset{y,z}\in N$. Now
	replace $N$ by $N \setminus \plainset f \cup \plainset e$. The new matching 
	has one more edge in common with $M$, contradiction.
	
	Hence, a maximum matching cannot mis two or more vertices. Thus $\nu (G) = 
	\frac12(|V|-1)$
	
\end{proof}

%10.2
\begin{cor}[Tutte's 1-Factor Theorem]
	A graph $G(V,E)$ has a perfect matching if and only if $G - U$ contains at
	most $|U|$ odd components for all $U \subseteq V$.
\end{cor}

\begin{cor}
	Let $G = (V,E)$ be a graph without isolated vertices. Then \[
		\rho(G) = \max_{U \subseteq V} \frac{|U| + o(G[U])}2
	\]
\end{cor}

\begin{proof}
	Homework
\end{proof}

\tocsubsection{Edmond's Matching Algorithm (blossom shrinking algorithm)}
	Again we are looking for $M$-augmenting paths. In bipartite graphs we just have 
	to find a shortest path in the orientation of $G$ by $M$. 
	
\begin{xmp+}
\begin{tikzpicture}
	\node (n1) {$1$};
	\node (n2) [right of=n1] {$2$};
	\node (n3) [right of=n2] {$3$};
	\node (n4) [above right of=n3] {$4$};
	\node (n5) [below right of=n3] {$5$};
	\node (n6) [right of=n4] {$6$};
	
	\path
		(n1) edge (n2)
		(n2) edge [red] (n3)
		(n3) edge (n4)
		(n3) edge (n5)
		(n4) edge [red] (n5)
		(n4) edge (n6)
	;
\end{tikzpicture}
und noch ein Graph
\end{xmp+}

\begin{xmp+}
	Weitere 2 graphen
\end{xmp+}

We define for sets $X$ and $Y$: \[
	X/Y:= 
		\begin{cases}
			X & \text{if } X \cap Y=\emptyset \\
			(X\setminus Y) \cup \plainset Y & \text{if } X \cap Y \neq \emptyset
		\end{cases}
\]

Thus, if $G=(V,E)$ is a graph and $C\subseteq V$, then $V/C$ is the set of ver-
tices where all vertices in $C$ are replaced by a singe vertex $C$.
For an edge $e\in E$, $e/C= e$ if $e$ and $C$ are disjoint, whereas $e/C=
\plainset{u,C}$ if $e \in \plainset{u,v}$ with $u \not \in C$, $v \in C$, and
$e/C=\plainset{C,C}$ if $e = \plainset{u,v}$ with $u,v \in C$.

The last type is unimporteant fo mathings and can be ignored. Further, for 
$F \subseteq E$, we have $F/C:= \plainset{f/C: f\in F}$ and thus $G/C:=(V/C,E/C)$
is again a graph which results from \new{shrinking}{Schrumpfen} of $C$.


\newpage

\printglossaries

\end{document}
