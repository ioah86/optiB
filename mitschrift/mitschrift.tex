\documentclass[a4paper,twoside,10pt]{article}

\usepackage[utf8]{inputenc}
\usepackage[T1]{fontenc}
\usepackage{cmbright}
\usepackage{tikz}
\usepackage{amsmath}
\usepackage{enumerate}
\usepackage{amssymb}
\usepackage{amsthm}
\usepackage[normalem]{ulem}
\usepackage{listings} %für code
\usepackage{hyperref}
\usepackage[acronym]{glossaries}
\usepackage{fancyhdr}

\makeglossaries

% \renewcommand{\chaptermark}[1]{\markboth{#1}{}}
% \renewcommand{\sectionmark}[1]{\markright{#1}{}}
  
\fancypagestyle{mystyle}{ %
	\fancyhf{} % remove everything
	\renewcommand{\headrulewidth}{0.25pt} % remove lines as well
	\renewcommand{\footrulewidth}{0.25pt}
	\fancyhead[LE,RO]{\thepage}
	%\fancyhead[CE,CO]{\sectionname}
	\fancyhead[RE,LO]{Lecture \thelec}
}

\pagestyle{mystyle}

\definecolor{highlight}{cmyk}{0.8,0,0.38,0.34}
\newcommand{\new}[2]{\newacronym{#1}{#1}{#2}\emph{\textcolor{highlight}{\gls{#1}}} (#2)}

\newtheoremstyle{style}
   {}
   {}
   {}
   {}
   {\normalfont\bfseries}
   {:}
   {\newline} 
   {}

\theoremstyle{style}

\newtheorem{thm}[subsection]{Theorem}
\newtheorem{sat}[subsection]{Satz}
\newtheorem{defn}[subsection]{Definition}
\newtheorem{defn+}[subsubsection]{Definition}
\newtheorem{lem}[subsection]{Lemma}
\newtheorem{cor}[subsection]{Corollary}
\newtheorem{bsp}[subsection]{Beispiel}
\newtheorem{xmp}[subsection]{Example}
\newtheorem{xmp+}[subsubsection]{Example}
\newtheorem{bem}[subsection]{Bemerkung}
\newtheorem{bem+}[subsubsection]{Bemerkung} %für nicht im krieg skript vorhandene Bemerkungen
\newcounter{question}
\newtheorem{qstn}[question]{Question}
\newcounter{remark}
\newtheorem{rem}[remark]{Remark}
\newcounter{approach}
\newtheorem{appr}[approach]{Approach}

\newcounter{lecture}
\newtheorem{lec}[lecture]{Lecture}
\newtheorem{vl}[lecture]{Vorlesung}

\renewcommand{\labelenumi}{(\alph{enumi})} %erste Ebene (a)
\newcommand{\N}[0]{\mathbb{N}}
\newcommand{\Z}[0]{\mathbb{Z}}
\newcommand{\Q}[0]{\mathbb{Q}}
\newcommand{\R}[0]{\mathbb{R}}
\newcommand{\C}[0]{\mathbb{C}}

\newcommand{\plainset}[1]{\left\{#1\right\}}
\newcommand{\ouptoset}[1]{\plainset{1, …, #1}} %one up to set
\newcommand{\zuptoset}[1]{\plainset{0, …, #1}} %zero up to set
\newcommand{\condset}[2]{\left\{ #1: \; #2\right\}} %condition set

\renewcommand{\Re}[0]{\operatorname{Re}}
\renewcommand{\Im}[0]{\operatorname{Im}}
\newcommand{\sgn}[0]{\operatorname{Sgn}}
\newcommand{\argmin}[0]{\operatorname{Argmin}}
\newcommand{\argmax}[0]{\operatorname{Argmax}}
%\newcommand{\deg}[0]{\operatorname{deg}}
%\newcommand{\det}[0]{\operatorname{det}}

\renewcommand{\thesubsubsection}[0]{\thesubsection(+\arabic{subsubsection})} %Subsections richtig numerieren

\lstset{numbers=left, basicstyle=\ttfamily, numberstyle=\tiny, mathescape=true} %listing style für code

\setlength{\parindent}{0pt}
\setlength{\parskip}{0.25em}

\begin{document}


\setcounter{lecture}{1}
\setcounter{section}{2}

\lec 2011-10-13


\begin{defn}[connected]
A graph is called \new{connected} (zusammenhängend) if there exists a [s,t]-Path between all pairs of vertices $s,t \in V$.
\end{defn}

\begin{defn}[forrest, tree, spanning, forest problem, minimum spanning tree]
A \new{forest} (Wald) is a graph that does not contain a cycle (Kreis). A connected forest is called a \new{tree} (Baum). A tree in a graph (as subgraph) is called \new{spanning} (aufspannend), if it contains all vertices.

Given a graph $G=(V,E)$ with edge weights $c_e \in \mathbb{R}$ for all $e \in E$, the task to find a forest $W \subset E$ such that $c(W):=\sum\limits_{e\in W} $ is maximal, is called the \new{Maximum Forest Problem} (Problem des maximalen Waldes). 
The task to find a tree $T\subset E$ which spans $G$ and which weight $c(T)$ is minimal, is called the \new{Minimum Spanning Tree} (MST) problem (minimaler Spannbaum).
\end{defn}

\begin{lem}
A tree $G=(V,E)$ with at leat 2 vertices has at least 2 vertices of degree 1.
\end{lem}
\begin{proof}
Let $v$ be arbitrary. Since $G$ is connected, $deg(v) \geq 1$. Assume $deg(v)=1$. So $\delta(v)=\{vw\}$. If $deg(w)=1$, we found two vertices with $degree$ 1. If $deg(w)>1$, there exist a neighbour of $w$, different from $v:u$. Now, again $u$ has $degree$ 1 or higher. If we repeat this procedure we either find a vertix of degree 1 or find again \new{new} vertices. Hence, after at most $n-1$ vertices we end up at a vertex of degree 1. 
Now, if $deg(v) \geq 2$, we do the same and find a vertex of degree 1, say $w$. Then repeat the above, staring from $w$ to find a second vertex of degree 1.
\end{proof}

\begin{cor}
A tree $G=(V,E)$ with maximum degree $\Delta$ has at least $\Delta$ vertices of degree 1.
\end{cor}

\begin{lem}
	\begin{enumerate}
	\item For every graph $G=(V,E)$ it holds that 2$|E|=\sum\limits_{u \in V} deg(u)$
	\item for every tree $G=(v,E)$ it holds that $|E|=|V|-1$.
	\end{enumerate}
	\end{lem}

\begin{proof}



	\begin{enumerate}
		\item trivial
		\item Proof by induction. Clearly, if $|V|=1$ or $|V|=2$ it holds. Assumption: true for $n \geq 2.$
		Let $G$ be a tree with $n+1$ vertices. By Lemma 2.3, there exists a vertex $v \in G$ with $deg(v)=1. G-v=G[V\setminus \{v\}$ is a tree again with $n$ vertices and thus $|E(G-v)|=V(G-v)|-1$. Since G differs by one vertex and one edge from $G-v$, the claim holds got $G$ as well.
	\end{enumerate}
\end{proof}


\begin{lem}
If $G=(V,E)$ whith $|V| \geq 2$ has $|E|< |V|-1, G$ is not connected.
\end{lem}

\section*{Algorithm MST}
$min_{x \in X} = -max_{x \in X} -c(x)$ \sout{maximal forest}\\
X spanning trees\\
$min_{x \in X} + (n-1)D= -max_{x \in X} -c(x) (n-1)D =max_{x\in X} \sum \underbrace{D-C_{ij}x_{ij}}_{\geq 0 if D \geq max_{ij \in E} c_{ij}}$

\begin{thm}
Kruskal's Algorithm returns the optimal solution.
\end{thm}
\begin{proof}
Let $T$ be Kruskal's tree and assume there exists a tree $T'$ with $c(T') < c(T)$. Then there exist an edge $e' \in T'\setminus T$. Then $T \cup \{e'\} $ contains a cycle $\{e_1, e_2, \hdots, e_k, e'\}$. Let $ c_f=max_{i=1, \hdots k}c_{l_i} $. 
At the moment Kruskal chooses edge $f$, edge $e'$ cannot be added yet and therefore $c(e')\geq c(f)$. Now exchange $e'$ by $f$ in $T'$. Hence the number of differences beetween $T'$ and $T$ is reduced by one, $C(T'_{new})\leq c(T') < c(T)$. Repeating the procedure results in $c(T) \leq \hdots < c(T)$, a contradiction.
\end{proof}

\begin{lec}[2011-10-17]\end{lec}

\begin{defn+}
The \new{running time of algorithms}{Laufzeit} of an algorithm is measured by the number of operations needed in worst case of a function of the input size. We use the $O(\cdot)$ notation (Big-O-notation) ot focus on the most important factor of the running time, ignoring constants and smaller factors.
\end{defn+}

\begin{xmp+}
	If the running time is $3n \cdot \log n + 26n$, the algorithm runs in $O(n \cdot \log n)$. If the running time is $3n \cdot \log n + 25 n^2$, the algorirthm runs in $O(n^2)$.
\end{xmp+}

For graph Problems, the running is expressed in the number of vertices $n=|V|$ and the number of edges $m=|E|$. Sometimes $m$ is approximated by $n^2$.

\begin{xmp+}[Kruskal's Algorithm]
	First, the edged are sorted according to nondecreasing weights. This can be done in $O(m\cdot \log m)$. Next, we repeatedly select an edge or reject its selection until $n-1$ edges are selected. Since the last selected edge might be after $m$ steps, this routine is performed at most $O(m)$ times.
	
	Checking whether the end nodes of $\{u,v\}$ are already in the same tree can be done in constant time, if we label the vertices of the trees selected so far: $r(u) = \# trees\: containing\: u$. If $r(u) \neq r(v)$, the trees are connected by $\{u,v\}$ to a new tree. 
	
	Without going into details, the resetting of labels in one of the old trees, can be done $O(\log n)$ on average. Since this update has to be done at most $n-1$ times, it takes $O(n\cdot \log n)$. 
	
	Overall, Kruskal runs in \[ O(n \log m + m + n \cdot \log n) = O(m \cdot \log m) = O(m \cdot \log n^2) = O(m \cdot \log n) \]
\end{xmp+}

\begin{defn+}[Shortest paths in acyclic digraphs]
A directed graph (digraph) $D=(V,A)$  is called \new{acyclic}{azyklisch} if it does not contain any \new{directed cycles}{}, i.e. a \new{chain}{Kette} $(v_0,a_1,v_1,a_2,v_2, … a_k,v_k)$, $k \geq 0$, with $a_i(v_{i-1},v_i) \in A$ and $v_k = v_0$. In particular, D does not contain \new{antiparallel}{} arcs: if $(u,v)\in A$, $(v,u)\not \in A$. With $\delta_D^+(v)$ we denote the arcs leaving vertex v: 
\[ \delta_D^+(v) = \{(u,w) \in A: u = v\}\]
similarly:
\[\delta_D^-(v) = \{(u,w) \in A: w = v\}\]
are the arcs entering v.

The \new{outdegree}{} of $v$ is $\deg^+_D(v) = | \delta^+(v) |$ (assuming simple digraph)

The \new{indegree}{} of $v$ is $\deg^-_D(v) = | \delta^-(v)|$
\end{defn+}

\stepcounter{section}

\begin{defn}
	The \new{shortest path}{} problem in a acyclic digraph is, given an acyclic digraph $D=(V,A)$, a length function $C: A\rightarrow \R$ and two vertices $s,t\in V$, find a $[s,t]$-path of minimal length.
\end{defn}

\begin{qstn}
Does there exist a $[s,t]$-path at all?
\end{qstn}
\begin{thm}
A digraph $D = (V, A)$ is acyclic, if and only if there exists a permutaion $\sigma : V \rightarrow \{ 1, .., n \}$ of the vertices such that $\deg^-_{D [v_1, …,  v_n]} (v_i) = 0$ for all $i = 1, …, n$ with $v_i = \sigma^{-1}(i)$.
\end{thm}
\begin{proof}
By induction:

For digraph with $| V |=1$, the statement is true. Assume the statement is true for all digraphs with $|V| \leq n$ and consider $D = (V, A)$ acyclic with $n+1$ vertices. 
If there does not exist a vertex with $\deg_D^-(v) = 0$, a directed cycle can be detected by following incoming arcs backwards until a vertex is repeated, a contradiction regarding the acyclic property of $D$.

Hence, let $v$ be a vertex with $\deg^-_D(v)=0$. Set $v_1 = v$. The digraph $D - v_1$ has $n$ vertices and is acyclic, and thus has a permutation $(v_2, …, v_{n+1})$ with
\[ \deg^-_{D[v_i, …, v_{n+1}]}(v_i) = 0 \qquad \forall i = 2, …, n+1 \]
Now, $(v_1, …, v_{n+1})$ is a permutation fulfilling the condition.

In reverse, if there exists a permutation $(v1, …, v_{n+1})$, $\deg^-_D(v_1) = 0$ and there cannot exist a directed cycle containing $v_1$. By induction, neither cycles containing $v_i, i = 2, …, n+1$ exist.
\end{proof}

\begin{thm}
A $[s,t]$-path exists in a acyclic Digraph $D = (V, A)$ if and only if in all permutations $\sigma: V \rightarrow \{ 1, …, n\}$ with $\deg^-_{D[v_i, …, v_n]}(v_i) = 0$ for all $i = 1, …, n$, it holds that $\sigma(s) < \sigma(t)$.
\end{thm}
\begin{proof}
Assume there exists a permutation $\sigma$ with $\sigma(s) > \sigma(t)$. Since outgoing arcs only go to higher ordered vertices, there does not exist a path from $s$ to $t$ in $D$.

In reverse, if there does not exist a path from $s$ to $t$, we order all vertices with paths to $t$ first, followed by $t$ and $s$ afterwards. 
\end{proof}

\begin{qstn}
How do we find the shortest $[s, t]$-path if it exists?
\end{qstn}
To simplify notation, let $V = \{1, …, n\}, s=1, t=n$ and $(i, j) \in A \Rightarrow i < j$.
Let $D(i)$ be the distance from $i$ to $n$ and $NEXT(i)$ be the next vertex on the shortest path from $i$ to $n$.

\subsubsection*{Bellman's Algorithm}
\begin{lstlisting}
$	D(i) = \{ \infty: i<n and NEXT(i) = NIL, 0: i = n \}$
FOR $i = n-1$ DOWNTO $1$ DO
	$D(i) = \min_{j=j+1, …, n} \{ D(j) + c(i, j) \}$ with $c(i, j) = \infty$ if $(i, j) \not\in A$
	$NEXT(i) = \argmin_{j=i+1, …, n} \{ D(j) + c(i, j) \}$
\end{lstlisting}

\begin{thm}
Bellman's Algorithm is correct and runs in $O(m + n)$ time.
\end{thm}
\begin{proof}
Every path from $1$ to $n$ passes through vertices of increasing ID. Assume there exists a path $(a_1, …, a_k)$ with $\sum_{i=1}^k c(a_i) < D(1)$.
Let $a_1 = (1, j_1)$. Since $D(1) \leq c(a_1) + D(j_1)$, it should hold that
\[ \sum\limits_{i=2}^2 c(a_i) < D(j_1) \]
But $D(j_1) \leq c(a_2) + D(j_2)$ with $a_2 = (j_1, j_2)$, etc.

In the end, $c(a_k) < D(j_{k-1})$ but $D(j_{k-1}) \leq c(a_k) + D(n) = c(a_k)$, contradiction.
\end{proof}

\begin{lec}[2011-10-20]\end{lec}

\begin{thm} % 3.4
Bellman's Algorithm is correct and runs in $O(m + n) = O(n)$.
\end{thm}
\begin{proof}
of runtime:

\[ D(i) = \min\limits_{(i,j) \in A} D(j) + D(i, j) \]

$\Rightarrow$ Every arc is considered once, and thus overall $O(m)$
computations are needed. Initialization costs $O(n)$.
\end{proof}

\begin{bem+}
The running time does not contain the time to find the permutation.
\end{bem+}

Observation 1: We not only found the shortest path from $1$ to $n$, but
also from $i$ to $n$, $i = 2, ..., n$.

Observation 2: We can use a similar procedure for the shortest path from
$1$ to $i$, $i = 2, ..., n$. (with $PREV(i)$ for previous instead of
$NEXT(i)$).

\begin{qstn}
Can we find a shortest path from $1$ to $i$ in a digraph that is
not acyclic, i.e. it contains cycles?
\end{qstn}

\begin{thm} % 4.1
The Moore-Bellman-Algorithm returns the shortest paths from $1$ to $i = 1,
..., n$ provided $D$ does not contain negative-weighted directed cycles.
\end{thm}
\begin{proof}
We call an arc $(i, j) \in A$ an \emph{upgoing} arc (Aufwärtsbogen) if $i < j$ and a
\emph{downgoing} arc (Abwärtsbogen) if $i > j$.

A shortest path from $1$ to $i$ contains at most $n-1$ arcs. If an upgoing
arc is followed by a downgoing arc (or vice versa), we have a \emph{change
of direction} (Richtungswechsel). With at most $n-1$ arcs, at most $n-2$
changes of direction are possible.

Let $D(i, m)$ be the value of $D(i)$ at the end of the $m$-th iteration.
We will show (and this is enough):
\[ D(i, m) = min \{ c(W): \textrm{$W$ is the directed $[1,i]$-path with at most $m$
changes of directions} \} \]

We prove it by induction on $m$.

\begin{itemize}
\item For $m = 0$, the algorithm is equivalent to Bellman's algorithm for acyclic
grpahs. Thus, $D(i, 0)$ is the length of the shortest path without any
changed of direction.

\item Now, let us assume, that the statement is true for $m \geq 0$ and the
subroutine is executed for the $m+1$-st time. The set of $[1,i]$-paths with
at most $m+1$ changes of direction consists of 
\begin{enumerate}[(a)]
\item\label{item:bellman:a} $[1,i]$-paths with $\leq m$ changes of direction
\item\label{item:bellman:b} $[1,i]$-paths with exactly $m+1$ changes of direction
\end{enumerate}
$\Rightarrow D(i, m)$

\item Since every path starts with an upgoing arc $(1, k)$, the last arc
after $m+1$ changes is either a downgoing arc if $m+1$ is odd or an upgoing
arc if $m+1$ is even. We restrict ourselves to $m+1$ odd ($m+1$ even is
similar).

%TODO: ref ist kaputt
To compute the minimum length path in (\ref{item:bellman:b}) we use an
additional induction on $i= n, n-1, ..., j+1$. Since every path ending at
$n$ ends with an upgoing arc, there do not exist such $[1,n]$-paths.
Hence, $D(n, m+1) = D(n, m)$.

Now assume that $D(k, m+1)$ is correctly computed for $i \leq k \leq n$. The
shortest path from $1$ to $i-1$ with exactly $m+1$ changes ends with a
downgoing arc $(j, i-1)$, $j > i-1$.
\end{itemize}
\end{proof}

\begin{lec}[2011-10-24]\end{lec}

\subsubsection*{Shortest paths between all pairs of vertices}

Solution 1: Apply Moore-Bellman or Dijkstra to all vertices $i$ as starting vertex

Solution 2: Apply Floyd's Algorithm

Notation: 

$w_{ij}$ is the length of the shortest $[i,j]$-path, $i \neq j$

$w_{ii}$ is the length of the shortest directed cycle containing $i$

$p_{ij}$ is the predecessor of $j$ on the shortest $[i,j]$-path (cycle)

$W =(w_{ij}) $ is the \new{shortest path length matrix}{???}

%\setcounter{section}{5}
%\setcounter{subsection}{0}

\stepcounter{section}

\begin{thm} %5.1
The Floyd Algorithm works correctly if and only if $D = (V,A)$ does not contain any negative weighted cycles.

$D$ contains a negative weighted cycle if and only if one of the diagonal elements $w_{ii}<0$.
\end{thm}

\begin{proof}
\newcommand{\wii}[0]{w_{ii}}
\newcommand{\wij}[0]{w_{ij}}

Let $W^k$ be the matrix $W$ after iteration $k$, with $W^0$ being the initial matrix. By induction on $k = 0, …, n$ we show that $W^k$ is the matrix of shortest path lengths with vertices $1, …, k$ as \emph{possible} internal vertices, provided $D$ does not contain a negative cycle on these vertices.

If $D$ has a negative cycle, then $\wii^k < 0$ for an $i \in \{ 1, …, n\}$

For $k = 0$, the statement clearly true.

Assume, it is correct for $k \geq 0$, and we have executed the $(k+1)$st iteration.

It holds that $\wij^{k+1} = \min\{\wij^k, w_{i,k+1}^k + w_{k+1,j}^k\}$. Note that, provided no negative cycle exists, $w_{i,k+1}^{k+1}$ does not have any vertex $k+1$ as internal vertex, and thus $w_{i,k+1}^{k+1} = w_{i,k+1}^k$ (similarly, $w_{k+1,j}^{k+1} = w_{k+1,j}^k$).

$w_{i,k+1}^k$ is the minimal length of a $[i,k+1]$-path with $\{1, …, k\}$ as allowed internal vertices. Similarly, $w_{k+1,j}^k$.

Thus, $w_{i,k+1}^k+w_{k+1,j}^k$ is the minimal length of an $[i,j]$-path (not necessarily simple) containing $k+1$ (mandatory) and $\{ 1, …, k \}$ (voluntary). If the shortest path from $i$ to $j$ using $\{1, …, k+1\}$ does not contain $k+1$, it only contains $\{1, … , k\}$ (voluntary) and, hence, $\wij^k$ is the right value.

What remains to show is that the connection of the $[i,k+1]$-path with the $[l+1,j]$-path is indeed a simple path.

Let $K$ be this chain. After removal of cycles, the chain $K$ contains (of course) a simple $[i,j]$-path $\bar K$. Since such cycles may only contain vertices from $\{1, …, k+1\}$, one cycle must contain $k+1$.
If this cycle is not negatively weighted, then path $\bar K$ is shorter and $\wij< w_{i,k+1}^k + w_{k+1,j}^k$.

If this cycle is negatively weighted, $w_{k+1,k+1}^k<0$ (the cycle only contains internal vertices from $\{1, …, k\}$) and algorithm would have stopped earlier.
\end{proof}

\subsection*{Min-Max-Theorems for combinatorial Optimization Problems}

From "Optimierung A": Duality of linear programs

\[\max_{\text{s. t.}, Ax \leq b, x \geq 0} c^Tx = \min_{\text{s.t.}, A^T y \geq c, y \geq 0} b^T y\]

For several combinatorial problems $\min\{c(x): x \in X\}$

We can define a second set $Y$ and a function $b(y)$ with $\max\{b(y): y \in Y \} = \min \{ c(x): x \in X\}$ where Y and b(y) have a graph theoretical interpretation. 

Existence of such a "Dual" Problem indicates often that the problem can be solved "efficiently". For the shortest path problem several max-min-theorems exist.

\begin{defn}
An \new{$(s,t)$-cut}{$(s,t)$-Schnitt} in a digraph $D=(V,A)$ with $s,t \in V$ is a subset $B \subset A$ of the arcs with the property that every $(s,t)$-path contains at least one arc of $B$.

Stated  otherwise, for every cut $B$, there exists a vertex set $W \subset V$ such that
\begin{itemize}
	\item $s \in W$, $t \in V \setminus W$
	\item $\delta^+(w) = \{(i,j) \in A: i \in W, j \in V \setminus W\} \subseteq B$
\end{itemize}
\end{defn}

\begin{thm}
	Let $D=(V,A)$ be a digraph, $c(a)=1 \, \forall a \in A, s,t \in V, s \neq t$. Then the minimum length of a $[s,t]$-path equals the maximum number of arc-disjoint $(s,t)$-cuts. 
\end{thm}

\begin{proof}
	Folows from \ref{!!!} %TODO! 5.4
\end{proof}

\begin{thm} %5.4
	Let $D=(V,A)$ be a digraph, $c(a) \in \Z_+ \, \forall a \in A \wedge s,t \in V \wedge s\neq t$. Then the $\min$ length of an $[s,t]$-path equals the maximum number $d$ of (not necessarily different) $(s,t)$-cuts $C_1, …, C_d$ such that every arc $a \in A$ is contained in at most $c(a)$ cuts.
\end{thm}

\begin{proof}
	We define $(s, t)$-cuts $C_i = \delta^+(v_i)$ with $V_i=\{v \in V: \exists \text{$(s,v)$-path with $c(P)\leq i-1$}\}$
	\begin{align*}
		v_1 & = \{s\} \\
		v_2 & = \{5,3,4\} \\
		v_3 & = \{5,2,3,4\} \\
		v_4 & = v_3 \cup \{ 6 \}
	\end{align*}
	
	(for the example graph on the board)
	
	The shortest $[s,t]$-path $P$ consists of arcs $a_1, … a_k$ with arc $a_j$ contained in $(s,t)$-cuts $C_i$,  $i \in \{\sum_{l=1}^{j-1} c(a_l) + 1, …, \sum_{l=1}^j a(a_l)\}$: exactly $c(a)$ cuts.
\end{proof}	

\begin{lec}[2011-10-26]\end{lec}

\setcounter{section}{6}
\setcounter{subsection}{0}

\subsubsection*{Knapsack problem} 

\begin{defn} %6.1
	The \new{Knapsack problem} is defined by a set of items $N = \{ 1, … , n \}$ weights $a_i \in \N$, value $c_i \in \N$, and a bound $b\in \N$. We search for a subset $S\subset \N$ such that 
	\[
		a(S) = \sum_{i \in S} a_i \leq b \; \text{and} \; c(S) = \sum_{i\in S} c_i \; \text{maximum}
	\]
\end{defn}

Appreach 1: Greedy algorithm

Idea: Items with small weight but high value are the most atrractive ones.

Procedure:
\begin{lstlisting}
Sort the items such that $\frac{c_1}{a_1} \leq \frac{c_2}{a_2} \leq … \leq \frac{c_n}{a_n}$.

Set $S = \emptyset$.
For $i = 1$ to $n$ do
	if $( a(s) + a_i \leq b) $ then
		$S = S \cup \{ i \}$
	endif

endfor
return $S$ and $c(S)$
\end{lstlisting}

\begin{thm}
	The greedy algorithm does \emph{ not } guarantee an optimal solution.
\end{thm}

\begin{proof}
	Let $b=10$, $n = 6$
	
	\begin{align*}
		\begin{tabular}{l|ccccc}
			$i$ & 2 & 3 & 4 & 5 & 6 \\ \hline
			$a_i$ & 9 & 2 & 2 & 2 & 2 \\
			$c_i$ &19 & 4 & 4 & 4 & 4
		\end{tabular}
	\end{align*}
	
	Greedy: $S=\{1\}$, $c(s)=20$
	
	Optimal: $S=\{2,3,4,5,6,\}$, $c(S)=20$
\end{proof}

Approach 2: Integer Linear Programming

The set of solutions $X$ of a combinatorial optimization problem can (almost always) be written as the intersection of integer points in $\N_0^n$ and a polyhedron $\{x \in \R^n: Ax\leq b\}$

Let $x \in \{0,1\}^n$ be a vector representing all solutions of the knapsack problem:
\[
	x_i = \begin{cases}
		1 \qquad \text{if} i \in S \\
		0 \qquad \text{otherwise}
	\end{cases}
\]

$X = \{0,1\} \cap \{ x \in \R^n: \sum\limits_{i = 0}^n a_ix_i\leq b\}$

Knapsack: $\max \sum{i = 0}^n c_i x_i$

The \new{linear relaxation} (Lineare Relaxierung) of an ILP is the linear program optained by relaxing the integrality of the variables:

$\max \sum_{i=1}^n c_i x_i$

s. t. $\sum_{i=1}^n a_i x_i \leq b, 0 \leq x_i \leq 1 \qquad \forall i \in \{1, …, n\}$

\begin{thm}
An optimal solution $\tilde{x}$ of the linear relaxation of the knapsack
problem is:

There exists a $k \in \{1, …, n\}$ such that 
\[
\tilde{x}_i = \left \{ \begin{array}{ll}
1 & \text{if } i \leq k \\ 
0 & \text{if } i > k+1 \\
(b-\sum^k_{i=1} a_j) / a_{k+1} & \text{if } i=k+1
\end{array} \right.
\]
where $c_1/a_1 \geq c_2 / a_2 \geq … \geq c_n / a_n$.
\end{thm}

\begin{proof}
Let $x^*$ be an optimal solution with $c^T c^* > c^T \tilde{x}$.
If $x^*_i < 1$ for $i \leq k$, there must exist a $j \geq k+1$ with $x^*_j >
\tilde{x}_j$.

We define $\bar{x}$ with $\varepsilon \leq x^*_j - \tilde{x}_j$ as
\[
\bar{x}_l = \left \{ \begin{array}{ll}
x^*_k & \text{for } l \not\in \{i, j\} \\
x^*_l - \varepsilon & \text{for } l=j \\
x^*_l + \frac{a_j}{a_l} \cdot \varepsilon & \text{for } l=i
\end{array} \right.
\]
Then $\bar{x}$ is feasible and
\[
c^T \bar{x} = \sum_{l=1}^n c_l \bar{x}_l =
\sum_{l=1}^n c_l x^*_l + \underbrace{c_i \cdot \frac{a_j}{a_i} \varepsilon - c_j
\varepsilon}_{\geq 0} \geq c^T x^* > c^T \tilde{x}
\]

Repetition yields $c^T \bar{x} > c^T \bar{x}$, a contradiction.
\end{proof}

Note:

If $\tilde{x}$ is integer valued, then the solution is also optimal for the
knapsack problem. In this case, also the greedy algorithm is optimal.

Approach 3: Dynamic Programming

A dynamic program algorithm to solve a problem first solves similar, but
smaller subproblems in order to use their solution to solve the original
problem.

The problem should conform to the \emph{optimality principle} of Bellman:
Given an optimal solution for the original problem, a partial solution
restricted to a subproblem is also optimal for the subproblem.

Let $f_k(b)$ be the optimal solution value of the knapsack problem with
total weight equal to $b$ and items from $\ouptoset{k}$.

\begin{thm}
$f_{k+1}(b) = max \{ f_k(b), f_k(b-a_{k+1} + c_{k+1} \}$.
\end{thm}
\begin{proof}
An optimal solution of $f_{k+1}(b)$ either contains item $k+1$ or not.
If $k+1$ is not contained, the problem is identical to $f_k(b)$.
If $k+1$ is contained, other items in the solution should have total weight
$b - a_{k+1}$.

Hence, $f_k(b - a_{k+1})$ is an optimal solution for the remaining items $+
c_{k+1}$ for the item $k+1$.
\end{proof}

\begin{cor}
The knapsack problem can be solved in $O(nb)$ with value $max_{d=0, ..., b}
f_n(d)$.
\end{cor}


\newpage

\printglossaries

\end{document}
