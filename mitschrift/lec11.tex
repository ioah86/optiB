\begin{lec}[2011-11-21]\end{lec}
\stepcounter{section}

\section*{Edmonds' Matching Algroithm}
Let $G$ be a graph, $M$ a matching in $G$ and $W$ the set of unmatched vertices.
A $M$-augmenting path is a $M$-alternating $W$-$W$ chain of positive length 
where all vertices are distinct.

We call a $M$-alternating chain $P=\plainset{v_0, v_0, ..., v_t}$ a \new{$M$-blossom}{$M$-Blüte}
if $v_0, ..., v_{t-1}$ are dstinct, $v_0$ is not matched in $M$ and $v_t = v_0$



%11.1
\begin{thm}
	Let $C$  be a $M$-blossom in $G$. $M$ is a maximum matching if and only if 
	$M \setminus C$ is a maximum matching in $G \setminus C$.
\end{thm}

\begin{proof}
	Let $C = \plainset{v_0, v_1, ..., v_t}$, $G' = G\setminus C$, $M' = M \setminus C$.
	First, let $P$ be a $M$-augmenting path in $G$.
	W.l.o.g we may assume that $P$ does not start in $v_0$ (therwise turnaround
	$P$).
	If $P$ does not visit any vertex of $C$, then $P$ is a $M$-augmenting
	path in $G'$ as well.
	If $P$ visits a vertex of $C$, then we can decompose $P$ in $Q$ and $R$
	with $Q$ ending at the first vertex of $C$.
	Replace this vertex in $Q$ with vertex $C$ in $G'$. Then Q is a $M$-
	augmenting path in $G'$.
	Now, on the reverse, let $P'$ be an $M'$-augmenting path in $G'$.
	If $P'$ does not visit vertex $C$, then $P'$ is $M$-augmenting path in $G$.
	If vertex $C$ is visited by $P'$, we may assume that $C$ is the end
	vertex (since $M'$ does not cover $C$). We thus can replace vertex $C$
	with a suitable vertex $v_i \in C$ such that the new path $Q$ ends at
	$v_2$ in $G$.
	If $i$ is odd, then we extend $Q$ with $v_{i+1}, v_{i+2}, ..., v_t = v_0$.
	If $i$ is even, then we extend $Q$ with $v_{i-1}, v_{i-2}, ..., v_0$.
	The resulting path is a $M$-augmenting path in $G$
\end{proof}
\section*{Edmonds' Algorithm}
Input: Mathing $M$ with set $W \subseteq V$ of unmatched vertices
Output: Mathing $N$ with $|N| = |M| + 1$ or a certificate that $M$ is a
maximum matching.
Description:
\begin{lstlisting}
Case 1: No $M$-alternating $W$-$W$ chain exists:
	Then $M$ is an maximum matching. 
	STOP.
Case 2: A $M$-alternating $W$-$W$ chain exists:
	Let $P = (v_0, v_1, ..., v_t)$ be a shortest $M$-alternating 
	$W$-$W$ chain
	Case 2A: $P$ is a path.
		Then $P$ is $M$-augmenting and $N:=M \Delta P$.
	Case 2B: $P$ is not a path.
		Choose $i<j$ with $v_i = v_j$ 
		and $j$ as small as possible
		Replace $M$ by $M \Delta \plainset{v_0, ..., v_i}$
		Then $C:=\plainset{v_i, v_{i+1}, ..., v_j}$ is a $M$-blossom.
		Apply the Algorithm recursively 
		until $G':=G \setminus C$ and $M' := M \setminus C$.
		
		* If a $M'$-augmenting path $P$ in $G'$
		  is returned, transform $P'$ to a 
		  $M$-autmenting path in $G$ 
		  (by \todo{Thm 11.1})
		* If $M'$ is a maximum matching in $G'$, 
		  then $M$ is a maximum matching in $G$
		  (\todo{Thm 11.1})
\end{lstlisting}
%11.2
\begin{thm}
	Edmonds' Algorithm is correct and needs at most $\frac12 |V|$ repetitions to find a maximum matching.
\end{thm}

\begin{proof}
	Follows directly from \todo{thm 7.11} and \todo{thm 11.1}.
\end{proof}

How do we find a $M$-alternating $W$-$W$ chain?
%11.3
\begin{defn}
	A \new{$M$-alternating forest}{$M$-alternierender Wald} $(V,F)$ is a
	forest with $ M \subseteq F$, each component of $(V,F)$ contains 
	\emph{either} a vertex for $W$ \emph{or} exists of a single edge in $M$,
	and each path in $(V,F)$ starting with a vertex in $W$ is 
	$M$-alternating.
	
	Let \begin{align*}
		even(F) =& \plainset{ v \in V: F \text{ contains a $W$-$v$ path of even length}} \\
		odd(F) =& \plainset{ v \in V: F \text{ contains a $W$-$v$ path of odd length}} \\
		free(F) =& \plainset{ v \in V: F \text{ does not contain a $W$-$v$ path}} 
	\end{align*}
	
	\todo{insert Figure ELISA 1}
	
	Note that each vertex $u \in odd(F)$ is incident to a unique edge in
	$F \setminus M$ and a unique edge in $M$.
\end{defn}

%11.4
\begin{lem}
	If no edge $e \in E$ exists that connects $even(F)$ with $even(F) \cup 
	free(F)$, then $M$ is a maximum matching.
\end{lem}

\begin{proof}
	If no such edge exists, then even(F) is a stable set in $G - odd(F)$.
	In fact every vertex $u \in even(F)$ is an odd component in $G - odd(F)$.
	Let $U:=odd(F)$.
	\begin{align*}
		& o(G-U) \geq | even(F)| = |W| + |odd(F)| = |V| - 2 |M| + |U| \\
		\Leftrightarrow& 2 |M| \geq |V| + |U| - o(G-U).
	\end{align*}
	By Tutte-Berge-Formula, $M$ is a maximum matching.
\end{proof}
\section*{Construction of an $M$-alternating forest}
Intialization: $F:=M$. Choose for every vertex $v \in V$ an edge $e_v = vu$ with $u \in W$ (if possible).

Iterate: Find $v \in even(F) \cup free(F)$ for which $e_v=vu$ exists. \newline
Case 1 : $v \in free(F)$: Add $uv$ to $F$. Let $vw\in M$. For all $wx \in E$, set $e_x = \plainset{x,w}$ \newline
Case 2: $v \in even(F)$: Find $W-u$ respectively $W-v$ paths $P$ and $Q$ in $F$ \newline
2a: If $P$ and $Q$ are disjoint, then $F \cup \plainset{uv} \cup Q$ is a $M$-augmenting path \newline
2b: If $P$ and $Q$ are not disjoint, then $P \cup Q \cup \plainset{uv}$ contains a $M$-blossom $C$. For all edges $cx$ with $c \in C$ and $x \not \in C$, set $e_x = C_x$. Replace $G$ by $G \setminus C$.
\todo{insert Figure ELISA 2}
\begin{align*}
	F=M\cup \plainset{} \\
	W=\plainset{1,16,18}
\end{align*}
{
\tiny
\begin{tabular}{ c | c | c | c | c | c | c | c | c | c |c | c | c | c | c |c | c | c | c | c |c | c | c | c | c}
  v & 1 & 2 & 3 & 4 & 5 & 6 & 7 & 8 & 9 & 10 & 11 & 12 & 13 & 14 & 15 & 16 & 17 & 18 & 19 & 20 & 21 & 22 & 23 & 24 \\
  \hline 
  $\plainset{v,u}=e_v$ & & 1 & & & & & & & & 18
\end{tabular}
}