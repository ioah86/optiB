\begin{lec}[2011-10-26]\end{lec}

\setcounter{section}{6}
\setcounter{subsection}{0}

\subsubsection*{Knapsack problem} 

\begin{defn} %6.1
	The \new{Knapsack problem} is defined by a set of items $N = \{ 1, ... , n \}$ weights $a_i \in \N$, value $c_i \in \N$, and a bound $b\in \N$. We search for a subset $S\subset \N$ such that 
	\[
		a(S) = \sum_{i \in S} a_i \leq b \; \text{and} \; c(S) = \sum_{i\in S} c_i \; \text{maximum}
	\]
\end{defn}

Appreach 1: Greedy algorithm

Idea: Items with small weight but high value are the most atrractive ones.

Procedure:
\begin{enumerate}
	\item Sort the items such that $\frac{c_1}{a_1} \leq \frac{c_2}{a_2} \leq ... \leq \frac{c_n}{a_n}$.
	
	Set $S = \emptyset$.
	\item For $i = 1$ to $n$ do
		if $( a(s) + a_i \leq b) $ then
		
			$S = S \cup \{ i \}$
			
		endif
	
	endfor
	\item return $S$ and $c(S)$
\end{enumerate}

\begin{thm}
	The greedy algorithm does \emph{ not } guarantee an optimal solution.
\end{thm}

\begin{proof}
	Let $b=10$, $n = 6$
	
	\begin{align*}
		\begin{matrix}
			i   & 2 & 3 & 4 & 5 & 6 \\
			a_i & 9 & 2 & 2 & 2 & 2 \\
			c_i &19 & 4 & 4 & 4 & 4
		\end{matrix}
	\end{align*}
	
	Greedy: $S=\{1\}$, $c(s)=20$
	
	Optimal: $S=\{2,3,4,5,6,\}$, $c(S)=20$
\end{proof}

Approach 2: Integer Linear Programming

The set of solutions $X$ of a combinatorial optimization problem can (almost always) be written as the intersection of integer points in $\N_0^n$ and a polyhedron $\{x \in \R^n: Ax\leq b\}$

Let $x \in \{0,1\}^n$ be a vector representing all solutions of the knapsack problem:
\[
	x_i = \begin{cases}
		1 \qquad \text{if} i \in S \\
		0 \qquad \text{otherwise}
	\end{cases}
\]

$X = \{0,1\} \cap \{ x \in \R^n: \sum\limits_{i = 0}^n a_ix_i\leq b\}$

Knapsack: $\max \sum{i = 0}^n c_i x_i$

The \new{linear relaxation} (Lineare Relaxierung) of an ILP is the linear program optained by relaxing the integrality of the variables:

$\max \sum_{i=1}^n c_i x_i$

s. t. $\sum_{i=1}^n a_i x_i \leq b, 0 \leq x_i \leq 1 \qquad \forall i \in \{1, ..., n\}$
