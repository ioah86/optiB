\begin{lec}[2011-10-20]\end{lec}

\begin{thm} % 3.4
Bellman's Algorithm is correct and runs in $O(m + n) = O(n)$.
\end{thm}
\begin{proof}
of runtime:

\[ D(i) = \min\limits_{(i,j) \in A} D(j) + D(i, j) \]

$\Rightarrow$ Every arc is considered once, and thus overall $O(m)$
computations are needed. Initialization costs $O(n)$.
\end{proof}

\begin{bem+}
The running time does not contain the time to find the permutation.
\end{bem+}

Observation 1: We not only found the shortest path from $1$ to $n$, but
also from $i$ to $n$, $i = 2, ..., n$.

Observation 2: We can use a similar procedure for the shortest path from
$1$ to $i$, $i = 2, ..., n$. (with $PREV(i)$ for previous instead of
$NEXT(i)$).

\begin{qstn}
Can we find a shortest path from $1$ to $i$ in a digraph that is
not acyclic, i.e. it contains cycles?
\end{qstn}

\stepcounter{section}

\begin{thm} % 4.1
The Moore-Bellman-Algorithm returns the shortest paths from $1$ to $i = 1,
..., n$ provided $D$ does not contain negative-weighted directed cycles.
\end{thm}
\begin{proof}
We call an arc $(i, j) \in A$ an \emph{upgoing} arc (Aufwärtsbogen) if $i < j$ and a
\emph{downgoing} arc (Abwärtsbogen) if $i > j$.

A shortest path from $1$ to $i$ contains at most $n-1$ arcs. If an upgoing
arc is followed by a downgoing arc (or vice versa), we have a \emph{change
of direction} (Richtungswechsel). With at most $n-1$ arcs, at most $n-2$
changes of direction are possible.

Let $D(i, m)$ be the value of $D(i)$ at the end of the $m$-th iteration.
We will show (and this is enough):
\[ D(i, m) = min \{ c(W): \textrm{$W$ is the directed $[1,i]$-path with at most $m$
changes of directions} \} \]

We prove it by induction on $m$.

\begin{itemize}
\item For $m = 0$, the algorithm is equivalent to Bellman's algorithm for acyclic
grpahs. Thus, $D(i, 0)$ is the length of the shortest path without any
changed of direction.

\item Now, let us assume, that the statement is true for $m \geq 0$ and the
subroutine is executed for the $m+1$-st time. The set of $[1,i]$-paths with
at most $m+1$ changes of direction consists of 
\begin{enumerate}[(a)]
\item\label{item:bellman:a} $[1,i]$-paths with $\leq m$ changes of direction
\item\label{item:bellman:b} $[1,i]$-paths with exactly $m+1$ changes of direction
\end{enumerate}
$\Rightarrow D(i, m)$

\item Since every path starts with an upgoing arc $(1, k)$, the last arc
after $m+1$ changes is either a downgoing arc if $m+1$ is odd or an upgoing
arc if $m+1$ is even. We restrict ourselves to $m+1$ odd ($m+1$ even is
similar).

To compute the minimum length path in (\ref{item:bellman:b}) we use an
additional induction on $i= n, n-1, ..., j+1$. Since every path ending at
$n$ ends with an upgoing arc, there do not exist such $[1,n]$-paths.
Hence, $D(n, m+1) = D(n, m)$.

Now assume that $D(k, m+1)$ is correctly computed for $i \leq k \leq n$. The
shortest path from $1$ to $i-1$ with exactly $m+1$ changes ends with a
downgoing arc $(j, i-1)$, $j > i-1$.

$D(j, m+1)$ is already computed correctly. If $PREV(j) > j$, no change of
direction is required in $j$ and $D(i-1, m+1) = D(j, m+1) + c(j,i-1)$
If $PREV(j) < j$, the last avec of the $[1,j]$-Path is upgoing, and thus $D(i-1,m+1)=D(j,m)+c(j,i)$. The last change of direction at $j$ is thus, in worst case, the $(m+1)$-st change. Hence, $D(i-1,m+1)$ fulfills the statement. 
\end{itemize}
\end{proof}

\begin{rem}
In fact, the algorithm finds the minimum length of a chain (kette) with at most $n-2$ changes of direction. In case of negative weighted cycles these might be in a chain several times. 

In case no negative weighted cycles exist, the min. length chains are indeed paths. Hence, the algoithm only works correctly if \emph{all} cycles are non-negative weighted.
\end{rem}

\begin{rem}
If a further executing of the subroutine $(m=n-1)$ results in at least one change of a value $D(i)$, then the digraph contains negative weighted cycles.
\end{rem}

\begin{rem}
A more efficient implementation is given by E'sopo-Pape-Variant.
\end{rem}

\tocsubsection{Dijkstra's Algorithm for non-negative weights}

\begin{thm}
Dijkstra returns the shortest paths from $1$ to $i, i= 1 ... n$, provided all weights $\geq 0$.
\end{thm}

\begin{proof}
Each step, one vetex is moved from $T$ to $S$. At the end of a step, $D(j)$ is the shortest path from $1$ to $j$ via vertices in $S$. 

If $S=V (T= \emptyset), D(i)$ is thus the shortiest $[1,i]$-path
\end{proof}
