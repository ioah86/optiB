\begin{lec}[2011-12-01]\end{lec}
\stepcounter{section}

\tocsection{Flows in Networks}

\tocsubsection{Menger's Theorem}

\begin{defn}
Let $D = (V, A)$ be a digraph and $S$, $T$ subsets of $V$.
A path is called a $S$-$T$-path, if the start vertex is in $S$ and the end
vertex is in $T$.

If $S = \{s\}$ and $T = \{t\}$ we also refer to the path as $s$-$t$-path
instead of $\{s\}-\{t\}$-path.
\end{defn}

\begin{defn}
Two $S$-$T$-paths $P_1$ and $P_2$ are called
\new{vertex disjoint}{knotendisjunkt} if $P_1$ and $P_2$ have no common
vertices.

Two $S$-$T$-paths are called internally vertex disjoint if they have no common
vertices, except for the start and end vertices.

Two $S$-$T$-paths are called \new{arc disjoint}{bogendisjunkt} if they have no
arcs in common.
\end{defn}
\emph{Question:} How many vertex/arc disjoint paths exist between $S$ and
$T$ resp. $s$ and $t$?

\begin{defn} 
A set $C \subset V$ \new{separates}{} $S$ from $T$ if every
$S$-$T$-path intersects with $C$ ($C$ can intersect $S \cup T$). $C$ is called
a \new{($S$-$T$)-separator}{}. 
\end{defn}

\begin{thm}(Menger's Theorem, directed vertex disjoint version)\\
Let $D=(V,A)$ be a digraph and $S,T \subset V$. Then, the maximum number of
pairwise vertex disjoint $S$-$T$-paths equals the minimum size of a
$S$-$T$-separator.
\end{thm}

\begin{proof}
Clearly, the number of vertex disjoint paths cannot exceed the size of a
$S$-$T$-separator $C$ (i.e. for $v \in C$, at most one path exists). We will
show $\geq$ by induction on $|A|$. For $|A|=0$, the statement ist trivial.
Now, let $k$ be the minimum size of a $S$-$T$-separator. Select $a=(u,v) \in
A$ arbitrarely. If every $S$-$T$-separator in $D \setminus a$ has size $\geq
k$, then by incduction $k$ vertex-disjoint paths exist in $D\setminus a$,
and thus in $D$ as well.\\
Thus, we can assume w.l.o.g that $D \setminus a $ has a $S$-$T$-separator $C$
with $|C| \leq k-1$. Then $C\cup \{u\}$ and $C\cup \{v\}$ are
$S$-$T$-separators in $D$ of size $k$.\\
Now, every $S$-$(C\cup\{u\})$-separator $B$ of $D \setminus a$ has size at
least $k$, since $B$ also separates $S$ from $T$ in $D$: every $S$-$T$-path in
$D$ intersects $C\cup\{u\}$ and thus $P$ contains $S$-$(C\cup \{u\})$-subpath
in $D\setminus a$. Therefore, the subpath and thus $P$ itself intersects
$B$. By induction $D\setminus a$ has $k$ vertex disjoint
$S$-$(C\cup\{u\})$-paths. Similarly there exist $k$ vertex disjoint
$(C\cup\{v\})$-$T$-paths in $D \setminus a$. Since both path sets use all
vertices in $C$, we can connect these $k-1$ $S$-$C$-paths with the $C$-$T$-paths.
One more path in $D$ can be established by connecting the $S$-$u$-path with
the $v$-$T$-path via arc $a=(u,v)$.
\end{proof}

\begin{defn}
A set $U\subset V$ is called a \new{($s$-$t$)-vertex cut}{} if $s,t \notin u$
and every $S$-$T$-path intersects $U$.
\end {defn}

\begin{cor}
Let $D=(V,A)$ be a digraph and $s,t$ two non-adjacent vertices. then the
maximum number of \new{internally vertex disjoint}{} $S$-$T$-paths equals the
minimum size of a $S$-$T$-vertex cut.
\end{cor}
\begin{proof}
Let $D'=D\setminus s - t.$
$S=N^+_D(s), \qquad T=N_D^-(t)$. Apply %TODO Thm 14.4
\end{proof} 

\begin{defn}
A arc set $C\subset A$ define a \new{($s$-$t$)-cut}{($s$-$t$)-Schnitt} if a subset
$U\subset V, s \in U, t \notin U$ with $\delta^+(U)\subset C$.
\end{defn}

\begin{cor}(Menger's Theorem, directed arc version)
let $D=(V,A)$ be a digraph with $s,t  \in V$. Then the maximum number of arc
disjoint $S$-$T$-paths equals the minimum size of a $S$-$T$-cut.
\end{cor}
\begin{proof}
Übungsblatt
\end{proof}

\emph{Question:} How to find arc disjoint $S$-$T$-paths?
\emph{Answer:}Given $D$ a digrpah and a path $P$, let us define $D/P$ as the
digraph in which all arcs of $P$ are reversed.

\begin{lstlisting}
Initialize $D_0:=D, \qquad k=0$;
WHILE $D_k$ contains a $S$-$T$-path $P_k$ DO
	$D_{k+1}:=D_k/P_k$
	$k:=k+1$
ENDWHILE
The reversed arcs in $D-k$ are $k-1$ arc disjoint $S$-$T$-paths.
\end{lstlisting}
Note: A $S$-$T$-path $P_k$ in $D_k$ can be found with e.g. Dijkstra. 
