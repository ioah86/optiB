\begin{lec}[2011-10-24]\end{lec}

\subsubsection*{Shortest paths between all pairs of vertices}

Solution 1: Apply Moore-Bellman or Dijkstra to all vertices $i$ as starting vertex

Solution 2: Apply Floyd's Algorithm

Notation: 

$w_{ij}$ is the length of the shortest $[i,j]$-path, $i \neq j$

$w_{ii}$ is the length of the shortest directed cycle containing $i$

$p_{ij}$ is the predecessor of $j$ on the shortest $[i,j]$-path (cycle)

$W =(w_{ij}) $ is the \new{shortest path length matrix}{???}

%\setcounter{section}{5}
%\setcounter{subsection}{0}

\stepcounter{section}

\begin{thm} %5.1
The Floyd Algorithm works correctly if and only if $D = (V,A)$ does not contain any negative weighted cycles.

$D$ contains a negative weighted cycle if and only if one of the diagonal elements $w_{ii}<0$.
\end{thm}

\begin{proof}
\newcommand{\wii}[0]{w_{ii}}
\newcommand{\wij}[0]{w_{ij}}

Let $W^k$ be the matrix $W$ after iteration $k$, with $W^0$ being the initial matrix. By induction on $k = 0, …, n$ we show that $W^k$ is the matrix of shortest path lengths with vertices $1, …, k$ as \emph{possible} internal vertices, provided $D$ does not contain a negative cycle on these vertices.

If $D$ has a negative cycle, then $\wii^k < 0$ for an $i \in \{ 1, …, n\}$

For $k = 0$, the statement clearly true.

Assume, it is correct for $k \geq 0$, and we have executed the $(k+1)$st iteration.

It holds that $\wij^{k+1} = \min\{\wij^k, w_{i,k+1}^k + w_{k+1,j}^k\}$. Note that, provided no negative cycle exists, $w_{i,k+1}^{k+1}$ does not have any vertex $k+1$ as internal vertex, and thus $w_{i,k+1}^{k+1} = w_{i,k+1}^k$ (similarly, $w_{k+1,j}^{k+1} = w_{k+1,j}^k$).

$w_{i,k+1}^k$ is the minimal length of a $[i,k+1]$-path with $\{1, …, k\}$ as allowed internal vertices. Similarly, $w_{k+1,j}^k$.

Thus, $w_{i,k+1}^k+w_{k+1,j}^k$ is the minimal length of an $[i,j]$-path (not necessarily simple) containing $k+1$ (mandatory) and $\{ 1, …, k \}$ (voluntary). If the shortest path from $i$ to $j$ using $\{1, …, k+1\}$ does not contain $k+1$, it only contains $\{1, … , k\}$ (voluntary) and, hence, $\wij^k$ is the right value.

What remains to show is that the connection of the $[i,k+1]$-path with the $[l+1,j]$-path is indeed a simple path.

Let $K$ be this chain. After removal of cycles, the chain $K$ contains (of course) a simple $[i,j]$-path $\bar K$. Since such cycles may only contain vertices from $\{1, …, k+1\}$, one cycle must contain $k+1$.
If this cycle is not negatively weighted, then path $\bar K$ is shorter and $\wij< w_{i,k+1}^k + w_{k+1,j}^k$.

If this cycle is negatively weighted, $w_{k+1,k+1}^k<0$ (the cycle only contains internal vertices from $\{1, …, k\}$) and algorithm would have stopped earlier.
\end{proof}

\subsection*{Min-Max-Theorems for combinatorial Optimization Problems}

From "Optimierung A": Duality of linear programs

\[\max_{\text{s. t.}, Ax \leq b, x \geq 0} c^Tx = \min_{\text{s.t.}, A^T y \geq c, y \geq 0} b^T y\]

For several combinatorial problems $\min\{c(x): x \in X\}$

We can define a second set $Y$ and a function $b(y)$ with $\max\{b(y): y \in Y \} = \min \{ c(x): x \in X\}$ where Y and b(y) have a graph theoretical interpretation. 

Existence of such a "Dual" Problem indicates often that the problem can be solved "efficiently". For the shortest path problem several max-min-theorems exist.

\begin{defn}
An \new{$(s,t)$-cut}{$(s,t)$-Schnitt} in a digraph $D=(V,A)$ with $s,t \in V$ is a subset $B \subset A$ of the arcs with the property that every $(s,t)$-path contains at least one arc of $B$.

Stated  otherwise, for every cut $B$, there exists a vertex set $W \subset V$ such that
\begin{itemize}
	\item $s \in W$, $t \in V \setminus W$
	\item $\delta^+(w) = \{(i,j) \in A: i \in W, j \in V \setminus W\} \subseteq B$
\end{itemize}
\end{defn}

\begin{thm}
	Let $D=(V,A)$ be a digraph, $c(a)=1 \, \forall a \in A, s,t \in V, s \neq t$. Then the minimum length of a $[s,t]$-path equals the maximum number of arc-disjoint $(s,t)$-cuts. 
\end{thm}

\begin{proof}
	Folows from \ref{!!!} %TODO! 5.4
\end{proof}

\begin{thm} %5.4
	Let $D=(V,A)$ be a digraph, $c(a) \in \Z_+ \, \forall a \in A \wedge s,t \in V \wedge s\neq t$. Then the $\min$ length of an $[s,t]$-path equals the maximum number $d$ of (not necessarily different) $(s,t)$-cuts $C_1, …, C_d$ such that every arc $a \in A$ is contained in at most $c(a)$ cuts.
\end{thm}

\begin{proof}
	We define $(s, t)$-cuts $C_i = \delta^+(v_i)$ with $V_i=\{v \in V: \exists \text{$(s,v)$-path with $c(P)\leq i-1$}\}$
	\begin{align*}
		v_1 & = \{s\} \\
		v_2 & = \{5,3,4\} \\
		v_3 & = \{5,2,3,4\} \\
		v_4 & = v_3 \cup \{ 6 \}
	\end{align*}
	
	(for the example graph on the board)
	
	The shortest $[s,t]$-path $P$ consists of arcs $a_1, … a_k$ with arc $a_j$ contained in $(s,t)$-cuts $C_i$,  $i \in \{\sum_{l=1}^{j-1} c(a_l) + 1, …, \sum_{l=1}^j a(a_l)\}$: exactly $c(a)$ cuts.
\end{proof}	
