\begin{lec}[2012-01-12]\end{lec}
\stepcounter{section}

\begin{qstn}
How should $A$ resp. $b$ look like to have
\[
	\max \{ c^T x \mid A x \leq b, x \in \Z^n \} = \max \{ c^T x \mid Ax \leq b, x \geq 0 \}
\]
\end{qstn}
Answer (partly):
A should be a \new{totally unimodular}{total unimodular} matrix.

\begin{defn}
A matrix $A$ is totally unimodular (TUM) if every $k \times k$ submatrix of
$A$ has determinant equal to $-1$, $0$ or $1$.
In particular, every entry of $A$ has to be $-1$,$0$ or $1$.
\end{defn}

% slides for ILPs

\begin{thm}
Let $A$ be a TU $m \times n$ matrix and let $b \in \Z^m$.
Then every vertex of the polyhedron $P = \{ x \in \R^n \mid A x \leq b \}$
is integer.
\end{thm}
\begin{proof}
Let $z$ be a vertex of $P$ and define $A_z$ as the submatrix of $A$
consisting of the rows $a_i$ of $A$ with $a_i z = b_i$. From linear
programming theory, it follows that $z$ is a vertex of $P$ if and only if
$rang(A_z) = n$.

Consequently, $A_z$ contains a non-singular $n \times n$ submatrix $A'$.
Let $b'$ be the corresponding subvector of $b$.
By definition, $A'z=b'$ and accordingly, $z = (A')^{-1} b'$.

From TU of $A$, it follows that $|\det(A')| = 1$ and all entries of
$(A')^{-1}$ are integer (Cramer's rule). Since $b \in \Z^n$, it follows $z
\in \Z^n$.
\end{proof}

\begin{defn}
A polyhedron $P$ is \new{integer}{ganzzahlig} if for all $c \in \R^n$ with
$\max\{ c^T x \mid x \in P \}$ finite the maximum is attained by an integer
vector $x$.

Stated otherwise, if $P = \{ x \mid Ax \leq b \}$ with $A$ a matrix of rang
$n$, then $P$ is integer if and only if all vertices of $P$ are integer.
\end{defn}

\begin{cor}
If $A$ is TU and $b \in \Z^m$, then $P = \{ x \mid Ax\leq b \}$ is integer.
\end{cor}

\begin{thm}[Hoffman-Kruskal, 1956]
Let $A$ be an integer $m \times n$ matrix.
Then $A$ is TU if and only if for all $b \in \Z^m$, $P = \{x \mid Ax \leq b,
x \geq 0\}$ is integer.
\end{thm}
\begin{proof}
To prove the forward direction, we define \\
$\overline{A} = \begin{pmatrix} A \\ -\I \end{pmatrix}$ with $\I$ the $n
\times n$ identity matrix and $\overline{b} = \begin{pmatrix} b \\ 0
\end{pmatrix}$.
Then $P = \{ x \mid \overline{A} x \leq \overline{b} \}$.

Every square submatrix of $\overline{A}$ consists of a part of the identity
matrix and a part of $A$. To compute the determinant, we perform a Laplacian
expansion along the rows of the identity part.

Hence, the determinant equals either zero or plus/minus the determinant of a
smaller submatrix. Thus $\overline{A}$ is TU if and only if $A$ is TU.
% todo: fix reference
By Corollary 22.4, $P$ is integer.

For the reverse, assume that $A$ is not TU.
\end{proof}

