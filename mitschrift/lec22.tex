\begin{lec}[2012-01-12]\end{lec}
\stepcounter{section}

\begin{qstn}
How should $A$ resp. $b$ look like to have
\[
	\max \{ c^T x \mid A x \leq b, x \in \Z^n \} = \max \{ c^T x \mid Ax \leq b, x \geq 0 \}
\]
\end{qstn}
Answer (partly):
A should be a \new{totally unimodular}{total unimodular} matrix.

\begin{defn}
A matrix $A$ is totally unimodular (TUM) if every $k \times k$ submatrix of
$A$ has determinant equal to $-1$, $0$ or $1$.
In particular, every entry of $A$ has to be $-1$,$0$ or $1$.
\end{defn}

% slides for ILPs

\begin{thm}
Let $A$ be a TU $m \times n$ matrix and let $b \in \Z^m$.
Then every vertex of the polyhedron $P = \{ x \in \R^n \mid A x \leq b \}$
is integer.
\end{thm}
\begin{proof}
Let $z$ be a vertex of $P$ and define $A_z$ as the submatrix of $A$
consisting of the rows $a_i$ of $A$ with $a_i z = b_i$. From linear
programming theory, it follows that $z$ is a vertex of $P$ if and only if
$rang(A_z) = n$.

Consequently, $A_z$ contains a non-singular $n \times n$ submatrix $A'$.
Let $b'$ be the corresponding subvector of $b$.
By definition, $A'z=b'$ and accordingly, $z = (A')^{-1} b'$.

From TU of $A$, it follows that $|\det(A')| = 1$ and all entries of
$(A')^{-1}$ are integer (Cramer's rule). Since $b \in \Z^n$, it follows $z
\in \Z^n$.
\end{proof}

\begin{defn}
A polyhedron $P$ is \new{integer}{ganzzahlig} if for all $c \in \R^n$ with
$\max\{ c^T x \mid x \in P \}$ finite the maximum is attained by an integer
vector $x$.

Stated otherwise, if $P = \{ x \mid Ax \leq b \}$ with $A$ a matrix of rang
$n$, then $P$ is integer if and only if all vertices of $P$ are integer.
\end{defn}

\begin{cor}\label{cor22.4}
If $A$ is TU and $b \in \Z^m$, then $P = \{ x \mid Ax\leq b \}$ is integer.
\end{cor}

\begin{thm}[Hoffman-Kruskal, 1956]
Let $A$ be an integer $m \times n$ matrix.
Then $A$ is TU if and only if for all $b \in \Z^m$, $P = \{x \mid Ax \leq b,
x \geq 0\}$ is integer.
\end{thm}
\begin{proof}
To prove the forward direction, we define \\
$\overline{A} = \begin{pmatrix} A \\ -\I \end{pmatrix}$ with $\I$ the $n
\times n$ identity matrix and $\overline{b} = \begin{pmatrix} b \\ 0
\end{pmatrix}$.
Then $P = \{ x \mid \overline{A} x \leq \overline{b} \}$.

Every square submatrix of $\overline{A}$ consists of a part of the identity
matrix and a part of $A$. To compute the determinant, we perform a Laplacian
expansion along the rows of the identity part.

Hence, the determinant equals either zero or plus/minus the determinant of a
smaller submatrix. Thus $\overline{A}$ is TU if and only if $A$ is TU.
% todo: fix reference
By Corollary 22.4, $P$ is integer.

For the reverse, assume that $A$ is not TU.
Then, there exists a $r \times r$ submatrix $B$ with $\det(B) \not\in
\{-1,0,1\}$ and $r \leq \min\{n,m\}$.
W.l.o.g. let $B = (a_{ij})_{1 \leq i,j \leq r}$.

Since $B$ is non-singular, $B^{-1}$ exists. By integrality of $A$, $\det(B)$
is integer and $\det(B) \cdot \det(B^{-1}) = 1$, $\det(B^{-1}) \not\in \Z$.
Thus $B^{-1}$ has at least one column $j$ with a non-integer entry: 
$B^{-1} e_j \not\in \Z^r$.

We construct $b' \in \Z^r$ as follows. Find $y \in \Z^r$ such that $B^{-1}
e_j + y \geq 0$ and define $b' = (b_1', \dots, b_r') = B ( B^{-1} e_j + y) =
e_j + B y \in \Z^r$.

But the unique solution to $Bz=b'$ is $z = B^{-1}e_j+y \not\in \Z^r$. \\
 Now extend $z$ as $\overline{x}:=\begin{pmatrix} z \\ 0 \end{pmatrix} \not \in
\Z^n$ and define for $j=r+1,\dots,m, \; b_j \in \Z$ such that $a_j
\overline{x} < b_j$, e.g. $b_j = \lceil a_j \overline{x} \rceil +1$. Then $\overline b
:=(b_1', \dots, b_r', b_{r+1}, \dots, b_m) \in \Z^m.$ \\
Further $\overline x \geq 0$ and $A \overline x $, hence $\overline x \in
\{x \in \R^n \mid Ax \leq \overline b \}$ with he first $r$ rows satisfied
with equality. Since $\det(B) \neq 0$, these rows are linearly independent. By
LP theory it follows that $\overline x$ is a vertex of $P$, but $\overline x
$ ist non-integer, a contradiction.
\end{proof}

\begin{tikzpicture}[auto, node distance=2cm]
	\node (q1) [state]{$1$};
	\node (q2) [state, below right of=q1]{$2$};
	\node (q3) [state, below left of =q1]{$3$};
	\path [-]
		(q1) edge (q2) (q1) edge (q3) (q2) edge (q3)
	;
\end{tikzpicture}

\begin{tabular}{c|ccc|c}
	  & 1 & 2 & 3 & \\
	  \hline
	1 & 1 & 1 & 0 & $\in R_1$\\
	2 & 1 & 0 & 1 & $\in R_2$ \\
	3 & 0 & 1 & 1 & $\in R_3$ \\
\end{tabular}
not TU by Hoffman.
	
A bipartite graph $G=(S \cup T, E)$ looks like
\begin{tabular}{c|c|ccc|}
  		& $s_1$ & 1 & & 1 \\
$R_1$ 	& $s_2$ & & & \\
		& $\vdots$ & & 1 & \\
\hline
  		& $t_1$ & 1 & & 1 \\
$R_2$ 	& $t_2$ & & & \\
		& $\vdots$ & & 1 & \\
\end{tabular}

\begin{defn}
Let $D=(V,A)$ be a digraph. The $V \times A $ incidence matrix $M$ of $D$ is
defined by 
$M_{V,a}:=\begin{cases}
			+1 & \text{if } a=(v,w) \text{ for some } w \in V\\
			-1 & \text{if } a=(w,v) \text{ for some } w \in V\\
			0 & \text{otherwise}
		\end{cases}
$
\end{defn}

\begin{tikzpicture}[auto, node distance=2cm]
	\node (q1) [state]{$1$};
	\node (q2) [state, below right of=q1]{$2$};
	\node (q3) [state, below left of =q1]{$3$};
	\path [->]
		(q1) edge (q2) (q2) edge (q3) (q3) edge (q1)
	;
\end{tikzpicture}
$\Rightarrow$
\begin{tabular}{c|ccc|c}
	  & 1 & 2 & 3 & \\
	  \hline
	1 & 1 & 0 & -1 & $\in R_1$\\
	2 & -1 & 1 & 0 & $\in R_2$ \\
	3 & 0 & -1 & 1 & $\in R_3$ \\
\end{tabular}
