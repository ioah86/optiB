\begin{lec}[2011-12-22]\end{lec}

\begin{proof} %proof of thm 19.7
\todo{der Beweis}
\end{proof}


\stepcounter{section}
\begin{thm}
Directed Hamiltonian Cycle (DHC) is $\npoly$-complete
\end{thm}
\begin{proof}
	Verifying whether a sequence of vertices includes a DHC can be done in polynomial time. Hence, the problem is a member of $\npoly$.
	
 Next, we reduce EXACT COVER to DHC. Let $\mathcal C=\plainset{C_1, ..., C_m}$ the collection of subsets of a ground set $\mathcal X = \plainset{x_1, ..., x_k}$.
 
 We construct a digraph $D=(V,A)$ with $V = \plainset{r_0, ..., r_m} \cup \plainset{s_0, ..., s_k} \cup T$ ($T$ is to be defined).
 
 For $i = 1, ..., m$, let $C_i = \plainset{x_1, ..., x_{i_t}}$. Construct the following digraph:

\todo{Graphen von Valentina einfügen}

The orange vertices are either between $r_{i-1}$ and $r_i$ in a Hamilton Cycle or between $s_{i_l-1}$ and $s_{i_l}$ for $l = 1, ..., t$. Finally add the arcs $(r_m, s_0$ and $(s_k, r_0)$. Hence, the overall graph looks like:

\todo{Graphen von Valentina einfügen}

Thus, $s_k$ has only one outgoing arc $(s_k, r_0)$, $s_0$ has only one incoming arc $(r_m, s_0)$ and from $s_i$ to $s_j$ we have to use 3 vertices from $T$ corresponding to a subset $C_j$ with $x_{i+1}\in \mathcal C_j$. In such cases $(r_{j-1}, r_j)$ hasve to be used on the path from $r_0$ to $r_m$. Since only one subset of $T$ can be used from $s_i$ to $s_{i+1}$, all other sets $C_k$ with $x_{i+1} \in C_k$ have to be visited on the path from $r_{k-1}$ to $r_k$ implying that all $x \in C_k$ are visited this way.

Now, on the one hand, if there exists a subcollection $\plainset{C_{j1},...,C_{jk}}$ which partitions $ \mathcal X$, then the path between $S_{i-1}$, and $s_i$ choose $(r_{j-1}· r_j)$ as part of the Hamilton cycle, and for all $C_j \not\in B$ follow the path via the orange vertices from $r_{j-1}$ to $r_j$. Then, all vertices are visited exactly once, hence the digraph is Hamiltonian.

On the other and, if there exists a Hamilton cycle $D$, $B$ can be constructed by selecting $C_j$ if $(r_{j-1},rj)$ is part of the cycle and otherwise not. In summary there exists an EXACT COVER if and only if $D$ is Hamiltonian. And the reduction is polynomial.

\end{proof}

%20.2
\begin{thm}
UNDIRECTED HC is $\npoly$-complete.
\end{thm}

\begin{proof}
	We reduce DHC to UHD. Let $D$ be a digraph. Replace every vertex $v$ by three vertices $v'$, $v'$, and $v'''$. Construct edges $\plainset{v',v''}$ and $\plainset{v'', v'''}$.	Further, replace $(v,w)$ by $\plainset{v',v'''}$.
	
	$G$ is Hamiltonian if and only if $D$ is Hamiltonian can be easily verified.
\end{proof}

\tocsubsection{TSP}

INPUT: A complete graph $G = (V,E)$, a length function $l$ on $E$ and a rational $r$.

FIND: Does there exists a Hamiltonian cycle with length $\leq r$?

\begin{thm}
	TSP is $\npoly$-complete.
\end{thm}

\begin{proof}
	Let $G = (V,E)$ be an instance of UHC. Define $G'$ as complete graph on vertex set $V$. Let $l(e):=0$ for all $e\in E$ and $l(e):=1$ for edges $e\not \in E$. Then $G$ is Hamiltonian, if and only if $G'$ has a Hamilton cycle of length $\leq 0$.
\end{proof}

