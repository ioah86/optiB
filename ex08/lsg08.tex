\documentclass[a4paper]{article}
\usepackage[utf8]{inputenc}
\usepackage{fancyhdr}
\usepackage{amsmath}
\usepackage[ngerman]{babel}
\usepackage{amsthm}
\usepackage{amssymb}
\usepackage{tikz}
\usepackage{listings}
%\usepackage{fullpage}

\lstset{numbers=left, basicstyle=\ttfamily, numberstyle=\tiny, mathescape=true} %listing style für code

%\usetikzlibrary{positioning, trees, snakes}
%\usetikzlibrary{automata, positioning, arrows, calc}

\setlength{\parindent}{0pt}
\setlength{\parskip}{1ex}
%\setlength{\headheight}{30pt}
\addtolength{\textwidth}{2in}
\addtolength{\textheight}{1.5in}
\addtolength{\hoffset}{-1in}
\addtolength{\voffset}{-0.75in}
\pagestyle{fancy}

% Kopfzeile
\lhead{Optimierung B}
\chead{Übung 8}
\rhead{Niklas Fischer 298418 \\ Gereon Kremer 288911}

% Fußzeile
\lfoot{}
\cfoot{Seite \thepage{}}
\rfoot{}

\renewcommand{\thesection}{}
\renewcommand{\thesubsection}{(\alph{subsection})}

\begin{document}

\section{Aufgabe 1}

\section{Aufgabe 2}

\section{Aufgabe 3}

\section{Aufgabe 4}

\subsection*{}
Konstruiere einen Graphen $G = (V, E)$, für den ein Min-Cost-Flow das
Problem löst. Sei $B$ die Menge der Bahnhöfe und $Z$ die Menge der (der
offensichtlichen, natürlichen Ordnung unterliegenden) Uhrzeiten
eines Tages. Sei nun $X$ die Menge der Zughalte, also
$X \subseteq B \times Z$ und $F \subseteq X \times X$ die Menge der Fahrten
mit minimaler Anzahl Waggons $W: F \rightarrow \mathbb{N}$. Aus der
Anwendung ergibt sich, dass für jede Fahrt $(x_1, x_2)$ die Uhrzeit von
$x_1$ kleiner als die von $x_2$ sein sollte. Dies ist aber in der
Modellierung nicht nötig -- unser Verfahren ist also auch für Züge mit
Überlichtgeschwindigkeitsantrieben gerüstet :-).

Als ausgezeichnete Mengen verwenden wir ausserdem $X_-$ und $X_+$ als Mengen
der zeitlich ersten und letzten Halte an jedem Bahnhof.
\begin{align*}
X_- &= \{ (b, z) \in X \mid \not\exists (b, z_2) \in X: z_2 < z \} \\
X_+ &= \{ (b, z) \in X \mid \not\exists (b, z_2) \in X: z_2 > z \} \\
\end{align*}
Ausserdem definieren wir $next$ als die Abbildung eines Halts auf den
nächsten Halt am selben Bahnhof. Formal:
\[
	next(b, z) = (b, z_2) \text{ so dass } z_2 > z 
		\wedge \not\exists z': (b, z') \in X \wedge z < z' < z_2
\] 

Nun konstruieren wir den Graphen $G$ sowie die zu den Kanten gehörenden
Kapazitäten und Kosten $c, w: E \rightarrow \mathbb{N}$ und den Wert des zu suchenden
Flusses $x$.
\begin{align*}
V = & \{s, t\} \cup X \\
E = & \{ (s, t) \} \\
	& \cup \{ (s, (b, z)) \mid (b, z) \in X_- \} \\
	& \cup \{ ((b, z), t) \mid (b, z) \in X_+ \} \\
	& \cup \{ ((b, z), next(b, z)) \mid (b, z) \in X \} \\
	& \cup F_1 = \{ ((b_1, z_1), (b_2, z_2)) \mid ((b_1, z_1), (b_2, z_2)) \in F \} \\
	& \cup F_2 = \{ ((b_1, z_1), (b_2, z_2)) \mid ((b_1, z_1), (b_2, z_2)) \in F \}
\end{align*}

$F_1$ und $F_2$ bilden hier parallele Kanten. Diese können durch Einfügen
eines Zwischenknotens in alle Kanten aus $F_2$ in normale Kanten überführt
werden. Aus Gründen der Übersichtlichkeit werden wir dies hier nicht tun.

Als $x$ wählen wir $\sum_{f \in F} W(f)$, also die Summe aller Waggons, die
in allen Fahrten mindestens bewegt werden sollen.

Für $c(e)$ gilt:
\begin{align*}
c(e) &= W(e) & e \in F_1 \\
c(e) &= x & \text{ sonst}
\end{align*}
Für $w(e)$ gilt:
\begin{align*}
w(e) &= 1 & e \in \{ (s, (b, z)) \mid (b, z) \in X_- \} \\
w(e) &= -x & e \in F_1 \\
w(e) &= 0 & \text{ sonst}
\end{align*}

\end{document}
