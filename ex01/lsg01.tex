\documentclass[a4paper]{article}
\usepackage[utf8]{inputenc}
\usepackage{fancyhdr}
\pagestyle{fancy}

% Kopfzeile
\lhead{Optimierung B}
\chead{Übung 1}
\rhead{Gereon Kremer 288911}

% Fußzeile
\lfoot{}
\cfoot{Seite \thepage{}}
\rfoot{}

\renewcommand{\thesection}{}
\renewcommand{\thesubsection}{(\alph{subsection})}

\begin{document}

\section{Aufgabe 1}

\subsection{}
Gesucht ist ein Kreis $K = v_0 e_1 v_1 ... e_n v_n$ im ungerichteten Graphen $G
= (V, E)$, so dass $E = \{ e_1, ..., e_n \}$, wobei das Gewicht
$g = min \sum\limits_{i=1}^{n} \delta(e_n)$ minimal ist.

\subsection{}
In einem eulerschen Graphen existiert ein Eulerkreis. Der Postbote muss dann
keinen Weg doppelt laufen und es gilt $g = \sum\limits_{e \in E} \delta(e)$.

\subsection{}
\begin{enumerate}
\end{enumerate}


\end{document}
