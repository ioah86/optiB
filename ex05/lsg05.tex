\documentclass[a4paper]{article}
\usepackage[utf8]{inputenc}
\usepackage{fancyhdr}
\usepackage{amsmath}
\usepackage[ngerman]{babel}
\usepackage{amsthm}
\usepackage{tikz}
\usepackage{listings}
%\usepackage{fullpage}

\lstset{numbers=left, basicstyle=\ttfamily, numberstyle=\tiny, mathescape=true} %listing style für code

\usetikzlibrary{positioning, trees, snakes}
\usetikzlibrary{automata, positioning, arrows, calc}

\setlength{\parindent}{0pt}
\setlength{\parskip}{1ex}
%\setlength{\headheight}{30pt}
\addtolength{\textwidth}{2in}
\addtolength{\textheight}{1.5in}
\addtolength{\hoffset}{-1in}
\addtolength{\voffset}{-0.75in}
\pagestyle{fancy}

% Kopfzeile
\lhead{Optimierung B}
\chead{Übung 5}
\rhead{Niklas Fischer 298418 \\ Gereon Kremer 288911}

% Fußzeile
\lfoot{}
\cfoot{Seite \thepage{}}
\rfoot{}

\renewcommand{\thesection}{}
\renewcommand{\thesubsection}{(\alph{subsection})}

\begin{document}

\section{Aufgabe 1}

Der Graph zeigt jeweils den Zustand \emph{vor} der rechts angegebenen
Operation.

\begin{tikzpicture}[node distance=1cm]
	\node (n1) {$1$};
	\node (n4) [right of=n1] {$4$};
	\node (n5) [right of=n4] {$5$};
	\node (n2) [above right=1.5cm and 1cm of n5] {$2$};
	\node (n3) [below right=1.5cm and 1cm of n5] {$3$};
	\node (n6) [below right=0.8cm and 1cm of n2] {$6$};
	\node (n7) [above right=0.8cm and 1cm of n3] {$7$};
	\node (n8) [below right=0.3cm and 1cm of n6] {$8$};
	\node (n9) [right=2cm of n2] {$9$};
	\node (n11) [right=2cm of n3] {$11$};
	\node (n13) [right of=n8] {$13$};
	\node (n10) [right=2cm of n9] {$10$};
	\node (n12) [right=2cm of n11] {$12$};
	\path
		(n1) edge (n2) (n1) edge (n3)
		(n2) edge (n3) (n2) edge (n4) (n2) edge (n5) (n2) edge (n6) (n2) edge (n9)
		(n3) edge (n4) (n3) edge (n5) (n3) edge (n7) (n3) edge (n11)
		(n4) edge (n5)
		(n6) edge (n7) (n6) edge (n8)
		(n7) edge (n8)
		(n9) edge (n10) (n9) edge (n13)
		(n10) edge (n13)
		(n11) edge (n12) (n11) edge (n13)
		(n12) edge (n13)
	;
\end{tikzpicture}
\parbox[b]{0.45\textwidth}{
$M = \emptyset$ \\
$W = V$.

Wähle als kürzesten $M$-alternierenden $W-W$-Weg \\
$P = 1, 2$.

$P$ ist ein Pfad.

$M = \{ (1, 2) \}$. \\
}

\begin{tikzpicture}[node distance=1cm]
	\node (n1) {$1$};
	\node (n4) [right of=n1] {$4$};
	\node (n5) [right of=n4] {$5$};
	\node (n2) [above right=1.5cm and 1cm of n5] {$2$};
	\node (n3) [below right=1.5cm and 1cm of n5] {$3$};
	\node (n6) [below right=0.8cm and 1cm of n2] {$6$};
	\node (n7) [above right=0.8cm and 1cm of n3] {$7$};
	\node (n8) [below right=0.3cm and 1cm of n6] {$8$};
	\node (n9) [right=2cm of n2] {$9$};
	\node (n11) [right=2cm of n3] {$11$};
	\node (n13) [right of=n8] {$13$};
	\node (n10) [right=2cm of n9] {$10$};
	\node (n12) [right=2cm of n11] {$12$};
	\path
		(n1) edge [dashed] (n2) (n1) edge (n3)
		(n2) edge (n3) (n2) edge (n4) (n2) edge (n5) (n2) edge (n6) (n2) edge (n9)
		(n3) edge (n4) (n3) edge (n5) (n3) edge (n7) (n3) edge (n11)
		(n4) edge (n5)
		(n6) edge (n7) (n6) edge (n8)
		(n7) edge (n8)
		(n9) edge (n10) (n9) edge (n13)
		(n10) edge (n13)
		(n11) edge (n12) (n11) edge (n13)
		(n12) edge (n13)
	;
\end{tikzpicture}
\parbox[b]{0.45\textwidth}{
$M = \{ (1, 2) \}$ \\
$W = V \setminus \{1, 2\}$.

Wähle als kürzesten $M$-alternierenden $W-W$-Weg \\
$P = 3, 4$.

$P$ ist ein Pfad.

$M = \{ (1, 2), (3, 4) \}$. \\
}

\begin{tikzpicture}[node distance=1cm]
	\node (n1) {$1$};
	\node (n4) [right of=n1] {$4$};
	\node (n5) [right of=n4] {$5$};
	\node (n2) [above right=1.5cm and 1cm of n5] {$2$};
	\node (n3) [below right=1.5cm and 1cm of n5] {$3$};
	\node (n6) [below right=0.8cm and 1cm of n2] {$6$};
	\node (n7) [above right=0.8cm and 1cm of n3] {$7$};
	\node (n8) [below right=0.3cm and 1cm of n6] {$8$};
	\node (n9) [right=2cm of n2] {$9$};
	\node (n11) [right=2cm of n3] {$11$};
	\node (n13) [right of=n8] {$13$};
	\node (n10) [right=2cm of n9] {$10$};
	\node (n12) [right=2cm of n11] {$12$};
	\path
		(n1) edge [dashed] (n2) (n1) edge (n3)
		(n2) edge (n3) (n2) edge (n4) (n2) edge (n5) (n2) edge (n6) (n2) edge (n9)
		(n3) edge [dashed] (n4) (n3) edge (n5) (n3) edge (n7) (n3) edge (n11)
		(n4) edge (n5)
		(n6) edge (n7) (n6) edge (n8)
		(n7) edge (n8)
		(n9) edge (n10) (n9) edge (n13)
		(n10) edge (n13)
		(n11) edge (n12) (n11) edge (n13)
		(n12) edge (n13)
	;
\end{tikzpicture}
\parbox[b]{0.45\textwidth}{
$M = \{ (1, 2), (3, 4) \}$ \\
$W = V \setminus \{1, 2, 3, 4\}$.

Wähle als kürzesten $M$-alternierenden $W-W$-Weg \\
$P = 6, 7$.

$P$ ist ein Pfad.

$M = \{ (1, 2), (3, 4), (6, 7) \}$. \\
}

\begin{tikzpicture}[node distance=1cm]
	\node (n1) {$1$};
	\node (n4) [right of=n1] {$4$};
	\node (n5) [right of=n4] {$5$};
	\node (n2) [above right=1.5cm and 1cm of n5] {$2$};
	\node (n3) [below right=1.5cm and 1cm of n5] {$3$};
	\node (n6) [below right=0.8cm and 1cm of n2] {$6$};
	\node (n7) [above right=0.8cm and 1cm of n3] {$7$};
	\node (n8) [below right=0.3cm and 1cm of n6] {$8$};
	\node (n9) [right=2cm of n2] {$9$};
	\node (n11) [right=2cm of n3] {$11$};
	\node (n13) [right of=n8] {$13$};
	\node (n10) [right=2cm of n9] {$10$};
	\node (n12) [right=2cm of n11] {$12$};
	\path
		(n1) edge [dashed] (n2) (n1) edge (n3)
		(n2) edge (n3) (n2) edge (n4) (n2) edge (n5) (n2) edge (n6) (n2) edge (n9)
		(n3) edge [dashed] (n4) (n3) edge (n5) (n3) edge (n7) (n3) edge (n11)
		(n4) edge (n5)
		(n6) edge [dashed] (n7) (n6) edge (n8)
		(n7) edge (n8)
		(n9) edge (n10) (n9) edge (n13)
		(n10) edge (n13)
		(n11) edge (n12) (n11) edge (n13)
		(n12) edge (n13)
	;
\end{tikzpicture}
\parbox[b]{0.45\textwidth}{
$M = \{ (1, 2), (3, 4), (6, 7) \}$ \\
$W = V \setminus \{1, 2, 3, 4, 6, 7\}$.

Wähle als kürzesten $M$-alternierenden $W-W$-Weg \\
$P = 9, 10$.

$P$ ist ein Pfad.

$M = \{ (1, 2), (3, 4), (6, 7), (9, 10) \}$. \\
}

\begin{tikzpicture}[node distance=1cm]
	\node (n1) {$1$};
	\node (n4) [right of=n1] {$4$};
	\node (n5) [right of=n4] {$5$};
	\node (n2) [above right=1.5cm and 1cm of n5] {$2$};
	\node (n3) [below right=1.5cm and 1cm of n5] {$3$};
	\node (n6) [below right=0.8cm and 1cm of n2] {$6$};
	\node (n7) [above right=0.8cm and 1cm of n3] {$7$};
	\node (n8) [below right=0.3cm and 1cm of n6] {$8$};
	\node (n9) [right=2cm of n2] {$9$};
	\node (n11) [right=2cm of n3] {$11$};
	\node (n13) [right of=n8] {$13$};
	\node (n10) [right=2cm of n9] {$10$};
	\node (n12) [right=2cm of n11] {$12$};
	\path
		(n1) edge [dashed] (n2) (n1) edge (n3)
		(n2) edge (n3) (n2) edge (n4) (n2) edge (n5) (n2) edge (n6) (n2) edge (n9)
		(n3) edge [dashed] (n4) (n3) edge (n5) (n3) edge (n7) (n3) edge (n11)
		(n4) edge (n5)
		(n6) edge [dashed] (n7) (n6) edge (n8)
		(n7) edge (n8)
		(n9) edge [dashed] (n10) (n9) edge (n13)
		(n10) edge (n13)
		(n11) edge (n12) (n11) edge (n13)
		(n12) edge (n13)
	;
\end{tikzpicture}
\parbox[b]{0.45\textwidth}{
$M = \{ (1, 2), (3, 4), (6, 7), (9, 10) \}$ \\
$W = V \setminus \{1, 2, 3, 4, 6, 7, 9, 10\}$.

Wähle als kürzesten $M$-alternierenden $W-W$-Weg \\
$P = 11, 12$.

$P$ ist ein Pfad.

$M = \{ (1, 2), (3, 4), (6, 7), (9, 10), (11, 12) \}$. \\
}

\begin{tikzpicture}[node distance=1cm]
	\node (n1) {$1$};
	\node (n4) [right of=n1] {$4$};
	\node (n5) [right of=n4] {$5$};
	\node (n2) [above right=1.5cm and 1cm of n5] {$2$};
	\node (n3) [below right=1.5cm and 1cm of n5] {$3$};
	\node (n6) [below right=0.8cm and 1cm of n2] {$6$};
	\node (n7) [above right=0.8cm and 1cm of n3] {$7$};
	\node (n8) [below right=0.3cm and 1cm of n6] {$8$};
	\node (n9) [right=2cm of n2] {$9$};
	\node (n11) [right=2cm of n3] {$11$};
	\node (n13) [right of=n8] {$13$};
	\node (n10) [right=2cm of n9] {$10$};
	\node (n12) [right=2cm of n11] {$12$};
	\path
		(n1) edge [dashed] (n2) (n1) edge (n3)
		(n2) edge (n3) (n2) edge (n4) (n2) edge (n5) (n2) edge (n6) (n2) edge (n9)
		(n3) edge [dashed] (n4) (n3) edge (n5) (n3) edge (n7) (n3) edge (n11)
		(n4) edge (n5)
		(n6) edge [dashed] (n7) (n6) edge (n8)
		(n7) edge (n8)
		(n9) edge [dashed] (n10) (n9) edge (n13)
		(n10) edge (n13)
		(n11) edge [dashed] (n12) (n11) edge (n13)
		(n12) edge (n13)
	;
\end{tikzpicture}
\parbox[b]{0.45\textwidth}{
$M = \{ (1, 2), (3, 4), (6, 7), (9, 10), (11, 12) \}$ \\
$W = V \setminus \{1, 2, 3, 4, 6,
7, 9, 10, 11, 12\}$.

Wähle als kürzesten $M$-alternierenden $W-W$-Weg \\
$P = 5, 3, 4, 5$.

$P$ ist kein Pfad. Fasse Blossom zu einem Knoten zusammen und führe
Algorithmus auf $G / P$ aus. \\
}

\begin{tikzpicture}[node distance=1cm]
	\node (n1) {$1$};
	\node (n4) [right of=n1] {};
	\node (n5) [right of=n4] {};
	\node (n2) [above right=1.5cm and 1cm of n5] {$2$};
	\node (n3) [below right=1.5cm and 1cm of n5] {$B$};
	\node (n6) [below right=0.8cm and 1cm of n2] {$6$};
	\node (n7) [above right=0.8cm and 1cm of n3] {$7$};
	\node (n8) [below right=0.3cm and 1cm of n6] {$8$};
	\node (n9) [right=2cm of n2] {$9$};
	\node (n11) [right=2cm of n3] {$11$};
	\node (n13) [right of=n8] {$13$};
	\node (n10) [right=2cm of n9] {$10$};
	\node (n12) [right=2cm of n11] {$12$};
	\path
		(n1) edge [dashed] (n2) (n1) edge (n3)
		(n2) edge (n3) (n2) edge (n6) (n2) edge (n9)
		(n3) edge (n7) (n3) edge (n11)
		(n6) edge [dashed] (n7) (n6) edge (n8)
		(n7) edge (n8)
		(n9) edge [dashed] (n10) (n9) edge (n13)
		(n10) edge (n13)
		(n11) edge [dashed] (n12) (n11) edge (n13)
		(n12) edge (n13)
	;
\end{tikzpicture}
\parbox[b]{0.45\textwidth}{
$M = \{ (1, 2), (6, 7), (9, 10), (11, 12) \}$ \\
$W = V \setminus \{1, 2, 6, 7, 9, 10, 11, 12\}$.

Wähle als kürzesten $M$-alternierenden $W-W$-Weg \\
$P = B, 7, 6, 8$.

$P$ ist ein Pfad.

$M = \{ (1,2), (B,7), (6,8), (9,10), (11,12) \}$ \\
}

\begin{tikzpicture}[node distance=1cm]
	\node (n1) {$1$};
	\node (n4) [right of=n1] {$4$};
	\node (n5) [right of=n4] {$5$};
	\node (n2) [above right=1.5cm and 1cm of n5] {$2$};
	\node (n3) [below right=1.5cm and 1cm of n5] {$3$};
	\node (n6) [below right=0.8cm and 1cm of n2] {$6$};
	\node (n7) [above right=0.8cm and 1cm of n3] {$7$};
	\node (n8) [below right=0.3cm and 1cm of n6] {$8$};
	\node (n9) [right=2cm of n2] {$9$};
	\node (n11) [right=2cm of n3] {$11$};
	\node (n13) [right of=n8] {$13$};
	\node (n10) [right=2cm of n9] {$10$};
	\node (n12) [right=2cm of n11] {$12$};
	\path
		(n1) edge [dashed] (n2) (n1) edge (n3)
		(n2) edge (n3) (n2) edge (n4) (n2) edge (n5) (n2) edge (n6) (n2) edge (n9)
		(n3) edge [dashed] (n4) (n3) edge (n5) (n3) edge (n7) (n3) edge (n11)
		(n4) edge (n5)
		(n6) edge [dashed] (n7) (n6) edge (n8)
		(n7) edge (n8)
		(n9) edge [dashed] (n10) (n9) edge (n13)
		(n10) edge (n13)
		(n11) edge [dashed] (n12) (n11) edge (n13)
		(n12) edge (n13)
	;
\end{tikzpicture}
\parbox[b]{0.45\textwidth}{
Ergebnis von rekursivem Aufruf:

$M_{rek} = \{ (1,2), (B,7), (6,8), (9,10), (11,12) \}$

Wandle $M_{rek}$ in Matching in $G$ um:

$M = \{ (1, 2), (3, 7), (4,5) (6, 8), (9, 10), (11, 12) \}$ \\
}

\begin{tikzpicture}[node distance=1cm]
	\node (n1) {$1$};
	\node (n4) [right of=n1] {$4$};
	\node (n5) [right of=n4] {$5$};
	\node (n2) [above right=1.5cm and 1cm of n5] {$2$};
	\node (n3) [below right=1.5cm and 1cm of n5] {$3$};
	\node (n6) [below right=0.8cm and 1cm of n2] {$6$};
	\node (n7) [above right=0.8cm and 1cm of n3] {$7$};
	\node (n8) [below right=0.3cm and 1cm of n6] {$8$};
	\node (n9) [right=2cm of n2] {$9$};
	\node (n11) [right=2cm of n3] {$11$};
	\node (n13) [right of=n8] {$13$};
	\node (n10) [right=2cm of n9] {$10$};
	\node (n12) [right=2cm of n11] {$12$};
	\path
		(n1) edge [dashed] (n2) (n1) edge (n3)
		(n2) edge (n3) (n2) edge (n4) (n2) edge (n5) (n2) edge (n6) (n2) edge (n9)
		(n3) edge (n4) (n3) edge (n5) (n3) edge [dashed] (n7) (n3) edge (n11)
		(n4) edge [dashed] (n5)
		(n6) edge (n7) (n6) edge [dashed] (n8)
		(n7) edge (n8)
		(n9) edge [dashed] (n10) (n9) edge (n13)
		(n10) edge (n13)
		(n11) edge [dashed] (n12) (n11) edge (n13)
		(n12) edge (n13)
	;
\end{tikzpicture}
\parbox[b]{0.45\textwidth}{
$M = \{ (1, 2), (3, 7), (4,5) (6, 8), (9, 10), (11, 12) \}$ \\
$W = \{13\}$

Wähle als kürzesten $M$-alternierenden $W-W$-Weg \\
$P = 13, 9, 10, 13$

$P$ ist kein Pfad. Fasse Blossom zu einem Knoten zusammen und führe
Algorithmus auf $G / P$ aus. \\
}

\begin{tikzpicture}[node distance=1cm]
	\node (n1) {$1$};
	\node (n4) [right of=n1] {$4$};
	\node (n5) [right of=n4] {$5$};
	\node (n2) [above right=1.5cm and 1cm of n5] {$2$};
	\node (n3) [below right=1.5cm and 1cm of n5] {$3$};
	\node (n6) [below right=0.8cm and 1cm of n2] {$6$};
	\node (n7) [above right=0.8cm and 1cm of n3] {$7$};
	\node (n8) [below right=0.3cm and 1cm of n6] {$8$};
	\node (n9) [right=4cm of n2] {$B$};
	\node (n11) [right=2cm of n3] {$11$};
	\node (n12) [right=2cm of n11] {$12$};
	\path
		(n1) edge [dashed] (n2) (n1) edge (n3)
		(n2) edge (n3) (n2) edge (n4) (n2) edge (n5) (n2) edge (n6) (n2) edge (n9)
		(n3) edge (n4) (n3) edge (n5) (n3) edge [dashed] (n7) (n3) edge (n11)
		(n4) edge [dashed] (n5)
		(n6) edge (n7) (n6) edge [dashed] (n8)
		(n7) edge (n8)
		(n11) edge [dashed] (n12) (n11) edge (n9)
		(n12) edge (n9)
	;
\end{tikzpicture}
\parbox[b]{0.45\textwidth}{
$M = \{ (1, 2), (3, 7), (4,5) (6, 8), (11, 12) \}$ \\
$W = \{B\}$

Wähle als kürzesten $M$-alternierenden $W-W$-Weg \\
$P = B, 11, 12, B$

$P$ ist kein Pfad. Fasse Blossom zu einem Knoten zusammen und führe
Algorithmus auf $G / P$ aus. \\
}

\begin{tikzpicture}[node distance=1cm]
	\node (n1) {$1$};
	\node (n4) [right of=n1] {$4$};
	\node (n5) [right of=n4] {$5$};
	\node (n2) [above right=1.5cm and 1cm of n5] {$2$};
	\node (n3) [below right=1.5cm and 1cm of n5] {$3$};
	\node (n6) [below right=0.8cm and 1cm of n2] {$6$};
	\node (n7) [above right=0.8cm and 1cm of n3] {$7$};
	\node (n8) [below right=0.3cm and 1cm of n6] {$8$};
	\node (n9) [right=of n8] {$B$};
	\path
		(n1) edge [dashed] (n2) (n1) edge (n3)
		(n2) edge (n3) (n2) edge (n4) (n2) edge (n5) (n2) edge (n6) (n2) edge (n9)
		(n3) edge (n4) (n3) edge (n5) (n3) edge [dashed] (n7) (n3) edge (n9)
		(n4) edge [dashed] (n5)
		(n6) edge (n7) (n6) edge [dashed] (n8)
		(n7) edge (n8)
	;
\end{tikzpicture}
\parbox[b]{0.45\textwidth}{
$M = \{ (1, 2), (3, 7), (4,5) (6, 8) \}$ \\
$W = \{B\}$

Wähle als kürzesten $M$-alternierenden $W-W$-Weg \\
$P = B, 3, 7, 6, 8, 7, 3, B$

$P$ ist kein Pfad. Tausche Matching-Kanten auf $P$ vor Blossom.

$M = \{ (1, 2), (3, B), (4,5) (6, 8) \}$ \\

Fasse Blossom zu einem Knoten zusammen und führe
Algorithmus auf $G / P$ aus. \\
}

\begin{tikzpicture}[node distance=1cm]
	\node (n1) {$1$};
	\node (n4) [right of=n1] {$4$};
	\node (n5) [right of=n4] {$5$};
	\node (n2) [above right=1.5cm and 1cm of n5] {$2$};
	\node (n3) [below right=1.5cm and 1cm of n5] {$3$};
	\node (n6) [below right=0.8cm and 1cm of n2] {};
	\node (n7) [above right=0.8cm and 1cm of n3] {};
	\node (n8) [below right=0.3cm and 0cm of n6] {$B_2$};
	\node (n9) [right=of n8] {$B$};
	\path
		(n1) edge [dashed] (n2) (n1) edge (n3)
		(n2) edge (n3) (n2) edge (n4) (n2) edge (n5) (n2) edge (n8) (n2) edge (n9)
		(n3) edge (n4) (n3) edge (n5) (n3) edge (n8) (n3) edge [dashed] (n9)
		(n4) edge [dashed] (n5)
	;
\end{tikzpicture}
\parbox[b]{0.45\textwidth}{
$M = \{ (1, 2), (3, 7), (4,5) \}$ \\
$W = \{B_2\}$

Es existiert kein $M$-alternierender $W-W$-Weg.

$\Rightarrow$ $M$ ist maximum Matching. \\
}

\begin{tikzpicture}[node distance=1cm]
	\node (n1) {$1$};
	\node (n4) [right of=n1] {$4$};
	\node (n5) [right of=n4] {$5$};
	\node (n2) [above right=1.5cm and 1cm of n5] {$2$};
	\node (n3) [below right=1.5cm and 1cm of n5] {$3$};
	\node (n6) [below right=0.8cm and 1cm of n2] {$6$};
	\node (n7) [above right=0.8cm and 1cm of n3] {$7$};
	\node (n8) [below right=0.3cm and 1cm of n6] {$8$};
	\node (n9) [right=of n8] {$B$};
	\path
		(n1) edge [dashed] (n2) (n1) edge (n3)
		(n2) edge (n3) (n2) edge (n4) (n2) edge (n5) (n2) edge (n6) (n2) edge (n9)
		(n3) edge (n4) (n3) edge (n5) (n3) edge (n7) (n3) edge [dashed] (n9)
		(n4) edge [dashed] (n5)
		(n6) edge (n7) (n6) edge [dashed] (n8)
		(n7) edge (n8)
	;
\end{tikzpicture}
\parbox[b]{0.45\textwidth}{
Ergebnis von rekursivem Aufruf: \\
$M_{rek} = \{ (1, 2), (3, B), (4,5) \}$

Wandle $M_{rek}$ in Matching in $G$ um:

$M = \{ (1, 2), (3, B), (4,5), (6, 8)\}$ \\
}

\begin{tikzpicture}[node distance=1cm]
	\node (n1) {$1$};
	\node (n4) [right of=n1] {$4$};
	\node (n5) [right of=n4] {$5$};
	\node (n2) [above right=1.5cm and 1cm of n5] {$2$};
	\node (n3) [below right=1.5cm and 1cm of n5] {$3$};
	\node (n6) [below right=0.8cm and 1cm of n2] {$6$};
	\node (n7) [above right=0.8cm and 1cm of n3] {$7$};
	\node (n8) [below right=0.3cm and 1cm of n6] {$8$};
	\node (n9) [right=4cm of n2] {$B$};
	\node (n11) [right=2cm of n3] {$11$};
	\node (n12) [right=2cm of n11] {$12$};
	\path
		(n1) edge [dashed] (n2) (n1) edge (n3)
		(n2) edge (n3) (n2) edge (n4) (n2) edge (n5) (n2) edge (n6) (n2) edge (n9)
		(n3) edge (n4) (n3) edge (n5) (n3) edge (n7) (n3) edge [dashed] (n11)
		(n4) edge [dashed] (n5)
		(n6) edge (n7) (n6) edge [dashed] (n8)
		(n7) edge (n8)
		(n11) edge (n12) (n11) edge (n9)
		(n12) edge [dashed] (n9)
	;
\end{tikzpicture}
\parbox[b]{0.45\textwidth}{
Ergebnis von rekursivem Aufruf: \\
$M_{rek} = \{ (1,2), (3,B), (4,5), (6,8) \}$

Wandle $M_{rek}$ in Matching in $G$ um:

$M = \{ (1, 2), (3, 11), (4,5), (6, 8), (12, B) \}$ \\
}

\begin{tikzpicture}[node distance=1cm]
	\node (n1) {$1$};
	\node (n4) [right of=n1] {$4$};
	\node (n5) [right of=n4] {$5$};
	\node (n2) [above right=1.5cm and 1cm of n5] {$2$};
	\node (n3) [below right=1.5cm and 1cm of n5] {$3$};
	\node (n6) [below right=0.8cm and 1cm of n2] {$6$};
	\node (n7) [above right=0.8cm and 1cm of n3] {$7$};
	\node (n8) [below right=0.3cm and 1cm of n6] {$8$};
	\node (n9) [right=2cm of n2] {$9$};
	\node (n11) [right=2cm of n3] {$11$};
	\node (n13) [right of=n8] {$13$};
	\node (n10) [right=2cm of n9] {$10$};
	\node (n12) [right=2cm of n11] {$12$};
	\path
		(n1) edge [dashed] (n2) (n1) edge (n3)
		(n2) edge (n3) (n2) edge (n4) (n2) edge (n5) (n2) edge (n6) (n2) edge (n9)
		(n3) edge (n4) (n3) edge (n5) (n3) edge (n7) (n3) edge [dashed] (n11)
		(n4) edge [dashed] (n5)
		(n6) edge (n7) (n6) edge [dashed] (n8)
		(n7) edge (n8)
		(n9) edge [dashed] (n10) (n9) edge (n13)
		(n10) edge (n13)
		(n11) edge (n12) (n11) edge (n13)
		(n12) edge [dashed] (n13)
	;
\end{tikzpicture}
\parbox[b]{0.45\textwidth}{
Ergebnis von rekursivem Aufruf: \\
$M_{rek} = \{ (1,2), (3,11), (4,5), (6,8), (12, B)\}$

Wandle $M_{rek}$ in Matching in $G$ um:

$M = \{ (1, 2), (3, 11), (4,5) (6, 8), (9, 10), (12, 13) \}$

Das maximale Matching ist $M$. \\
}


\section{Aufgabe 2}

\section{Aufgabe 3}
Der Satz von Tutte besagt für einen Graphen $G = (V, E)$:
\[
	\exists M \text{ perfektes Matching in } G \Leftrightarrow
	\forall S \subseteq V: |S| \geq odd(G - S)
\]

Sei $G = (V, E)$ ein regulärer Graph ohne Brücken.
Betrachte eine beliebige Knotenteilmenge $S \subseteq V$.
Wir werden zeigen, dass $|S| \geq odd(G - S)$. Dann hat $G$ ein perfektes
Matching.
Betrachte nun alle Zusammenhangskomponenten von $G - S$ mit ungeradem Grad.
Wir nennen diese $C_i, i = 1 ... k$.

Da $G$ 3-regulär, muss die Anzahl der ausgehenden Kanten von jedem $C_i$
nach dem Handschlaglemma ungerade sein. Da ausserdem keine Brücken
existieren, muss die Anzahl größer als $1$ sein, also mindestens $3$.

Die Anzahl der von $S$ ausgehenden Kanten kann gleichzeitig maximal
$3 \cdot |S|$ sein, da $G$ 3-regulär.
Die Anzahl der $C_i$ ist also durch $3 \cdot |S| / 3$ beschränkt, also
$odd(G - S) \leq |S|$.
\qed



\section{Aufgabe 4}

\section{Aufgabe 5}
Sei $G = (V, E)$ bipartit mit Farbklassen $U, W$ sowie $M_1, M_2 \subseteq E$ Matchings in $G$
gegeben. Konstruiere nun ein Matching $M \subseteq M_1 \cup M_2$ so dass $M$
alle Knoten aus $U$ überdeckt, die auch von $M_1$ überdeckt werden, und alle
Knoten aus $W$ überdeckt, die auch von $M_2$ überdeckt werden.

Betrachten wir zunächst $M' = M_1 \cup M_2$. Wir suchen nun eine Auswahl von
Kanten $M \subseteq M'$, so das $M$ die obigen Forderungen erfüllt.
Wir wissen, da $M_1, M_2$
Matchings, dass $M'$ nur aus zwei Typen von Zusammenhangskomponenten
bestehen kann: aus Kreisen und aus Pfaden. Diese Komponenten können
unabhängig voneinander betrachtet werden.

Im Falle eines Kreises $K$ wählen wir oBdA $M = M_1 \cap K$. Da in einem
Kreis stets abwechselnd Kanten aus $M_1$ und $M_2$ vorkommen, ist $M$ ein
Matching, falls wir alle Kanten aus $M_1$ wählen. Zudem sind damit alle
Knoten des Kreises abgedeckt, insbesondere auch die von $M_1$ in $U$ und von
$M_2$ in $W$ abgedeckten.

Im Falle eines Pfades $P$ kommen auch hier abwechselnd Kanten aus $M_1$ und
$M_2$ vor. Wir müssen zwischen Pfaden gerader und ungerader Länge unterscheiden.
Bei Pfaden ungerader Länge können wir die Kanten in $M$ so wählen, das alle
Knoten auf dem Pfade überdeckt werden. Hier gilt wiederum $M = P \cap M_1$
oder $M = P \cap M_2$, je nachdem ob die erste Kante des Pfades zu $M_1$
oder $M_2$ gehört.

Hat der Pfad jedoch gerade Länge, so können nicht alle Knoten überdeckt
werden, genauer können alle bis auf einen der Endpunkte überdeckt werden.
Zunächst beobachten wir, dass die erste Kante zu $M_1$ und die
letzte zu $M_2$ gehört (oder umgekehrt, oBdA betrachten wir diesen Fall).
Wir müssen nun entscheiden, welchen der beiden Endpunkte wir nicht
überdecken.

Da der Graph bipartit ist, muss ein Graph gerader Länge in der selben
Farbklasse beginnen und enden. Also seien beide Endpunkte $v_1, v_2 \in U$
(wiederum oBdA, für $v_1, v_2 \in W$ gilt die Aussage analog).
Dann wählen wir $M = P \cap M_1$, da so genau der Endpunkt herausfällt, der
nicht von $M_1$ sondern von $M_2$ überdeckt ist. Da der Endpunkt aber in
$U$, muss er von $M$ auch nicht überdeckt werden.

So können wir für jede Zusammenhangskomponente einzeln ein geeignetes
Matching erzeugen, das gesamte Matching $M$ erhalten wir dann durch
Vereinigung aller Matchings auf den Komponenten. $\qed$

\section{Aufgabe 6}
\subsection{}
Die Tutte-Berge-Formel besagt, dass die Größe eines Maximum Matchings
\[
	\frac{1}{2} \min_{U \subseteq V} ( |U| - odd(G - U) + |V|)
\]
entspricht. Da kein perfektes Matching existieren soll, darf die folgende
Gleichheit nicht erfüllt sein:
\begin{align*}
|V| &= \min_{U \subseteq V} ( |U| - odd(G - U) + |V|) \\
	&= \min_{U \subseteq V} ( |U| - odd(G - U) ) + |V| \\
\Leftrightarrow 0 &= \min_{U \subseteq V} ( |U| - odd(G - U) )
\end{align*}
Es darf also kein $U \subseteq V$ existieren, so dass $|U| = odd(G - U)$.


\subsection{}

\tikzstyle{level 1}=[sibling angle=120]
\tikzstyle{level 2}=[sibling angle=45]
\begin{figure}[h]\caption{Beispielgraph für $k=3$}
\begin{center}
\begin{tikzpicture}[grow cyclic, cap=round]
\node (vc) {$v_c$}
	child foreach \i in {1,2,3} {
		node (v\i-1) {$v_{\i,i}$}
		child foreach \k in {1,2} {
			node (v\i\k) {$v_{\i,\k}$}
			child { node (v\i-t\k) {$v_{\i,t_\k}$} }
		}
	}
;
\foreach \i in {1,2,3}
	\draw (v\i1) -- (v\i-t2) (v\i2) -- (v\i-t1) (v\i-t1) -- (v\i-t2);
\end{tikzpicture}
\end{center}
\end{figure}

\tikzstyle{level 1}=[sibling angle=72]
\tikzstyle{level 2}=[sibling angle=30]
\begin{figure}[h]\caption{Beispielgraph für $k=5$}
\begin{center}
\begin{tikzpicture}[grow cyclic, cap=round]
\node (vc) {$v_c$}
	child foreach \i in {1,2,3,4,5} {
		node (v\i-i) {$v_{\i,i}$} 
 		child [level distance=2cm] { node (v\i-1) {$v_{\i,1}$} }
		child { node (v\i-2) {$v_{\i,2}$}
			child { node (v\i-t1) {$v_{\i,t_1}$} }
		}
		child { node (v\i-3) {$v_{\i,3}$}
			child { node (v\i-t2) {$v_{\i,t_2}$} }
		}
		child [level distance=2cm] { node (v\i-4) {$v_{\i,4}$} }
	}
;
\foreach \i in {1,2,3,4,5}
	\draw
		(v\i-1) -- (v\i-2) (v\i-2) -- (v\i-3)
		(v\i-3) -- (v\i-4) (v\i-1) edge [bend left] (v\i-4)
		(v\i-1) -- (v\i-t1) (v\i-1) -- (v\i-t2)
		(v\i-2) -- (v\i-t2)	(v\i-3) -- (v\i-t1)
		(v\i-4) -- (v\i-t1) (v\i-4) -- (v\i-t2)
		(v\i-t1) -- (v\i-t2)
;
\end{tikzpicture}
\end{center}
\end{figure}

\end{document}
