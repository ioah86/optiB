\documentclass[a4paper]{article}
\usepackage[utf8]{inputenc}
\usepackage{fancyhdr}
\usepackage{amsmath}
\usepackage[ngerman]{babel}
\usepackage{amsthm}
\usepackage{tikz}
\usepackage{listings}
%\usepackage{fullpage}

\lstset{numbers=left, basicstyle=\ttfamily, numberstyle=\tiny, mathescape=true} %listing style für code

\usetikzlibrary{positioning, trees, snakes}
\usetikzlibrary{automata, positioning, arrows, calc}

\setlength{\parindent}{0pt}
\setlength{\parskip}{1ex}
%\setlength{\headheight}{30pt}
\addtolength{\textwidth}{2in}
\addtolength{\textheight}{1.5in}
\addtolength{\hoffset}{-1in}
\addtolength{\voffset}{-0.75in}
\pagestyle{fancy}

% Kopfzeile
\lhead{Optimierung B}
\chead{Übung 5}
\rhead{Niklas Fischer 298418 \\ Gereon Kremer 288911}

% Fußzeile
\lfoot{}
\cfoot{Seite \thepage{}}
\rfoot{}

\renewcommand{\thesection}{}
\renewcommand{\thesubsection}{(\alph{subsection})}

\begin{document}

\section{Aufgabe 1}
\section{Aufgabe 2}
\section{Aufgabe 3}
\section{Aufgabe 4}
\section{Aufgabe 5}
\section{Aufgabe 6}
\subsection{}
Die Tutte-Berge-Formel besagt, dass die Größe eines Maximum Matchings
\[
	\frac{1}{2} \min_{U \subseteq V} ( |U| - odd(G - U) + |V|)
\]
entspricht. Da kein perfektes Matching existieren soll, darf die folgende
Gleichheit nicht erfüllt sein:
\begin{align*}
|V| &= \min_{U \subseteq V} ( |U| - odd(G - U) + |V|) \\
	&= \min_{U \subseteq V} ( |U| - odd(G - U) ) + |V| \\
\Leftrightarrow 0 &= \min_{U \subseteq V} ( |U| - odd(G - U) )
\end{align*}
Es darf also kein $U \subseteq V$ existieren, so dass $|U| = odd(G - U)$.


\subsection{}

\tikzstyle{level 1}=[sibling angle=120]
\tikzstyle{level 2}=[sibling angle=45]
\begin{figure}[h]\caption{Beispielgraph für $k=3$}
\begin{center}
\begin{tikzpicture}[grow cyclic, cap=round]
\node (vc) {$v_c$}
	child foreach \i in {1,2,3} {
		node (v\i-1) {$v_{\i,i}$}
		child foreach \k in {1,2} {
			node (v\i\k) {$v_{\i,\k}$}
			child { node (v\i-t\k) {$v_{\i,t_\k}$} }
		}
	}
;
\foreach \i in {1,2,3}
	\draw (v\i1) -- (v\i-t2) (v\i2) -- (v\i-t1) (v\i-t1) -- (v\i-t2);
\end{tikzpicture}
\end{center}
\end{figure}

\tikzstyle{level 1}=[sibling angle=72]
\tikzstyle{level 2}=[sibling angle=30]
\begin{figure}[h]\caption{Beispielgraph für $k=5$}
\begin{center}
\begin{tikzpicture}[grow cyclic, cap=round]
\node (vc) {$v_c$}
	child foreach \i in {1,2,3,4,5} {
		node (v\i-i) {$v_{\i,i}$} 
 		child [level distance=2cm] { node (v\i-1) {$v_{\i,1}$} }
		child { node (v\i-2) {$v_{\i,2}$}
			child { node (v\i-t1) {$v_{\i,t_1}$} }
		}
		child { node (v\i-3) {$v_{\i,3}$}
			child { node (v\i-t2) {$v_{\i,t_2}$} }
		}
		child [level distance=2cm] { node (v\i-4) {$v_{\i,4}$} }
	}
;
\foreach \i in {1,2,3,4,5}
	\draw
		(v\i-1) -- (v\i-2) (v\i-2) -- (v\i-3)
		(v\i-3) -- (v\i-4) (v\i-1) edge [bend left] (v\i-4)
		(v\i-1) -- (v\i-t1) (v\i-1) -- (v\i-t2)
		(v\i-2) -- (v\i-t2)	(v\i-3) -- (v\i-t1)
		(v\i-4) -- (v\i-t1) (v\i-4) -- (v\i-t2)
		(v\i-t1) -- (v\i-t2)
;
\end{tikzpicture}
\end{center}
\end{figure}

\end{document}
